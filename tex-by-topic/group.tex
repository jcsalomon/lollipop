\subject[group] Grouping

\TeX\ has a grouping mechanism that is able to confine most
changes to a~particular locality. This chapter explains
what sort of actions can be local, and how groups are formed.


\invent
\item bgroup 
Implicit beginning of group character.
\item egroup 
Implicit end of group character.
\item begingroup 
Open a group that must be closed with \cs{endgroup}.
\item endgroup 
Close a group that was opened with \cs{begingroup}.
\item aftergroup 
Save the next token for insertion after the current group ends.
\item global 
Make assignments, macro definitions, and arithmetic global.
\item globaldefs 
Parameter for overriding \cs{global} prefixes.
\IniTeX\ default:~0.
\inventstop

\point The grouping mechanism

A group is a sequence of tokens starting with a
\term grouping\par
`beginning of group' token,
and ending with an `end of group'
token, and in which all such tokens are properly balanced. 

The grouping mechanism of \TeX\ is not the same as
the block structure
of ordinary programming languages.
Most languages with block structure are only able  to have
local definitions. \TeX's grouping mechanism is stronger: 
most assignments made inside a group
are local to that group unless explicitly indicated otherwise,
and outside the group old values are restored.

An example of local definitions
\Ver>{\def\a{b}}\a<Rev gives an `undefined control sequence'
message because \cs{a} is only defined inside the group.
Similarly, the code
\Ver>\count0=1 {\count0=2 } \showthe\count0<Rev
will display the value~1; the assignment made inside the group
is undone at the end of the group.


Bookkeeping of values that are to be restored outside the group
is done through the mechanism
\term save stack\par
of the `save stack'. Overflow of the save stack is treated
in Chapter~\ref[error]. The save stack is also used for
a few other purposes: in calls such as \hbox{\ver>\hbox to 100pt{...}>}
the specification \hbox{\n{to 100pt}} is put on the save
stack before a new level of grouping is opened.

In order to prevent a lot of trouble with the save stack,
\IniTeX\ does not allow dumping a format inside a group.
The \cs{end} command is allowed to occur inside a group,
but \TeX\ will give a diagnostic message about this.

The \cs{aftergroup} control sequence saves a token for
insertion after the current group. Several tokens can be
set aside by this command, and they are inserted in the left-to-right
order in which they were stated.
This is treated in Chapter~\ref[expand].


\point[global:assign] Local and global assignments

An assignment or macro definition
is usually made global by prefixing it with \cs{global},
\term local statements\par\term global statements\par
\csterm global\par\csterm globaldefs\par
but non-zero values of the \gr{integer parameter}
\cs{globaldefs} override \cs{global}
specifications: if \cs{globaldefs} is positive every assignment
is implicitly prefixed with \cs{global}, and if
\cs{globaldefs} is negative, \cs{global} is
ignored. Ordinarily this parameter is zero.

Some assignment are always global: the \gr{global assignment}s are
\description\FlushRight:no
\item \gr{font assignment}
assignments involving \cs{fontdimen}, \cs{hyphenchar}, 
and \cs{skew\-char}.
\item \gr{hyphenation assignment}
\cs{hyphenation} and \cs{patterns} commands
(see Chapter~\ref[line:break]).
\item \gr{box size assignment}
altering box dimensions with \cs{ht}, \cs{dp}, and~\cs{wd}
(see Chapter~\ref[boxes]).
\item \gr{interaction mode assignment}
run modes for a \TeX\ job (see Chapter~\ref[run]).
\item \gr{intimate assignment}
assignments to a \gr{special integer} or \gr{special dimen};
see %Chapters \ref[number] and~\ref[glue].
pages \pgref[special:int:list] and~\pgref[special:dimen:list].
\descriptionstop

\point Group delimiters

A group can be delimited by character tokens of category code~1 
\term delimiter, group~--\par
for `beginning of group' and code~2 for `end of  group', or
control sequence tokens that are \cs{let} to such characters,
the \cs{bgroup} and \cs{egroup} in plain \TeX.
Implicit and explicit braces can match to delimit
a group.

Groups can also be delimited by \cs{begingroup} and
\cs{endgroup}. These two control sequences must
\csterm begingroup\par\csterm endgroup\par
be used together: they cannot be matched with implicit
or explicit braces, nor can they function as the braces
surrounding, for instance, boxed material.

Delimiting with \cs{begingroup} and \cs{endgroup} can
\label[begin:end:macros]%
provide a limited form of run-time error checking. 
In between these two group delimiters an excess
open or close brace would result in
\Ver>\begingroup ... } ... \endgroup<Rev or
\Ver>\begingroup ... { ... \endgroup<Rev
In both cases \TeX\ gives an error message about improper
balancing. Using \cs{bgroup} and \cs{egroup} here would
make an error much harder to find, because of the incorrect
matching that would occur. This idea is used in the environment
macros of several formats.

The choice of the brace characters for the beginning and end of group
characters is not hard-wired in \TeX. It is arranged
\csterm bgroup\par\csterm egroup\par
like this in the plain format:
\Ver>\catcode`\{=1 % left brace is begin-group character
\catcode`\}=2 % right brace is end-group character<Rev
Implicit braces have also been defined in the plain format:
\Ver>\let\bgroup={ \let\egroup=}<Rev

Special cases are the following:
\itemlist \item The replacement text of a macro must be enclosed
in  explicit beginning and end of group character tokens.
\item  The open and close braces for boxes, \cs{vadjust},
and \cs{insert} can be implicit. This makes it possible
to define, for instance
\Ver>\def\openbox#1{\setbox#1=\hbox\bgroup}
\def\closebox#1{\egroup\box#1}
\openbox{15}Foo bar\closebox{15}<Rev
\item The right-hand side of a token list assignment and the
argument of the commands \cs{write}, \cs{message}, \cs{errmessage}, 
\cs{uppercase}, \cs{lowercase}, 
\cs{special}, and \cs{mark} is a \gr{general text}, defined
as
\Disp \gr{general text} $\longrightarrow$ \gr{filler}\lb
      \gr{balanced text}\gr{right brace}\Dispstop
meaning that the left brace can be implicit, but the closing
right brace must be an explicit character token with category
code~2. \itemliststop

In cases where an implicit left brace suffices, and where
expansion is not explicitly inhibited, \TeX\ will
expand tokens until a left brace is encountered. This
is the basis for such constructs as
\ver=\uppercase\expandafter{\romannumeral80}=,
which in this unexpanded form do not adhere to the
syntax. If the first unexpandable token is not a left
brace \TeX\ gives an error message.

The grammar of \TeX\ (see Chapter~\ref[gramm])  uses
\gr{left brace} and \gr{right brace} for explicit
characters, that is, character tokens,
and \n{\lb} and~\n{\rb} 
for possibly implicit characters,
\altt
that is, control sequences that have been \cs{let} to such
explicit characters.

\point More about braces


\spoint Brace counters

\TeX\ has two counters for keeping  track of grouping levels:
\term braces\par
the {\it master counter} and the {\it balance counter}.
Both of these counters are syntactic counters: they count the
explicit brace character tokens, but are not affected by implicit
braces (such as \cs{bgroup}) that are semantically equivalent
to an explicit brace.

The balance counter handles braces in all cases except in
alignment. Its workings are intuitively clear: it goes up
by one for every opening and down for every closing
brace that is not being skipped. Thus
\Ver>\iffalse{\fi<Rev increases the balance counter if
this statement is merely scanned (for instance if it
appears in a macro definition text); if this statement
is executed the brace is skipped, so there is no effect on
the balance counter.

The master counter is more tricky;
it is used in alignments instead of the balance counter.
This counter records all braces, even when they are skipped
such as in \ver>\iffalse{\fi>.
For this counter uncounted skipped braces are still possible:
the alphabetic constants \n{`\lb} and \n{`\rb} have
no effect on this counter when they are
use by the execution processor as a~\gr{number};
they do affect this counter when they are seen by the 
input processor (which merely sees characters, and not
the context).

\spoint The brace as a token

Explicit braces are character tokens, and as such they are
unexpandable. This implies that they survive until the
last stages of \TeX\ processing. For example,
\Ver>\count255=1{2}<Rev will assign~1 to \cs{count255},
and print~`2', because the
opening brace functions as a delimiter for the number~1.
Similarly \Ver>f{f}<Rev will prevent \TeX\ from forming
an `\hbox{ff}' ligature.

From the fact that braces are unexpandable,
it follows that their nesting is independent
of the nesting of conditionals. For instance
\Ver>\iftrue{\else}\fi<Rev will give an open brace,
as conditionals are handled by expansion. The closing
brace is simply skipped as part of the \gr{false text};
any consequences it has for grouping only come into
play in a later stage of \TeX\ processing.

Undelimited macro arguments are either single tokens
or groups of tokens enclosed in explicit braces.
Thus it is not possible for an explicit open or close brace
to be a macro argument. However, braces can be assigned
with \cs{let}, for instance as in \Ver>\let\bgroup={<Rev
This is used in the plain \cs{footnote} macro
(see page~\pgref[footnote:ex]).

\spoint \csc{\char 123} and \csc{\char 125}

The control sequences \ver-\{- and \ver-\}- do not really belong
\csterm\char123\par\csterm\char125\par
in this chapter,  not being concerned with grouping.
They have been defined with \cs{let} as synonyms of
\cs{lbrace} and \cs{rbrace} respectively,
and these control sequences are \cs{delimiter} instructions
(see Chapter~\ref[mathchar]).

The Computer Modern Roman font has no braces, but there are
braces in the typewriter font, and for mathematics 
there are braces of different sizes \ldash and extendable ones \rdash in
the extension font.

\endinput
