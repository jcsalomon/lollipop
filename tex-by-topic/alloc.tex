\subject[alloc] Allocation

\TeX\ has registers of a number of types. For some of these,
explicit commands exist to define a synonym for a certain register;
for all of them macros exist in the plain format
to allocate an unused register. This chapter treats
the synonym and allocation commands, and discusses
some guidelines for macro writers regarding allocation.

\invent
\item countdef 
      Define a synonym for a \cs{count} register.
\item dimendef
      Define a synonym for a \cs{dimen} register.
\item muskipdef
      Define a synonym for a \cs{muskip} register.
\item skipdef 
      Define a synonym for a \cs{skip} register.
\item toksdef 
      Define a synonym for a \cs{toks} register.
\item newbox
      Allocate an unused \cs{box} register.
\item newcount
      Allocate an unused \cs{count} register.
\item newdimen
      Allocate an unused \cs{dimen} register.
\item newfam
      Allocate an unused math family.
\item newinsert
      Allocate an unused insertion class.
\item newlanguage
      (\TeX3 only)
      Allocate a new language number.
\item newmuskip
      Allocate an unused \cs{muskip} register.
\item newskip
      Allocate an unused \cs{skip} register.
\item newtoks
      Allocate an unused \cs{toks} register.
\item newread
      Allocate an unused input stream.
\item newwrite
      Allocate an unused output stream.
\>

\point Allocation commands

In plain \TeX, \cs{new...} macros are defined for
allocation of registers.
The registers of \TeX\ fall into two classes that are 
\term registers, allocation of\par
allocated in different ways. This is treated below.

The \cs{newlanguage} macro of plain \TeX\ 
\csterm newlanguage\par
does not allocate any register. Instead it merely assigns
a number, starting from~0.
\TeX\ (version~3) can have at most 256 different
sets of hyphenation patterns.

The \cs{new...} macros of plain \TeX\ are defined to be
\cs{outer} (see Chapter~\ref[macro] for a precise explanation),
which precludes use of the allocation macros in other macros.
Therefore the \LaTeX\ format redefines these macros
without the \cs{outer} prefix.

\spoint \cs{count}, \cs{dimen}, \cs{skip}, \cs{muskip}, \cs{toks}

For these registers there exists a \gr{registerdef} command,
for instance \cs{countdef}, to couple a specific register
to a control sequence:
\Disp\gr{registerdef}\gr{control 
    sequence}\gr{equals}\gr{8-bit number}\Dispstop

After the definition \Ver>\countdef\MyCount=42<Rev
the allocated register can be used as
\Ver>\MyCount=314<Rev or \Ver>\vskip\MyCount\baselineskip<Rev

The \gr{registerdef} commands are used in plain \TeX\ macros
\cs{newcount} et cetera that allocate an unused register;
after\Ver>\newcount\MyCount<Rev \cs{MyCount} can be used
exactly as in the above two examples.

\spoint \cs{box}, \cs{fam}, \cs{write}, \cs{read}, \cs{insert}

For these registers there exists no  \gr{registerdef} command in \TeX,
so \cs{chardef} is used to allocate box registers
in the corresponding plain \TeX\ macros \cs{newbox}, for instance.

The fact that \cs{chardef} is used implies that the
defined control sequence does not stand for the register itself,
but only for its number. Thus after \Ver>\newbox\MyBox<Rev
it is necessary to write \Ver>\box\MyBox<Rev 
Leaving out the \cs{box} means that the character
in the current font with number
\cs{MyBox} is typeset. The \cs{chardef} command
is treated further in Chapter~\ref[char].

\point Ground rules for macro writers

The \cs{new...} macros of plain \TeX\ have been designed
to form a foundation for macro packages, such that
several of such packages can operate without collisions
in the same run of \TeX. In appendix~B of \TeXbook\
Knuth formulates some ground rules that macro writers should
adhere to.
\enumerate
\item The \cs{new...} macros do not allocate registers
with numbers~0--9. These can therefore be used as `scratch'
registers. However, as any macro family can use them,
no assumption can be made about the permanency of their
contents. Results that are to be passed from one call to
another should reside in specifically allocated registers.

Note that count registers 0--9 are used for page identification
in the \n{dvi} file (see Chapter~\ref[TeXcomm]), so no global assignments
to these should be made.

\item \cs{count255}, \cs{dimen255}, and \cs{skip255} are
also available. This is because inserts are
allocated from 254 downward  and, together with an insertion box,
a count, dimen, and skip register, 
all with the same number, are allocated.
Since \cs{box255} is used by the output routine 
(see Chapter~\ref[output]),
the count, dimen, and skip with number~255 are freely available.

\item Assignments to scratch registers~0, 2, 4, 6, 8, and~255
should be local; assignments to registers~1, 3, 5, 7,~9
should be \cs{global} (with the exception of the \cs{count}
registers). This guideline prevents `save
stack build-up' (see Chapter~\ref[error]).

\item Any register can be used inside a group, as \TeX's
grouping mechanism will restore its value outside
the group. There are two conditions on this use of
a register:
no global assignments should be made to it, and 
it must not be possible that other macros may be
activated in that group that perform global assignments
to that register.

\item Registers that are used over longer periods of time,
or that have to survive in between calls of different
macros, should be allocated by \cs{new...}.
\enumeratestop


\endinput
