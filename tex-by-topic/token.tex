\subject[token] Token Lists

\TeX\ has only one type of data structure: the token list.
\term token lists\par
There are token list registers that are available to the user,
and \TeX\ has some special token lists: the \cs{every...}
variables, \cs{errhelp}, and \cs{output}.


\invent
\item toks 
      Prefix for a token list register.

\item toksdef 
      Define a control sequence to be a synonym for
      a~\cs{toks} register.

\item newtoks 
      Macro that allocates a token list register.

\inventstop

\point Token lists

Token lists are the only type of data structure that \TeX\ knows.
They can contain character tokens and control sequence tokens.
Spaces in a token list are significant.
The only operations on token lists are assignment and
unpacking.

\TeX\ has 256 token list registers \ver|\toks|$nnn$ that can be
allocated using the macro \ver|\newtoks|, or explicitly
assigned by \cs{toksdef}; see below.

\point Use of token lists

Token lists are  assigned by a \gr{variable assignment},
which is in this case takes one of the forms
\disp\gr{token variable}\gr{equals}\gr{general text}\nl
     \gr{token variable}\gr{equals}\gr{filler}\gr{token variable}\dispstop
Here a \gr{token variable} is an explicit \cs{toks}$nnn$
register, something that has been defined to such a register
by \cs{toksdef} (probably hidden in \cs{newtoks}),
or one of the special \gr{token parameter}
lists below.
A~\gr{general text} has an explicit closing brace, but the
open brace can be implicit.

Examples of token lists are (the first two lines are equivalent):
\Ver>\toks0=\bgroup \a \b cd}
\toks0={\a \b cd}
\toks1=\toks2<Rev

Unpacking a token list is done by the command \cs{the}:
the expansion of \cs{the}\gr{token variable} is the 
sequence of tokens that was in the token list.

Token lists have a special behaviour in \cs{edef}:
when prefixed by \ver|\the| they are unpacked, 
but the resulting tokens
are not evaluated further. Thus
\Ver>\toks0={\a \b} \edef\SomeCs{\the\toks0}<Rev
gives
\Ver>\SomeCs: macro:-> \a \b<Rev
This is in contrast to what happens ordinarily in an~\cs{edef};
see page~\pgref[expand:edef].


\point \gr{token parameter}

There are in \TeX\ a number of token lists that are automatically
inserted at certain points. These \gr{token parameter}s are
the following:
\description \item \cs{output}
   this token list is inserted
   whenever \TeX\ decides it has sufficient material for a page,
   or when the user forces activation by a penalty~$\leq-10\,000$
   in vertical mode
   (see Chapter~\ref[output]);
\item \cs{everypar}
   is inserted when \TeX\ switches from external or internal
   vertical mode to unrestricted horizontal mode 
   (see Chapter~\ref[par:start]);
\item \cs{everymath}
   is inserted after a single math-shift character that starts
   a formula;
\item \cs{everydisplay}
   is inserted after a double math-shift character that starts
   a display formula;
\item \cs{everyhbox}
   is inserted when an \cs{hbox} begins (see Chapter~\ref[boxes]);
\item \cs{everyvbox}
   is inserted when a vertical box begins (see Chapter~\ref[boxes]);
\item \cs{everyjob}
   is inserted when a job begins (see Chapter~\ref[run]);
\item \cs{everycr}
   is inserted in alignments after \cs{cr} or a non-redundant
   \cs{crcr} (see Chapter~\ref[align]);
\item \cs{errhelp}
   contains tokens to supplement an \cs{errmessage}
    (see Chapter~\ref[error]).
\descriptionstop

A \gr{token parameter} behaves the same as an explicit \cs{toks}$nnn$
list, or a quantity defined by \cs{toksdef}.

\point Token list registers

Token lists can be stored in \cs{toks} registers:
   \csterm toks\par
\Disp\cs{toks}\gr{8-bit number}\Dispstop
which is a \gr{token variable}.
Synonyms for token list registers can be made by the \gr{registerdef}
command \cs{toksdef} in a \gr{shorthand definition}:
   \csterm toksdef\par
\Disp\cs{toksdef}\gr{control sequence}\gr{equals}\gr{8-bit number}
\Dispstop A control sequence defined this way is called
a \gr{toksdef token}, and this is also a token variable
(the remaining third kind of token variable is
the \gr{token parameter}).

The plain \TeX\ macro \cs{newtoks} uses \cs{toksdef} to 
   \csterm newtoks\par
allocate unused token list registers. This macro is \cs{outer}.

\point Examples

Token lists are probably among the least obvious components
of \TeX: most \TeX\ users will never find occasion  for their use,
but format designers and other macro writers
can find interesting applications.
Following are some examples of the sorts of things that can be
done with token lists.

\spoint Operations on token lists: stack macros

The number of primitive operations available for token lists is
\howto Stack macros\par
rather limited: assignment and unpacking. However, these are
sufficient to implement other operations such as appending.

Let us say we have allocated a token register
\Ver>\newtoks\list \list={\c} <Rev
and we want to add tokens to it,
\alt
using the syntax
\Ver>\Prepend \a \b (to:)\list<Rev
such that \Ver>\showthe\list<Rev gives \Ver>> \a \b \c .<Rev
For this the original list has to be unpacked, and 
\alt
the new tokens followed by the old contents have to assigned
again to the register. Unpacking can be done with \cs{the}
inside an \cs{edef}, so we arrive at the following macro:
\Ver>
\def\Prepend#1(to:)#2{\toks0={#1}%
    \edef\act{\noexpand#2={\the\toks0 \the#2}}%
    \act}<Rev
Note that the tokens that are to be added are first packed
\alt
into a temporary token list, which is then again unpacked
inside the \cs{edef}. Including them directly would have
led to their expansion.

Next we want to use token lists as a sort of stack:
we want a `pop' operation that removes the first element
from the list. Specifically,
\Ver>\Pop\list(into:)\first
\show\first \showthe\list<Rev
should give
\Ver>
> \first=macro:
->\a .<Rev
and for the remaining list
\Ver> 
> \b \c .<Rev

Here we make creative use of delimited and undelimited
parameters. With an \cs{edef} we unpack the list,
and the auxiliary macro \cs{SplitOff} scoops up the elements
as one undelimited argument, the first element, and one
delimited argument, the rest of the elements.\Ver>
\def\Pop#1(into:)#2{%
    \edef\act{\noexpand\SplitOff\the#1%
              (head:)\noexpand#2(tail:)\noexpand#1}%
    \act}
\def\SplitOff#1#2(head:)#3(tail:)#4{\def#3{#1}#4={#2}}<Rev

\spoint Executing token lists

The \cs{the} operation for unpacking token lists was used above
only inside an \cs{edef}. Used on its own it has the effect
of feeding the tokens of the list to \TeX's expansion mechanism.
If the tokens have been added to the list in a uniform syntax,
this gives rise to some interesting possibilities.

Imagine that we are implementing the bookkeeping of external
files for a format. Such external files can be used for
table of contents, list of figures, et cetera.
If the presence
of such objects is under the control of the user, we need some
general routines for opening and closing files, and keeping
track of what files we have opened at the user's request.

Here only  some routines for bookkeeping will be described.
Let us say there is a list of auxiliary files, and an auxiliary
counter: \Ver>
\newtoks\auxlist \newcount\auxcount<Rev
First of all there must be an operation to add auxiliary files:
\Ver>\def\NewAuxFile#1{\AddToAuxList{#1}%
    % plus other actions
    }
\def\AddToAuxList#1{\let\\=\relax
    \edef\act{\noexpand\auxlist={\the\auxlist \\{#1}}}%
    \act}<Rev
This adds the name to the list in a uniform format:
\Ver>\NewAuxFile{toc} \NewAuxFile{lof}
\showthe\auxlist
> \\{toc}\\{lof}.<Rev
using the control sequence \ver>\\> which is left undefined.

Now this control sequence can be used for instance to
count the number of elements in the list:\Ver>
\def\ComputeLengthOfAuxList{\auxcount=0 
    \def\\##1{\advance\auxcount1\relax}%
    \the\auxlist}
\ComputeLengthOfAuxList \showthe\auxcount
> 2.<Rev
Another use of this structure is the following:
at the end of the job we can now close all auxiliary
files at once, by\Ver>
\def\CloseAuxFiles{\def\\##1{\CloseAuxFile{##1}}%
    \the\auxlist}
\def\CloseAuxFile#1{\message{closing file: #1. }%
    % plus other actions
    }
\CloseAuxFiles<Rev
which gives the output
\Ver>closing file: toc.  closing file: lof. <Rev

\endinput
\spoint Dynamic macro definition

Unpacking token lists inside an \cs{edef} can be put to a
rather ambitious use: dynamic definition of macros.
Consider a simple example.
\altt
We set ourselves the goal of letting
the user define macros, without ever having to use \cs{def}.
The syntax for this could look like\Ver>
\startdefinition
\do:this
\do:that
\define:MyMacro<Rev
such that \ver>\show\MyMacro> gives \Ver>
> \MyMacro=macro:
->\this \that .<Rev
An implementation of this uses a token list to collect
the commands that the user specifies:\Ver>
\newtoks\actionlist<Rev
The first command is easy:\Ver>
\def\startdefinition{\actionlist{}}<Rev
Now the \cs{do} command has to hang control sequences
in the \cs{actionlist}:\Ver>
\def\do:#1 {%
    \edef\act{\noexpand\appendaction
              \expandafter\noexpand\csname#1\endcsname}%
    \act}<Rev
The \cs{edef} is used solely to form the actual control sequence.
The next macro uses \cs{edef} to unpack the \cs{actionlist} so far:
\Ver>
\def\appendaction#1{%
    \edef\act{\noexpand\actionlist=
                 {\the\actionlist \noexpand#1}}%
    \act}<Rev
Finally, definition of the user macro also needs an \cs{edef}.
Some \cs{expandafter} trickery is necessary here to form
the control sequence of the user macro:\Ver>
\def\define:#1 {%
    \expandafter\edef\csname#1\endcsname{\the\actionlist}}<Rev

Of course, this is a very simple, rather pointless, example.
However, it illustrates an important principle of how
token lists can be used to implement another syntax level
in \TeX\ (see~\cite[EL]). This principle underlies the
\term Lollipop\par
`Lollipop' format that was used to typeset this book.

\endinput

