\subject[rules]  Rules and Leaders

Rules and leaders are two ways of getting \TeX\ to draw a line.
Leaders are more general than rules: they can also fill
available space with copies of a certain box. This chapter
explain how rules and leaders work, and how they interact with modes.


\invent
\item hrule 
      Rule that spreads in horizontal direction.

\item vrule 
      Rule that spreads in vertical direction.

\item leaders 
      Fill a specified amount of space with a rule or copies of box.

\item cleaders 
      Like \ver=\leaders=, but with box leaders 
      any excess space is split equally before and after the leaders.

\item xleaders 
      Like \ver=\leaders=, but with box leaders any excess space is 
      spread equally before, after, and between the boxes.

\inventstop

\point Rules

\TeX's rule commands give
\term rules\par
rectangular black patches with horizontal and vertical sides.
Most of the times, a rule command will give output that
looks like a rule, but~\hbox{\vrule height 1.5ex width 1.5ex}
can also be produced by a rule.

\TeX\ has both horizontal and vertical rules, 
but the names do not necessarily imply anything about the shape.
\csterm hrule\par\csterm vrule\par
They do, however, imply something about modes:
an \cs{hrule} command can only be used in vertical mode,
and a \cs{vrule} only in horizontal mode.
In fact, an \cs{hrule} is a \gr{vertical command}, and a \cs{vrule}
is a \gr{horizontal command}, so \TeX\ may change
modes when encountering these commands.

Why then is a \cs{vrule} called a {\em vertical\/} rule?
The reason is that a \cs{vrule} can expand arbitrarily
far in the vertical direction: if its height and depth are not
specified explicitly it will take as much room as its
surroundings allow\altt.

\example \Ver>\hbox{\vrule\ text \vrule}<Rev
looks like \disp\leavevmode\hbox{\vrule\ text \vrule}\dispstop
and \Ver>\hbox{\vrule\ A gogo! \vrule}<Rev looks like
\disp\leavevmode\hbox{\vrule\ A gogo! \vrule}\dispstop
\>

For the \cs{hrule} command a similar statement is true:
a horizontal rule can spread to assume the width of
its surroundings. Thus 
\Ver>\vbox{\hbox{One line of text}\hrule}<Rev looks like
\disp\leavevmode\vtop{\hbox{One line of text}\hrule}\dispstop


\spoint Rule dimensions

Horizontal and vertical rules have a default thickness:
\Disp \cs{hrule}\quad is the same as\quad \ver-\hrule height.4pt depth0pt-
\Dispstop
and 
\Disp \cs{vrule}\quad is the same as\quad \ver-\vrule width.4pt- \Dispstop
and if the remaining dimension remains unspecified, the rule
extends in that direction to fill the enclosing box.

Here is the formal specification of how to indicate rule sizes:
\disp\gr{vertical rule} $\longrightarrow$ 
                        \cs{vrule}\gr{rule specification}\nl
     \gr{horizontal rule} $\longrightarrow$
                        \cs{hrule}\gr{rule specification}\nl
     \gr{rule specification} $\longrightarrow$
                        \gr{optional spaces} \nl \indent$|$
                        \gr{rule dimensions}\gr{rule specification}\nl
     \gr{rule dimension} $\longrightarrow$
                        \n{width}\gr{dimen} $|$ \n{height}\gr{dimen} $|$
                        \n{depth}\gr{dimen}
     \dispstop
If a rule dimension is specified twice, the second instance
takes precedence over the first. This makes it possible
to override the default dimensions. For instance,
after
\alt
\howto Change the default dimensions of rules\par
\Ver>\let\xhrule\hrule  \def\hrule{\xhrule height .8pt}<Rev
the macro \cs{hrule} gives a horizontal rule
of double the original height, and it is still possible
with \Ver>\hrule height 2pt<Rev
to specify other heights.

It is possible to specify all three dimensions; then
\Ver>\vrule height1ex depth0pt width1ex<Rev and
\Ver>\hrule height1ex depth0pt width1ex<Rev look the same.
Still, each of them can be used only in the appropriate mode.

\point Leaders

Rules are intimately connected to modes, which makes it easy
\term leaders\par
to obtain some effects. For instance, a typical application
of a vertical rule looks like
\Ver>\hbox{\vrule width1pt\ Important text! \vrule width 1pt}<Rev
which gives
\disp\leavevmode\hbox{\vrule width 1pt\ Important text! 
                      \vrule width 1pt}\>
However, one might want to have a horizontal rule
in horizontal mode for effects such as
\disp\leavevmode
\vbox{\hbox to 5cm{$\longleftarrow$\hfil 5cm\hfil$\longrightarrow$}
    \hbox to 5cm{from here\leaders\hrule\hfil to there}}\dispstop
An \cs{hrule} can not be used in horizontal mode, and
a vertical rule will not spread automatically.

However, there is a way to use an \cs{hrule} command in
horizontal mode and a \cs{vrule} in vertical mode,
and that is with `leaders', so called because
they lead your eye across the page. 
A~leader command tells \TeX\
to fill a~specified space, in whatever mode it is in,
with as many copies of some box or rule specification
as are needed. For instance, the above example
was given as
\disp\ver>\hbox to 5cm{from here\leaders\hrule\hfil to there}>\dispstop
that is, with an \cs{hrule} that was allowed to stretch along
an \cs{hfil}.
Note that the leader was given a horizontal skip,
corresponding to the horizontal mode in which it appeared.

A general leader command looks like
\Disp \gr{leaders}\gr{box or rule}%
      \gr{vertical/horizontal/mathematical skip}\Dispstop
where \gr{leaders} is \cs{leaders}, \cs{cleaders}, 
or~\cs{xleaders}, a \gr{box~or~rule}
is a~\gr{box}, \cs{vrule}, or~\cs{hrule}, and the
lists of horizontal and vertical skips appear in Chapter~\ref[hvmode];
a~mathematical skip is either a horizontal skip or an~\cs{mskip}
(see page~\pgref[muglue]).
Leaders can thus be used in all three modes. Of course, the
appropriate kind of skip must be specified. 

A horizontal (vertical) box containing leaders has at least
the height and depth (width) of the \gr{box~or~rule} used
in the leaders, even if, as can happen in the case of box leaders,
no actual leaders are placed.

\spoint Rule leaders

Rule leaders fill the specified amount of space with a rule
\term rule leaders\par\csterm leaders\par
extending in the direction of the skip specified.
The other dimensions of the resulting rule leader
are determined by the sort of rule that is used:
either dimensions can be specified explicitly, or
the default values can be used.

For instance, 
\Ver>\hbox{g\leaders\hrule\hskip20pt f}<Rev
gives \disp\leavevmode\hbox{g\leaders\hrule\hskip20pt f}\dispstop
because a horizontal rule has a default height of~\n{.4pt}.
On the other hand,
\Ver>\hbox{g\leaders\vrule\hskip20pt f}<Rev
gives \disp\leavevmode\hbox{g\leaders\vrule\hskip20pt f}\dispstop
because the height and depth of a vertical rule
by default fill the surrounding box.

Spurious rule dimensions are ignored: in horizontal mode
\Ver>\leaders\hrule width 10pt \hskip 20pt<Rev
is equivalent to
\Ver>\leaders\hrule \hskip 20pt<Rev

If the width or height-plus-depth
of either the skip or the box is negative, 
\TeX\ uses ordinary glue instead of leaders.

\spoint Box leaders

Box leaders fill the available spaces with copies of
a given box, instead of with a rule. 

\newbox\centerdot  \setbox\centerdot=\hbox{\hskip.7em.\hskip.7em}

For all of the following examples, assume that a box register
has been allocated:
\Ver>\newbox\centerdot  \setbox\centerdot=\hbox{\hskip.7em.\hskip.7em}<Rev
Now the output of
\Ver>\hbox to 8cm {here\leaders\copy\centerdot\hfil there}<Rev
is 
\disp\leavevmode\hbox to 8cm {here\leaders\copy\centerdot\hfil there}
\dispstop That is, copies of the box register fill up the
available space.

Dot leaders, as in the above example, are often used for
tables of contents. In such applications it is desirable that
dots on subsequent lines are vertically aligned.
The \cs{leaders} command does this automatically:
\Ver>\hbox to 8cm {here\leaders\copy\centerdot\hfil there}
\hbox to 8cm {over here\leaders\copy\centerdot\hfil over there}<Rev
gives \disp\leavevmode
\vtop{\hbox to 8cm {here\leaders\copy\centerdot\hfil there}
\hbox to 8cm {over here\leaders\copy\centerdot\hfil over there\strut}}
\dispstop
The mechanism behind this is the following:
\TeX\ acts as if an infinite row of boxes starts (invisibly) at 
the left edge of the surrounding box, 
and the row of copies actually placed is 
merely the part of this row that is not obscured by
the other contents of the box.

Stated differently, box leaders are a window on an infinite
row of boxes, and the row starts at the left edge of the
surrounding box. Consider the following example:
\Ver>\hbox to 8cm {\leaders\copy\centerdot\hfil}
\hbox to 8cm {word\leaders\copy\centerdot\hfil}<Rev
which gives
\disp\leavevmode\vtop{\hbox to 8cm {\leaders\copy\centerdot\hfil}
\hbox to 8cm {word\leaders\copy\centerdot\hfil\strut}}\dispstop
The row of leaders boxes becomes visible as soon as it
does not coincide with other material.

The above discussion only talked about leaders in horizontal
mode. Leaders can equally well be placed in vertical mode;
for box leaders the `infinite row' then starts at the top
of the surrounding box.

\spoint Evenly spaced leaders

Aligning subsequent box leaders in the way described above
means that the white space before and after the
leaders will in general be different.
If vertical alignment is not
an issue it may be aesthetically more pleasing to have
the leaders evenly spaced.
The \cs{cleaders} command is like \cs{leaders},
\csterm cleaders\par
except that it splits excess space before and after the leaders
into two equal parts, centring the row of boxes in the
available space.

\example\message{check verbatim indentation}
\Ver>\hbox to 7.8cm {here\cleaders\copy\centerdot\hfil there}
\hbox to 7.8cm {here is\cleaders\copy\centerdot\hfil there}<Rev
gives \disp\leavevmode\vbox{
\hbox to 7.8cm {here\cleaders\copy\centerdot\hfil there}
\hbox to 7.8cm {here is\cleaders\copy\centerdot\hfil there\strut}
}\dispstop
The `expanding leaders' \cs{xleaders} spread excess space evenly
\csterm xleaders\par
between the boxes, with equal globs of glue before, after,
and in between leader boxes.
\>

\example \Ver>
\hbox to 7.8cm{here\hskip.7em
      \xleaders\copy\centerdot\hfil  \hskip.7em there}<Rev
gives \disp\leavevmode
\hbox to 7.8cm {here\hskip.7em\xleaders\copy\centerdot\hfil\hskip.7em there}
\dispstop Note that the glue in the leader box is balanced here
with explicit glue before and after the leaders;
leaving out these glue items, as in\Ver>
\hbox to 7.8cm {here\xleaders\copy\centerdot\hfil there}<Rev
gives \disp\leavevmode
\hbox to 7.8cm {here\xleaders\copy\centerdot\hfil there}
\dispstop
which is clearly not what was intended.
\>

\point Assorted remarks

\spoint Rules and modes

Above it was explained how rules can only occur in the 
appropriate modes. Rules also influence mode-specific
quantities:
no baselineskip is added before rules in 
vertical mode. In order to prevent glue after rules,
\TeX\ sets \cs{prevdepth} to
\n{\hbox{-}1000pt}
(see Chapter~\ref[baseline]).
Similarly the \cs{spacefactor} is set to 1000 after a \cs{vrule}
in horizontal mode (see Chapter~\ref[line:break]).


\spoint[par:leaders:end] Ending a paragraph with leaders

An attempt to simulate an \cs{hrule} at the end of a paragraph by
\howto End a paragraph with leaders\par
\Ver>\nobreak\leaders\hrule\hfill\par<Rev
does not work. The reason for this is that \TeX\
performs an \cs{unskip} at the end of a paragraph,
which removes the leaders. Normally this \cs{unskip} removes
any space token inserted by the input processor after the
last line. Remedy: stick an \ver.\hbox{}. at the end of
the leaders.

\spoint Leaders and box registers

In the above examples the leader box was inserted with
\cs{copy}. The output of
\Ver>\hbox to 8cm {here\leaders\box\centerdot\hfil there}
\hbox to 8cm {over here\leaders\box\centerdot\hfil 
                   over there}<Rev
is
\disp\leavevmode
     \vtop{\hbox to 8cm {here\leaders\box\centerdot\hfil there}
           \hbox to 8cm {over here\leaders\box\centerdot\hfil over there}
           }\dispstop
The box register is emptied after the first leader command,
but more than one copy is placed in that first command.

\spoint Output in leader boxes

Any \cs{write}, \cs{openout}, or \cs{closeout} operation
appearing in leader boxes is ignored. 
Otherwise such an operation would be executed once for every
copy of the box that would be shipped out.

\spoint Box leaders in trace output

The dumped box representation obtained from,
for instance, \cs{tracingoutput}
does not write out box leaders in full: only the total size and
one copy of the box used are dumped. In particular,
the surrounding white space before and after the leaders
is not indicated.

\spoint Leaders and shifted margins

If margins have been shifted,
leaders may look different
depending on how the shift has been realized.
For an illustration of how \cs{hangindent} and \cs{leftskip}
influence the look of leaders, consider the following
examples, where
\Ver>\setbox0=\hbox{K o }<Rev
The horizontal boxes above  the leaders
\altt
serve to indicate the starting point of the row of leaders.

First
\Ver>\hbox{\leaders\copy0\hskip5cm}
\noindent\advance\leftskip 1em
      \leaders\copy0\hskip5cm\hbox{}\par<Rev
gives\message{examples on}
\disp\leavevmode\vbox{\leftskip=0pt \hsize=7cm
\setbox0=\hbox{K o }
\hbox{\leaders\copy0\hskip5cm}
\noindent\advance\leftskip 1em
      \leaders\copy0\hskip5cm\hbox{}\par
    }\>
Then
\Ver>\hbox{\kern1em\hbox{\leaders\copy0\hskip5cm}}
\hangindent=1em \hangafter=-1 \noindent
      \leaders\copy0\hskip5cm\hbox{}\par<Rev
gives (note the shift with respect to the previous example)
\disp\leavevmode\vbox{\leftskip=0pt \hsize=7cm
\setbox0=\hbox{K o }
\hbox{\kern1em\hbox{\leaders\copy0\hskip5cm}}
\hangindent=1em \hangafter=-1 \noindent
      \leaders\copy0\hskip5cm\hbox{}\par}\dispstop
\message{one page}
In the first paragraph the \cs{leftskip} glue only obscures
the first leader box; in the second paragraph the hanging
indentation actually shifts the orientation point for the 
row of leaders. Hanging indentation is performed in \TeX\
by a \cs{moveright} of the boxes containing the lines
of the paragraph.

\endinput

