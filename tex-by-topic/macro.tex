\subject[macro]  Macros

Macros are \TeX's abbreviation mechanism for sequences of commands
that are needed more than once,
somewhat like procedures in ordinary programming languages.
\TeX's parameter mechanism, however, is quite unusual.
This chapter explains how \TeX\ macros work. It also
treats the commands \cs{let} and~\cs{futurelet}.

\invent
\item def 
      Start a macro definition.

\item gdef 
      Synonym for \ver-\global\def-.

\item edef 
      Start a macro definition; 
      the replacement text is expanded at definition time.
      This command is treated also in the next chapter.

\item xdef 
      Synonym for \ver-\global\edef-.

\item csname 
      Start forming the name of a control sequence.

\item endcsname 
      Stop forming the name of a control sequence.

\item global 
      Make the next definition, arithmetic statement,
      or assignment global.

\item outer 
      Prefix indicating that the macro being defined 
      can be used on the `outer' level only.

\item long 
      Prefix indicating that the arguments of the macro being defined
      may contain \cs{par} tokens.

\item let 
      Define a control sequence to be equivalent to the next token.

\item futurelet 
      Define a control sequence to be equivalent to
      the token after the next token.

\inventstop

\point Introduction

A macro is basically a sequence of tokens that has
\term macros\par
been abbreviated into a control sequence.
Statements starting with (among others) \cs{def}
are called {\italic macro definitions}\alt, and
writing \Ver>\def\abc{\de f\g}<Rev defines the macro \cs{abc},
with the {\italic replacement text\/} \ver>\de f\g>.
Macros can be used in this way to abbreviate
pieces of text or sequences of commands
that have to be given more than once.
Any time that \TeX's expansion processor
encounters the control sequence \cs{abc},
it replaces it by the replacement text.

If a macro should be sensitive to the context
where it is used, it can be defined with parameters:
\Ver>\def\PickTwo#1#2{(#1,#2)}<Rev
takes two arguments and reproduces them in parentheses.
The call \cs{PickTwo 12} gives `(1,2)'.

The activity of substituting the replacement text
for a macro is called {\italic macro expansion}.

\point Layout of a macro definition

A macro definition consists of, in sequence,
\term macro definition\par
\enumerate \item any number of \cs{global},
\cs{long}, and \cs{outer} prefixes,
\item a \gr{def} control sequence, or anything
that has been \cs{let} to one,
\item a control sequence or active character to be defined, 
\item possibly a \gr{parameter text} specifying among other things
how many parameters the macro has, and
\item a replacement text enclosed in explicit character tokens
with category codes 1 and~2, by default \ver-{- and~\ver-}-
in plain \TeX.
\>

The `expanding' definitions \cs{edef} and \cs{xdef}
are treated in Chapter~\ref[expand].

\point Prefixes

There are three prefixes that alter the status of the
\term macro prefixes\par
macro definition: \description
\item \cs{global}
If the definition occurs inside a  group, this prefix
makes the definition global\csterm global\par. 
This prefix can also be used for assignments other than
macro definitions; in fact,
for macro definitions abbreviations exist obviating the
use of \cs{global}:
\disp\ver>\gdef\foo...>\quad is equivalent to\quad \ver>\global\def\foo...>
\dispstop and
\disp\ver>\xdef\foo...>\quad is equivalent to\quad \ver>\global\edef\foo...>
\dispstop

If the parameter \cs{globaldefs}
is positive, all assignments are
implicitly global;
if \cs{globaldefs} is negative any \cs{global} prefixes are
ignored,
and \cs{gdef} and \cs{xdef} make local definitions
(see Chapter~\ref[group]).

\item \cs{outer}
The mechanism of `outer' macros is supposed to facilitate
\term outer macros\par\csterm outer\par
locating (among other errors) unbalanced braces: an \cs{outer}
macro is supposed
to  appear only in non-embedded contexts.
To be precise, it is not allowed to occur 
\itemlist
\item in macro replacement texts (but it can appear in
    for instance \cs{edef} after 
    \cs{noexpand}, and after \cs{meaning}),
\item in parameter texts,
\item in skipped conditional text,
\item in alignment preambles, and
\item in the \gram{balanced text} of a \cs{message}, \cs{write},
et cetera. \itemliststop
For certain applications, however, it is inconvenient
that some of the plain macros  are outer, 
in particular macros such as \cs{newskip}. One remedy is to
redefine them, without the `outer' option, which
is done for instance in \LaTeX, but  cleverer tricks are possible.

\item \cs{long}
Ordinarily, macro parameters are not supposed to contain
\csterm long\par
\cs{par} tokens. This restriction is useful (much more so
than the \cs{outer} definitions) in locating
forgotten closing braces. 
For example, \TeX\ will complain about a `runaway argument'
\message{Example on}
in the following sequence:\Ver>
\def\a#1{ ... #1 ... }
\a {This sentence should be in braces.

And this is not supposed to be part of the argument<Rev
\message{one page}
The empty line generates a \cs{par}, which most of the times
means that a closing brace has been forgotten.

If arguments to a particular macro should be allowed
to contain \cs{par} tokens,  then the macro must be declared
to be \cs{long}. \descriptionstop

The \cs{ifx} test for equality of tokens 
(see Chapter~\ref[if]) takes prefixes into
account when testing whether two tokens have the same definition.

\comment
With a little ingenuity it is possible 
for \cs{par} tokens to sneak into macro arguments anyway.
Consider the example
\Ver>\def\a#1\par!{ ... }
\a bc\par ef\par!<Rev
Here the macro \cs{a} is not \cs{long}, but the argument
is \ver>bc\par ef>, which contains a \cs{par} token.
However,
this is of no importance in general.
\endcomment

\point The definition type

There are four \gr{def} control sequences in \TeX:
\csterm def\par\csterm gdef\par\csterm edef\par\csterm xdef\par
\cs{def}, \cs{gdef}, \cs{edef}, and \cs{xdef}.
The control sequence 
\alt
\cs{gdef} is a synonym for \ver>\global\def> and
\cs{xdef} is a synonym for \ver>\global\edef>.
The `expanding definition' \cs{edef} is treated in 
Chapter~\ref[expand].

The difference between the various types of macro definitions
is only relevant at the time of the definition.
When a macro is called there is no way of telling how
it was defined.

\point[param:text] The parameter text

Between the control sequence or active character to be defined
\term parameters\par\term arguments\par
and the opening brace of the replacement text, a \gr{parameter
text} can occur. This specifies whether the macro has parameters,
how many, and how they are delimited. 
The \gr{parameter text} cannot contain
explicit braces.

A macro can have at most nine parameters. 
A~parameter is indicated by a parameter token,
consisting of a macro parameter character
(that is, a character of category code~6, in plain \TeX~\ver=#=) 
followed by a digit~\n1--\n9. 
For instance, \ver>#6>~denotes the sixth parameter of a macro.
Parameter tokens cannot appear outside the context
of a macro definition.

In the parameter text,
parameters must be numbered consecutively, starting at~1.
A~space after a parameter token is significant,
both in the parameter text and the replacement text.

Parameters can be delimited or undelimited. A~parameter
is called undelimited if it is followed immediately
by another parameter in the \gr{parameter text}
or by the opening brace of the replacement text;
it is called delimited if it is followed by any other token.

The tokens (zero or more) that are substituted for
a parameter when a macro is expanded (or `called')
are called
the `argument' corresponding to that parameter.

\spoint Undelimited parameters

When a macro with an undelimited parameter, for instance
\term undelimited parameters\par
a macro \cs{foo} with one parameter
\Ver>\def\foo#1{ ... #1 ...}<Rev
is expanded, \TeX\ scans ahead (without expanding)
until a non-blank token is found.
If this token is not an explicit \gr{left brace}, 
it is taken to be the argument
corresponding to the parameter. Otherwise a \gr{balanced text}
is absorbed by scanning until the matching explicit
\gr{right brace} has been found.
This balanced text then
constitutes the argument.

An example with three undelimited parameters follows: with
\Ver>\def\foo#1#2#3{#1(#2)#3}<Rev
the macro call \cs{foo123} gives `\hbox{1(2)3}';
but \hbox{\ver-\foo 1 2 3-} also gives the same result.
In the call
\disp\cs{foo}\n{\char32 1\char32 2\char 32 3}\dispstop
the first space is skipped in the input processor of \TeX.
The argument corresponding to the first parameter is then
the~\n1. In order to find the second parameter \TeX\ then
skips all blanks, in this case exactly one. As second
parameter \TeX\ finds then the~\n2. Similarly the third
parameter is~\n3.


In order to pass several tokens as one undelimited argument
one can use braces. With the above definition of \cs{foo}
the call \ver>\foo a{bc}d> gives `\hbox{a(bc)d}'.
When the argument of a macro is a balanced text instead of
a single token, the delimiting braces are not inserted when 
the argument is
inserted in the replacement text.
For example:\Ver>\def\foo#1{\count0=1#1\relax}
\foo{23}<Rev will expand to \ver>\count0=123\relax>,
which assigns the value of 123 to the counter.
On the other hand,  the statement \Ver>\count0=1{23}<Rev would
assign~1 and print~23.

\spoint Delimited parameters

Apart from enclosing it in braces there is another way
\term delimited parameters\par
to pass a sequence of tokens as a single argument to a macro,
namely by using delimited parameters.

Any non-parameter tokens in the \gr{parameter text} occurring
after a macro parameter (that is, after the parameter number
following the parameter character)
act as a delimiter for that parameter. This includes space tokens:
a space after a parameter number is significant.
Delimiting tokens can also occur between the control
sequence being defined and the first parameter token~\ver>#1>.

Character tokens acting as delimiters in the parameter text
have both their character code and
category code stored; the delimiting character tokens of the
actual arguments have to match both.
Category codes of such characters may include some that
can normally only appear in special contexts; for instance, after
the definition \Ver>\def\foo#1_#2^{...}<Rev the macro \cs{foo}
can be used outside math mode.

When looking for the argument corresponding to
a delimited parameter, \TeX\ absorbs all tokens without expansion (but
balancing braces) until the 
(exact sequence of) delimiting tokens is encountered.
The delimiting tokens are not part of the argument;
they are removed from the input stream during the macro call.

\spoint Examples with delimited arguments

As a simple example, \Ver>\def\DoASentence#1#2.{{#1#2.}}<Rev
defines a macro with an undelimited first parameter,
and a second parameter delimited by a period.
In the call\Ver>
\DoASentence \bf This sentence is the argument.<Rev
the arguments are:
\Ver>#1<-\bf
#2<-This sentence is the argument<Rev
Note that the closing period is not in the argument, but it has
been absorbed; it is no longer in the input stream.

A~commonly used delimiter is \cs{par}:
\Ver>
\def\section#1. #2\par{\medskip\noindent {\bf#1. #2\par}}<Rev
This macro has a first parameter that is delimited by~`\n{.\char32}',
and a second parameter that is delimited by \cs{par}.
The call\message{example on one page}
\Ver>
\section 2.5. Some title

The text of the section...<Rev
will give
\disp\ver>#1<-2.5>\nl
\ver>#2<-Some title>\n{\char32}\dispstop
Note that there is a space at the end of the second argument
generated by the line end. If this space is unwanted one might
define \Ver>\def\section#1. #2 \par{...}<Rev
with \n{\char32}\cs{par} delimiting the second
argument. This approach, however,
precludes  the user's writing the \cs{par} explicitly:
\Ver>\section 2.5 Some title\par<Rev
One way out of this dilemma is to write
\ver>#2\unskip> on all places in the definition text
where the trailing space would be unwanted.

Control sequences acting as delimiters need not be defined,
as they are absorbed without expansion. Thus
\Ver>\def\control#1\sequence{...}<Rev is a useful
definition, even if \cs{sequence} is undefined.

The importance of category codes in delimited arguments
is shown by the following example:
\Ver>
\def\a#1 #2.{ ... }
\catcode`\ =12
\a b c
d.<Rev
which gives
\Ver>\a #1 #2.-> ...
#1<- b c
#2<-d<Rev
Explanation: the delimiter between parameters 1 and~2 is a space
of category~10. In between \n{a} and \n{b} there is a space
of category~12; the first space of  category~10
is the space that is generated by the line end.

For a `real-life' application of matching of category codes,
see the explanation of \cs{newif} in Chapter~\ref[if],
and the example on page~\pgref[ex:jobnumber].


\spoint Empty arguments

If the user specifies a \gr{balanced text} in braces
when \TeX\ expects a macro
argument, that text is used as the argument.
Thus, specifying \ver-{}- will give an argument that is
an empty list of tokens; this is called an `empty argument'.

Empty arguments can also arise from the use of delimited
parameters. For example, after the definition
\Ver>\def\mac#1\ro{ ... }<Rev
the call
\Ver>\mac\ro<Rev will give an empty argument. 

\comment
However, only
one empty argument can be created this way: 
if the macro had been defined as
\Ver>\def\mac#1#2\ro{ ... }<Rev the same call
\Ver>\mac\ro \othermacro \stillothermacro<Rev
will probably cause a `\n{Runaway argument?}' error message.
Explanation: the first parameter is undelimited, so the corresponding
argument is `\cs{ro}'; after that \TeX\ starts looking for a list
of tokens delimited by~\cs{ro}.
\endcomment

\spoint The macro parameter character

When \TeX's input processor scans a macro definition text, 
\term parameter character\par
it inserts a parameter token for any
occurrence of a macro parameter character followed by a digit.
In effect, a parameter token in the replacement text
states `insert parameter number such and such here'.
Two parameter characters in a row are replaced by a single one.

The latter fact can be used for nested macro definitions.
\label[nest:def]\howto Nested macro definitions\par
Thus \Ver>\def\a{\def\b#1{...}}<Rev gives an error message
because \cs{a} was defined without parameters, and
yet there is a parameter token in its replacement text.

The following
\Ver>\def\a#1{\def\b#1{...}}<Rev defines a macro \cs{a} that
defines a macro \cs{b}. However, \cs{b} still does not
have any parameters: the call
\Ver>\a z<Rev defines a macro \cs{b} without parameters,
that has to be followed by a~\n z.
Note that this
does not attempt to define a macro \cs{bz}, because the
control sequence \cs{b} has already been formed in \TeX's
input processor when that input line was read.

Finally,
\Ver>\def\a{\def\b##1{...}}<Rev defines a macro \cs{b} 
with one parameter.

Let us examine the handling of the parameter character
in some detail.
Consider \Ver>\def\a#1{ .. #1 .. \def\b##1{ ... }}<Rev
When this is read as input, the input processor
\itemlist
\item replaces the characters \ver>#1> by \gr{parameter token$_1$}, and
\item replaces the characters \ver>##> by \ver>#>\>
A macro call of \cs{a} will then let the input processor scan
\Ver>\def\b#1{ ... }<Rev
in which the two characters \ver>#1> are
\alt
replaced by a parameter token.

\spoint Brace delimiting

Ordinarily, it is not possible to have left or right
braces in the \gr{parameter text} of a definition.
There is a special mechanism, however, that can make
the last parameter of a macro act as if it is delimited
by an opening brace. 

If the last parameter token
is followed by a parameter character (\ver>#>),
which in turn is followed by the opening brace of the
replacement text, \TeX\ makes the last parameter
be delimited by a beginning-of-group character.
Furthermore, unlike other delimiting tokens in
parameter texts, this opening brace is not
removed from the input stream.

Consider an example.
Suppose we want to have a macro
\cs{every} that can fill token lists as follows:
\Ver>\every par{abc} \every display{def}<Rev
This macro can be defined as
\Ver>\def\every#1#{\csname every#1\endcsname}<Rev
In the first call above, the argument corresponding to
the parameter is \n{abc}, so the call 
expands to
\Ver>\csname everypar\endcsname{abc}<Rev
which gives the desired result.


\point[cs:name] Construction of control sequences

The commands \cs{csname} and \cs{endcsname} can be used
\csterm csname\par\csterm endcsname\par
to construct a control sequence. 
For instance \Ver>\csname hskip\endcsname 5pt<Rev
is equivalent to \ver=\hskip5pt=.

During this construction process
all macros and other expandable control sequences
between \cs{csname} and \cs{endcsname}
are expanded as usual, until only unexpandable
character tokens remain. A~variation of the above example,
\Ver>\csname \ifhmode h\else v\fi skip\endcsname 5pt<Rev
performs an \cs{hskip} or \cs{vskip} depending on the mode.
The final result of the expansion should 
consist of only character tokens, but
their category codes do not matter.
An unexpandable control sequence gives an error here:
\TeX\ will insert an \cs{endcsname} right before it
as an attempt at error recovery.

With \cs{csname} it is possible to construct
control sequences that cannot ordinarily be written,
because the constituent character tokens may have another category
\alt
than~11, letter. This principle can be used to hide
\howto Hide counters from the user\par
inner control sequences of a macro package from the user.
\example\Ver>
\def\newcounter#1{\expandafter\newcount
    \csname #1:counter\endcsname}
\def\stepcounter#1{\expandafter\advance
    \csname #1:counter\endcsname 1\relax}<Rev
In the second definition the \cs{expandafter} is superfluous,
but it does no harm, and it is conceptually clearer.
\>

The name of the actual counter created by \cs{newcounter}
contains a colon, so that it takes some effort to write this
control sequence. In effect, the counter
is now hidden from the user, who can only
access it through control sequences such as \cs{stepcounter}.
By the way, the macro \cs{newcount} is defined \cs{outer} in
the plain format, so the above definition of \cs{newcounter}
can only be written after \cs{newcount} has been redefined.

If a control sequence formed with \ver>\csname...\endcsname>
has not been defined
before, its meaning is set to \cs{relax}.
Thus if \ver=\xx= is an undefined control sequence, the
command \Ver>\csname xx\endcsname<Rev will {\em not\/}
give an error message, as it is equivalent to \ver=\relax=.
Moreover, after this execution of the
\ver-\csname...\endcsname- statement, the control sequence
\ver=\xx= is itself equivalent to \cs{relax}, so it
will no longer give an `undefined control sequence' error
(see also page~\pgref[relax:cs]).


\point Token assignments by \cs{\let} and \cs{\futurelet}

There are two \gr{let assignment}s in \TeX.
Their syntax is
\disp\cs{let}\gr{control sequence}\gr{equals}%
     \gr{one optional space}\gr{token}\nl
     \cs{futurelet}\gr{control sequence}\gr{token}\gr{token}
     \>
In the syntax of a \cs{futurelet} assignment
no optional equals sign appears.

\spoint[let] \cs{let}

The primitive command \cs{let} assigns the current meaning
\csterm let\par
of a~token to a control sequence or active character.

For instance, in the plain format \cs{endgraf} is defined
as \Ver>\let\endgraf=\par<Rev
This enables macro writers to redefine \cs{par}, while
still having the functionality of the primitive \cs{par}
command available. For example,
\Ver>\everypar={\bgroup\it\def\par{\endgraf\egroup}}<Rev

The case where the \gr{token} to be assigned is not a control
sequence but a character token instead has been treated 
in Chapter~\ref[char].

\spoint \cs{futurelet}

As was explained above, the sequence with \cs{let}
\disp\cs{let}\gr{control sequence}\gr{token$_1$}\gr{token$_2$}%
       \gr{token$_3$}\gr{token$\cdots$}\dispstop
assigns (the meaning of) \gr{token$_1$} to the control sequence, 
and the remaining input stream looks like
\disp\gr{token$_2$}\gr{token$_3$}\gr{token$\cdots$}\dispstop
That is, the \gr{token$_1$} has disappeared from the stream.

The command \cs{futurelet} works slightly differently:
\csterm futurelet\par
given the input stream
\disp\cs{futurelet}\gr{control sequence}\gr{token$_1$}\gr{token$_2$}%
       \gr{token$_3$}\gr{token$\cdots$}\dispstop
it assigns (the meaning of) \gr{token$_2$} to the control sequence, 
and the remaining stream looks like
\disp\gr{token$_1$}\gr{token$_2$}\gr{token$_3$}\gr{token$\cdots$}\dispstop
That is, neither \gr{token$_1$} nor \gr{token$_2$} has
been lifted from the stream.
However, now \gr{token$_1$}
`knows' what \gr{token$_2$} is, without having had to absorb it
as a macro parameter. See an example below.

If a character token has been \cs{futurelet} to a control
sequence, its category code is fixed.
The subsequent \gr{token$_1$} cannot change
it anymore.

\point Assorted remarks

\spoint Active characters

Character tokens of category~13, `active characters',
\altt
can be defined just like
\term active character\par
control sequences.
If the definition of the character appears inside a macro,
the character has to be active at the time of the definition
of that macro.

Consider for example the following definition
(taken from Chapter~\ref[mouth]):\Ver>
{\catcode`\^^M=13 %
 \gdef\obeylines{\catcode`\^^M=13 \def^^M{\par}}%
}<Rev
The unusual category of the \ver>^^M> character
has to be set during the definition of \cs{obeylines},
otherwise \TeX\ would think that the line ended
after \cs{def}.

\spoint Macros versus primitives

The distinction between primitive commands and user macros
\term primitive commands\par
is not nearly as important in \TeX\ as it is in other
programming languages.\itemlist
\item The user can use primitive commands under different names:
      \Ver>\let\StopThisParagraph=\par<Rev
\item Names of primitive commands can be used for
      user macros: \Ver>\def\par{\hfill$\bullet$\endgraf}<Rev
\item Both user macros and a number of \TeX\ primitives
      are subject to expansion, for instance all conditionals,
      and commands such as \cs{number} and~\cs{jobname}.
\itemliststop

\spoint Tail recursion

Macros in \TeX, like procedures in most modern programming
\term recursion\par
languages, are allowed to be recursive: that is, the 
definition of a macro can contain a call to this same macro,
or to another macro that will call this macro.
Recursive macros tend to clutter up \TeX's memory
if too many `incarnations' of such a macro are active
at the same time. However, \TeX\ is able to prevent this
in one frequently occurring case of recursion: tail recursion.

In order to  appreciate what goes on here, some background
knowledge is needed. When \TeX\ starts executing a macro
it absorbs the parameters, and places an item pointing to
the replacement text on the input stack,
\term input stack\par
so that the scanner will next be directed to
this replacement. Once it has been processed, the item on the 
input stack can be removed.
However, if the definition text
of a macro contains further macros, this process will be
repeated for them: new items may be placed on the input stack
directing the scanner to other macros
even before the first one has been completed.

In general this `stack build-up' is a necessary evil, but
it can be prevented if the nested macro call is the
{\em last\/} token in the replacement text of the original
macro. After the last token no further tokens need to be
considered, so one might as well clear the top item
from the input stack
before a new one is put there.
This is what \TeX\ does.

The \cs{loop} macro of plain \TeX\ provides a good illustration
\label[loop:ex]\csterm loop\par
of this principle. The definition is
\Ver>\def\loop#1\repeat{\def\body{#1}\iterate}
\def\iterate{\body \let\next=\iterate
    \else \let\next=\relax\fi \next}<Rev
and this macro can be called for example as follows:
\Ver>\loop \message{\number\MyCount}
    \advance\MyCount by 1
    \ifnum\MyCount<100 \repeat<Rev
The macro \cs{iterate} can call itself and, when it does so,
the recursive call is performed by the last token in the list.
It would have been possible to define \cs{iterate}
as \Ver>
\def\iterate{\body \iterate\fi}<Rev
but then \TeX\ would not have been able to resolve the recursion
as the call \cs{iterate} is not the last token in the replacement
text of \cs{iterate}. Assigning \ver>\let\next=\iterate>
is here a way to let
the recursive call be the last token in the list.

Another way of resolving tail recursion is to use
\cs{expandafter} (see page~\pgref[after:cond]): in
\Ver>\def\iterate{\body \expandafter\iterate\fi}<Rev
it removes the \cs{fi} token.
Tail recursion would also be resolved if the last
tokens in the list were arguments for the
recursive macro.

An aside: by defining \cs{iterate} as
\Ver>\def\iterate{\let\next\relax 
    \body \let\next\iterate \fi \next}<Rev
it becomes possible to write
\Ver>\loop ... \if... ... \else ... \repeat<Rev

\point Macro techniques

\spoint Unknown number of arguments

In some applications,
\howto  Macros with an undetermined number
of arguments\par
a macro is needed that can have a
number of arguments that is not specified in advance.

Consider the problem of translating a position on a chess board
(for full macros and fonts, see~\cite[chess] and~\cite[Tut]),
given like
\Ver>\White(Ke1,Qd1,Na1,e2,f4)<Rev 
to a sequence of typesetting instructions
\Ver>\WhitePiece{K}{e1} \WhitePiece{Q}{d1} \WhitePiece{N}{a1} 
\WhitePiece{P}{e2} \WhitePiece{P}{f4}<Rev
Note that for pawns the `P' is omitted in the list of positions.

The first problem is that the list of pieces 
is of variable length, so we append a terminator piece:
\Ver>\def\White(#1){\xWhite#1,xxx,}
\def\endpiece{xxx}<Rev
for which we can test.
Next, the macro \cs{xWhite} takes one position from the list,
tests whether it is the terminator, and if not,
subjects it to a test to see whether it is a pawn.
\Ver>
\def\xWhite#1,{\def\temp{#1}%
    \ifx\temp\endpiece 
    \else \WhitePieceOrPawn#1XY%
          \expandafter\xWhite 
    \fi}<Rev
An \cs{expandafter} command is necessary to remove the
\cs{fi} (see page~\pgref[after:cond]), so that 
\cs{xWhite} will get the next position as argument
instead of \cs{fi}.

Positions  are either two or three characters long.
The call to \cs{White\-Piece\-OrPawn}, a four-parameter macro,
appended a terminator string \n{XY}. 
In the case of a pawn, therefore, argument~3 is the character~\n X
and argument~4 is empty; for all other pieces argument~1
is the piece, 2~and~3 are the position, and argument~4 is~\n X.
\Ver>\def\WhitePieceOrPawn#1#2#3#4Y{
    \if#3X \WhitePiece{P}{#1#2}%
    \else  \WhitePiece{#1}{#2#3}\fi}<Rev

\spoint Examining the argument

It may be necessary in some cases to test whether a macro
\howto Examine a macro argument for the presence of some element\par
\howto Apply \cs{uppercase} when the argument has a \cs{footnote}\par
argument contains some element. For a real-life example,
consider the following (see also the \cs{DisplayEquation}
\alt
example on page~\pgref[left:display]).

Suppose the title and author of an article are given as
\Ver>\title{An angle trisector}
\author{A.B. Cee\footnote*{Research supported by the
Very Big Company of America}}<Rev with multiple authors
given as
\Ver>
\author{A.B. Cee\footnote*{Supported by NSF grant 1}
        \and
        X.Y. Zee\footnote{**}{Supported by NATO grant 2}}<Rev
Suppose further that the \cs{title} and \cs{author} macros
are defined as
\Ver>\def\title#1{\def\TheTitle{#1}}  \def\author#1{\def\TheAuthor{#1}}<Rev
which will be used as
\Ver>
\def\ArticleHeading{ ... \TheTitle ... \TheAuthor ... }<Rev

For some journals it is required to
have the authorship and the title of the article in all capitals.
The implementation of this could be
\Ver>
\def\ArticleCapitalHeading
   { ...
    \uppercase\expandafter{\TheTitle}
     ...
    \uppercase\expandafter{\TheAuthor}
     ...
   }<Rev
Now the \cs{expandafter} commands will expand the title and
author into the actual texts, and the \cs{uppercase} commands
will capitalize them. However, for the authors this is wrong,
since the \cs{uppercase} command will also capitalize the
footnote texts.
The problem is then to uppercase only the parts
of the title in between the footnotes.

As a first attempt, let us take the case of one author, and
let the basic call be
\Ver>\expandafter\UCnoFootnote\TheAuthor<Rev
This expands into
\Ver>\UCnoFootnote A.B. Cee\footnote*{Supported ... }<Rev
The macro
\Ver>
\def\UCnoFootnote#1\footnote#2#3{\uppercase{#1}\footnote{#2}{#3}}<Rev
will analyse this correctly:
\Ver>#1<-A.B. Cee
#2<-*
#3<-Supported ...<Rev
However, if there is no footnote, this macro is completely wrong.

As a first refinement we add a footnote ourselves, just to make
sure that one is present:
\Ver>
\expandafter\UCnoFootnote\TheAuthor\footnote 00<Rev
Now we have to test what kind of footnote we find:
\Ver>
\def\stopper{0}
\def\UCnoFootnote#1\footnote#2#3{\uppercase{#1}\def\tester{#2}%
    \ifx\stopper\tester
    \else\footnote{#2}{#3}\fi}<Rev
With \cs{ifx} we test the delimiter footnote sign against the
actual sign encountered. Note that a solution with
\Ver>\ifx0#2<Rev would be wrong if the footnote sign consists
of more than one token, for instance~\ver>{**}>.

The macro so far is correct if there was no footnote,
but if there was one it is wrong:
the terminating tokens remain to be disposed of.
They are taken care of in the following version:
\Ver>
\def\stopper{0}
\def\UCnoFootnote#1\footnote#2#3{\uppercase{#1}\def\tester{#2}%
    \ifx\stopper\tester
    \else\footnote{#2}{#3}\expandafter\UCnoFootnote
    \fi}<Rev
A repeated call to \cs{UCnoFootnote} removes the delimiter tokens
(the \cs{expandafter} first removes the \cs{fi}),
and as an added bonus, this macro is also correct for multiple
authors.


\spoint Optional macro parameters with \cs{futurelet}

One standard application of \cs{futurelet} is implementing
\howto Macros with optional parameters\par
optional parameters of macros. The general course of action
is as follows:
\Ver>\def\Com{\futurelet\testchar\MaybeOptArgCom}
\def\MaybeOptArgCom{\ifx[\testchar \let\next\OptArgCom 
                 \else \let\next\NoOptArgCom \fi \next}
\def\OptArgCom[#1]#2{ ... }\def\NoOptArgCom#1{ ... }<Rev
Note that \cs{ifx} is used even though it tests
for a character. The reason is of course that,
if the optional argument is omitted, there might be an
expandable control sequence behind the~\cs{Com}.

The macro \cs{Com} now has one optional and one regular
argument; it can be called as 
\Ver>\Com{argument}<Rev or as\Ver>
\Com[optional]{argument}<Rev
Often the call without the optional argument will insert some
default value:
\disp\ver>\def\NoOptArgCom#1{\OptArgCom[>%
{\italic default\/}\ver>]{#1}}>\dispstop
This mechanism is widely used in formats such as \LaTeX\ and
\LamsTeX; see also~\cite[svb:future].



\spoint Two-step macros

Often what looks to the user like one macro is in reality
a two-step process, where one macro will set up conditions,
and a second macro will do the work.

As an example, here is
a macro \cs{PickToEol}\label[pick:eol]
\howto Take an input line as macro argument\par
with an argument that is delimited by the line end.
First we write a macro without arguments that 
changes the category code of the line end, and then
calls the second macro.
\Ver>\def\PickToEol{\begingroup\catcode`\^^M=12 \xPickToEol}<Rev
The second macro can then take as an argument everything
up to the end of the line:
\Ver>\def\xPickToEol#1^^M{ ... #1 ... \endgroup}<Rev
There is one problem with this definition: the \ver>^^M> character
should have category~12. We arrive at the following:
\Ver>\def\PickToEol{\begingroup\catcode`\^^M=12 \xPickToEol}
{\catcode`\^^M=12 %
 \gdef\xPickToEol#1^^M{ ... #1 ... \endgroup}%
}<Rev
where the category code of \ver>^^M> is changed for the
sake of the definition of \cs{xPickToEol}. Note that
the \ver>^^M> in \cs{PickToEol} occurs in a control symbol,
so there the category code  is irrelevant. Therefore that
definition can be outside the group where the category code 
of \ver>^^M> is redefined.


\spoint  A comment environment

As an application of the above idea of two-step macros,
\howto Comment environment\par
and in order to illustrate tail recursion, here are 
macros for a `comment' environment.

Often it is necessary to remove a part of \TeX\
input temporarily. For this one would like to
write \Ver>\comment
...
\endcomment<Rev
The simplest implementation of this, 
\Ver>\def\comment#1\endcomment{}<Rev 
has a number of weaknesses. For instance,
it cannot cope with outer macros or input that 
does not have balanced braces. Its worst
shortcoming, however, is that it reads the complete
comment text as a macro argument. This limits the size
of the comment to that of \TeX's input buffer.

It would be a better idea to take on the out-commented
text one line at a time. For this we want to write
a recursive macro with a basic structure
\Ver>\def\comment#1^^M{ ... \comment }<Rev
In order to be able to write this definition at all,
the category code of the line end must be changed; as above
\altt
we will have
\Ver>\def\comment{\begingroup \catcode`\^^M=12 \xcomment}
{\catcode`\^^M=12 \endlinechar=-1 %
 \gdef\xcomment#1^^M{ ... \xcomment}
}<Rev
Changing the \cs{endlinechar} is merely to 
prevent having to put comment characters at the end
of every line of the definition.

Of course, the process must stop at a certain time.
To this purpose we investigate the line that was
scooped up as macro argument:
\Ver>{\catcode`\^^M=12 \endlinechar=-1 %
 \gdef\xcomment#1^^M{\def\test{#1}
    \ifx\test\endcomment \let\next=\endgroup
    \else \let\next=\xcomment \fi
    \next}
}<Rev
and we have to define \cs{endcomment}:
\Ver>\def\endcomment{\endcomment}<Rev
This command will never be executed: it is merely for purposes
of testing whether the end of the environment has been reached.

We may want to comment out text that is not syntactically
correct. Therefore we switch to a verbatim mode
\term verbatim mode\par
when commenting. The following macro is given 
in plain \TeX:\Ver>
\def\dospecials{\do\ \do\\\do\{\do\}\do\$\do\&%
  \do\#\do\^\do\^^K\do\_\do\^^A\do\%\do\~}<Rev
We use it to define \cs{comment} as follows:
\Ver>\def\makeinnocent#1{\catcode`#1=12 }
\def\comment{\begingroup
    \let\do=\makeinnocent \dospecials
    \endlinechar`\^^M \catcode`\^^M=12 \xcomment}<Rev
Apart from the possibility mentioned above of commenting
out text that is not syntactically correct, for instance
because of unmatched braces, this solution can handle
outer macros. The former implementation of \cs{xcomment}
would cause a \TeX\ error if one occurred in the comment text.

However, using verbatim mode poses the problem of concluding the 
environment.
\altt
The final line of the comment is now not the control sequence
\cs{endcomment}, but the characters constituting it. We have
to test for these then:
\Ver>{\escapechar=-1
 \xdef\endcomment{\string\\endcomment}
}<Rev The sequence \ver>\string\\> gives a backslash.
We could not have used
\Ver>\edef\endcomment{\string\endcomment}<Rev because
the letters of the word \n{endcomment} would then have
category code~12, instead of the 11 that the ones on the
last line of the comment will have.

\endinput
