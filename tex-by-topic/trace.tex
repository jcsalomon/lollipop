\subject[trace]  Tracing

\TeX's workings are often quite different from what
\term tracing\par\term statistics\par
the programmer expected, so there are ways to discover how \TeX\
arrived at the result it did. The \cs{tracing...} 
commands write
all information of a certain kind to the log file 
(and to the terminal if \cs{tracingonline} is positive),
and a number of \cs{show...} commands can be used to ask the
current status or value of various items of \TeX.

In the following list, only \cs{show} and \cs{showthe}
display their output on the terminal by default,
other \cs{show...} and \cs{tracing...} commands
write to the log file. They will write in addition to
the terminal if \cs{tracingonline} is positive.

\invent
\item meaning 
      Give the meaning of a control sequence as a string of characters.

\item show 
      Display the meaning of a control sequence.

\item showthe 
      Display the result of prefixing a token with \cs{the}.

\item showbox 
      Display the contents of a box.

\item showlists 
      Display
      the contents of the partial lists
      currently built in all modes.
      This is treated on
      page~\pgref[showlists].

\item tracingcommands
      If this is~1 \TeX\ displays primitive commands executed; 
      if this is 2~or more the outcome of conditionals is also recorded.
      \csterm tracingcommands\par

\item tracingmacros 
      If this is~1, \TeX\ shows expansion of macros 
      that are performed and the actual values of the arguments; 
      if this is 2~or more \gr{token parameter}s such as
      \cs{output} and \cs{everypar} are also traced.
      \csterm tracingmacros\par

\item tracingoutput 
      If this is positive, the log file shows a dump of boxes 
      that are shipped to the \n{dvi} file.

\item showboxdepth  
      The number of levels of box dump that are shown when 
      boxes are displayed.

\item showboxbreadth 
      Number of successive elements on each level that are shown when 
      boxes are displayed.

\item tracingonline 
      If this parameter is positive, \TeX\ will write trace      
      information to the terminal in addition to the log file.
      \csterm tracingonline\par

\item tracingparagraphs 
      If this parameter is positive, \TeX\ generates      
      a trace of the line breaking algorithm.

\item tracingpages 
      If this parameter is positive, \TeX\ generates      
      a trace of the page breaking algorithm.
      \csterm tracingpages\par

\item tracinglostchars 
      If this parameter is positive, \TeX\ gives      
      diagnostic messages whenever a character is accessed that      
      is not present in a font.
      \csterm tracinglostchars\par
      Plain default:~1.

\item tracingrestores 
      If this parameter is positive, \TeX\ will report      
      all values that are restored when a group ends.
      \csterm tracingrestores\par

\item tracingstats 
      If this parameter is~1, \TeX\ reports at the      
      end of the job the usage of various internal arrays;
      if it is~2, the memory demands are given whenever
      a page is shipped out.

\inventstop

\point Meaning and content: \cs{\show}, \cs{\showthe}, \cs{\meaning}

The meaning of control sequences, and the contents of those
that represent internal quantities, can be obtained
by the primitive commands \cs{show}, \cs{showthe},
and~\cs{meaning}.

The control sequences \cs{show} and \cs{meaning} are similar:
\alt
the former will give
\csterm show\par\csterm meaning\par
output to the log file and the terminal, whereas the latter
will produce the same tokens, but they are placed in \TeX's
input stream.

The meaning of a primitive command of \TeX\ is that command itself:
\Ver>\show\baselineskip<Rev gives
\Ver>\baselineskip=\baselineskip<Rev
The meaning of a defined  quantity is its definition:
\Ver>\show\pageno<Rev gives
\Ver>\pageno=\count0<Rev
The meaning of a macro is its parameter text and replacement text:
\Ver>\def\foo#1?#2\par{\set{#1!}\set{#2?}}
\show\foo<Rev gives
\Ver>\foo=macro:
#1?#2\par ->\set {#1!}\set {#2?}<Rev
For macros without parameters the part before the arrow
(the parameter text) is empty.

The \cs{showthe} command will display on the log file and terminal 
\csterm showthe\par
the tokens that \cs{the} produces. 
After \cs{show}, \cs{showthe}, \cs{showbox}, and \cs{showlists}
\TeX\ asks the user for input; this can be prevented
by specifying \cs{scrollmode}.
Characters generated
by \cs{meaning} and \cs{the} have category~12, except for spaces
(see page~\pgref[cat12]);
the value of \cs{escapechar} is used when control sequences
are represented.

\point Show boxes: \cs{\showbox}, \cs{\tracingoutput}

If \cs{tracingoutput} is positive the log file will
\csterm tracingoutput\par\csterm showbox\par
receive a dumped representation of all boxes that are
written to the \n{dvi} file with \cs{shipout}.
The same representation is used
by the command \cs{showbox}\gr{8-bit number}.

In the first case \TeX\ will report `Completed box being shipped out';
in the second case it will enter \cs{errorstopmode}, and
tell the user `OK. (see the transcript file)'.
If \cs{tracingonline} is positive, the box is also displayed
on the terminal; if \cs{scrollmode} has been specified,
\TeX\ does not stop for input.

The upper bound on the
number of nested boxes that is dumped is \cs{showboxdepth};
\csterm showboxdepth\par\csterm showboxbreadth\par
each time a level is visited at most \cs{showboxbreadth}
items are shown, the remainder of the list is summarized
with~\n{etc.}
For each box its height, depth, and width
are indicated in that order, and for characters it is
stated from what font they were taken. 

\example After
\Ver>\font\tenroman=cmr10 \tenroman
\setbox0=\hbox{g}
\showbox0<Rev 
the log file will show
\Ver>\hbox(4.30554+1.94444)x5.00002
.\tenroman g<Rev indicating that the box was \n{4.30554pt} high,
\n{1.94444pt} deep, and \n{5.00002pt} wide, and that it contained
a character `g' from the font \cs{tenroman}. 
Note that the fifth decimal of all sizes may be rounded
because \TeX\ works with multiples of $2^{-16}$\n{pt}.
\message{ifmath: scriptfont fam0 fill!}
\>

The next example has nested boxes, 
\Ver>\vbox{\hbox{g}\hbox{o}}<Rev
and it contains \cs{baselineskip} glue between the boxes.
After a \cs{showbox} command the log file output is:
\Ver>\vbox(16.30554+0.0)x5.00002
.\hbox(4.30554+1.94444)x5.00002
..\tenroman g
.\glue(\baselineskip) 5.75002
.\hbox(4.30554+0.0)x5.00002
..\tenroman o<Rev
Each time a new level is entered an extra dot is added to
the front of the line. Note that \TeX\ tells explicitly
that the glue is \cs{baselineskip} glue;
it inserts names like this for all automatically inserted glue.
The value of
the baselineskip glue here is such that the baselines of
the boxes are at 12 point distance.

Now let us look at explicit (user) glue. \TeX\ indicates the ratio
by which it is stretched or shrunk. 

\examples
\Ver>\hbox to 20pt {\kern10pt \hskip0pt plus 5pt}<Rev
gives (indicating that the available stretch has been 
multiplied by~\n{2.0}):
\Ver>\hbox(0.0+0.0)x20.0, glue set 2.0
.\kern 10.0
.\glue 0.0 plus 5.0<Rev
and
\Ver>\hbox to 0pt {\kern10pt \hskip0pt minus 20pt}<Rev
gives (the shrink has been multiplied by~\n{0.5})
\Ver>\hbox(0.0+0.0)x0.0, glue set - 0.5
.\kern 10.0
.\glue 0.0 minus 20.0<Rev
respectively.
\>

This is an example with infinitely stretchable or shrinkable
glue:
\Ver>\hbox(4.00000+0.14000)x15.0, glue set 9.00000fil<Rev
This means that the horizontal box contained \n{fil} glue, and
it was set such that its resulting width was \n{9pt}.

Underfull boxes are dumped like all other boxes, but
the usual `\n{Underfull hbox detected at line...}'
is given. Overfull horizontal boxes contain a vertical rule
of width \cs{overfullrule}:
\Ver>\hbox to 5pt {\kern10pt}<Rev gives
\Ver>\hbox(0.0+0.0)x5.0
.\kern 10.0
.\rule(*+*)x5.0<Rev


Box leaders are not dumped completely:
\Ver>.\leaders 40.0
..\hbox(4.77313+0.14581)x15.0, glue set 9.76852fil
...\tenrm a
...\glue 0.0 plus 1.0fil<Rev
is the dump for
\Ver>\leaders\hbox to 15pt{\tenrm a\hfil}\hskip 40pt<Rev
Preceding or trailing glue around the leader
boxes is also not indicated.

\point Global statistics

The parameter \cs{tracingstats} can be used to force \TeX\
\csterm tracingstats\par
to report at the end of the job the global use of resources.
Some production versions of \TeX\ may not have this option.

As an example, here are the statistics for this book:
\Ver>Here is how much of TeX's memory you used:<Rev
String memory (bounded by `pool size'):
\Ver> 877 strings out of 4649
 9928 string characters out of 61781<Rev
Main memory, control sequences, font memory:
\Ver> 53071 words of memory out of 262141
 2528 multiletter control sequences out of 9500
 20137 words of font info for 70 fonts,
             out of 72000 for 255<Rev
Hyphenation:
\Ver> 14 hyphenation exceptions out of 607<Rev
Stacks: input, nest, parameter, buffer, and save stack respectively,
\Ver> 17i,6n,19p,245b,422s stack positions out of 
 300i,40n,60p,3000b,4000s<Rev

\endinput

\point Line breaking: \cs{\tracingparagraphs}

If \cs{tracingparagraphs} is positive, \TeX's line breaking
\csterm tracingparagraphs\par
algorithm will generate trace output. However, on some \TeX\
implementations this trace mode may have been disabled to get a 
faster running system.

Consider an example paragraph of \TeX:
\Ver>\hsize=3in \parindent=0cm \frenchspacing
\pretolerance=500
This is a sample paragraph to show the trace output that
\TeX's line breaking algorithm produces. Some \TeX\ systems
cannot generate this trace, as the relevant piece of code
has been commented out for speed optimisation. 
With ever faster computers this won't be necessary any more.<Rev

\TeX\ first attempts to break the paragraph without
hyphenation, and it will accept solutions where each
line has a badness less than \cs{pretolerance}.
\Ver>@firstpass<Rev Report that the first pass has started;
\Ver>[]\tenrm This is a sample paragraph to show the trace <Rev
Apparently this is the only way to fill the first line;
\Ver>@ via @@0 b=263 p=0 d=84529<Rev and doing so
had badness~263, a zero penalty, and a resulting 84529
demerit points.
\Ver>@@1: line 1.0 t=84529 -> @@0<Rev Conclusion:
breakpoint~1 (\ver>@@1>) occurs on line~1, and it makes the
line `very loose' (indicated by the~\n{.0}), 
and the total demerits are
84529 if the previous breakpoint was number~0.

The first pass is now aborted.
\Ver>
@secondpass
[]\tenrm This is a sam-ple para-graph to show the trace out-
@\discretionary via @@0 b=2 p=50 d=2644
@@1: line 1.2- t=2644 -> @@0<Rev
With a very small badness of~2, but with 50 penalty points
for breaking at a hyphen, this line is `decent' 
(indicated by the~\n{.2}), and the total of demerit points
is~2644.

The second line is also straighforward:
\Ver>
put that T[]X's line break-ing al-go-rithm pro-duces. 
@ via @@1 b=0 p=0 d=100
@@2: line 2.2 t=2744 -> @@1<Rev
The demerits now derive solely from the \cs{linepenalty},
which is~10. Similarly the third line:
\Ver>
Some T[]X sys-tems can-not gen-er-ate this trace, as 
@ via @@2 b=1 p=0 d=121
@@3: line 3.2 t=2865 -> @@2<Rev

For the fourth line two possibilities exist:
it can be set `loose' with 9409 demerit points
\Ver>
the rel-e-vant piece of code has been com-mented 
@ via @@3 b=87 p=0 d=9409
@@4: line 4.1 t=12274 -> @@3<Rev
or, fitting in an extra word, it can be set `tight' with
2601 demerit points
\Ver>out 
@ via @@3 b=41 p=0 d=2601
@@5: line 4.3 t=5466 -> @@3<Rev

Line 5 can be set in three ways:
coming from breakpoint~4 it can be broken as
\Ver>
for speed op-ti-mi-sa-tion. With ever faster com-
@\discretionary via @@4 b=0 p=50 d=2600
@@6: line 5.2- t=14874 -> @@4<Rev
and coming from breakpoint~5 there are two ways:
\Ver>put-
@\discretionary via @@5 b=2 p=50 d=2644
@@7: line 5.2- t=8110 -> @@5<Rev
and \Ver>ers 
@ via @@5 b=84 p=0 d=8836
@@8: line 5.3 t=14302 -> @@5<Rev
Of the three, the last possibility is the only one that
does not involve hyphenating line~5.

As line 6 is the last line of the paragraph, coming from
breakpoints 6 or~7 gives an extra 5000 demerit points
from the \cs{finalhyphendemerits}.
\Ver>this won't be nec-es-sary any more. 
@\par via @@6 b=0 p=-10000 d=5100
@\par via @@7 b=0 p=-10000 d=5100
@\par via @@8 b=0 p=-10000 d=100
@@9: line 6.2- t=13210 -> @@7<Rev
However, coming from breakpoint 7 still gives the least
demerits.









