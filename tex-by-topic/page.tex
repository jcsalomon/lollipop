\subject[page:shape]  Page Shape

This chapter treats some of the parameters that 
determine the size of the page and how it appears on paper.

\invent
\item topskip 
      Minimum distance between the top of the page box
      and the baseline of the first box on the page. 
      Plain \TeX\ default:~\n{10pt}

\item hoffset \cs{voffset}
      Distance by which the page is shifted right/""down 
      with respect to the reference point.

\item vsize 
      Height of the page box.
      Plain \TeX\ default:~\n{8.9in}

\item maxdepth 
      Maximum depth of the page box.
      Plain \TeX\ default:~\n{4pt}

\item splitmaxdepth 
      Maximum depth of a box split off by a \cs{vsplit} operation. 
      Plain \TeX\ default:~\cs{maxdimen}

\inventstop

\point The reference point for global positioning

It is a \TeX\ convention, to which output device drivers
\term page positioning\par
must adhere, that the top left point of the page is
one inch from the page edges. Unfortunately this
may lead to lots of trouble, for instance if a printer
(or the page description language it uses)
takes, say, the {\em lower\/} left corner as the
reference point, and is factory set to US paper sizes,
but is used with European standard A4 paper.

The page is shifted on the paper if one assigns non-zero
\csterm hoffset\par\csterm voffset\par
values to \cs{hoffset} or \cs{voffset}: positive values
shift to the right and down respectively.

\point \cs{\topskip}

The \cs{topskip} ensures to a certain point
\csterm topskip\par
that the first baseline of a page
will be at the same location from page to page,
even if font sizes
are switched between pages or if the first line has
no ascenders.

Before the first box on each page some glue is inserted.
This glue has the same stretch and shrink as \cs{topskip}, but
the natural size is the natural size of \cs{topskip}
minus the height of the first box, or zero if this
would be negative.

Plain \TeX\ sets \cs{topskip} to {\tt 10pt}.
Thus the top lines of pages will have their baselines
at the same place if
the top portion of the characters is ten point or less.
For the Computer Modern fonts this condition is satisfied
if the font size is less than (about) 13~points; 
for larger fonts
the baseline of the top line will drop.

The height of the page box for a page containing only
text (and assuming a zero \cs{parskip})
will be the \cs{topskip} plus a number of times
the \cs{baselineskip}. Thus one can define a macro
to compute the \cs{vsize} from the number of lines
on a page:
\howto Specify page height in lines\par
\Ver>\def\HeightInLines#1{\count@=#1\relax
    \advance\count@ by -1\relax
    \vsize=\baselineskip
    \multiply\vsize by \count@
    \advance\vsize by \topskip}<Rev
Calculating the \cs{vsize} this way will prevent
underfull boxes for text-only pages.

In cases where the page does not start with a line of text
(for instance a rule), the topskip may give unwanted effects.
To prevent these, start the page with
\Ver>\hbox{}\kern-\topskip<Rev
followed by what you wanted on top. 

Analogous to the \cs{topskip}, there is a \cs{splittopskip}
for pages generated by a \cs{vsplit} operation; see
the next chapter.

\point Page height and depth

\TeX\ tries to build pages as a \cs{vbox} of height \cs{vsize};
\alt
\csterm vsize\par
see also \cs{pagegoal} in the next chapter.

If the last item on a page has an excessive depth,
\term page depth\par
that page would be noticeably longer than other pages.
To prevent this phenomenon \TeX\ uses \cs{maxdepth} as
\csterm maxdepth\par
the maximum depth of the page box. If adding an item to the
page would make the depth exceed this quantity, then the
reference point of the page is moved down to make the depth
exactly \cs{maxdepth}.

The `raggedbottom' effect is obtained in plain \TeX\
\csterm raggedbottom\par
by giving the \cs{topskip} some finite stretchability:
\hbox{\n{10pt plus 60pt}}.
Thus the natural height of box~255 can vary when it reaches
the output routine.
Pages are then shipped out (more or less) as
\Ver>\dimen0=\dp255 \unvbox255
\ifraggedbottom \kern-\dimen0 \vfil \fi <Rev
The \cs{vfil} causes the topskip to be set at natural
width, so the effect is one of a fixed top line  and a
variable bottom line of the page.

Before \cs{box255} is unboxed in the plain \TeX\ output routine,
\cs{boxmaxdepth} is set to \cs{maxdepth}
so that this box will made under the same assumptions
that the page builder used when putting together \cs{box255}.

The depth of box split off by a \cs{vsplit} operation
is controlled by the \cs{splitmaxdepth} parameter.

\subject[page:break]  Page Breaking

This chapter treats the `page builder': the part of \TeX\
that decides where to break the main vertical list into pages.
The page builder operates before the output routine,
and it hands its result in \cs{box255} to the output routine.

\invent
\item vsplit
      Split of a top part of a box. This is comparable
      with page breaking.

\item splittopskip 
      Minimum distance between the top of what remains after a
      \cs{vsplit} operation, and the first item in that box.
      Plain \TeX\ default:~\n{10pt}

\item pagegoal 
      Goal height of the page box. This starts at \cs{vsize},
      and is diminished by heights of insertion items.

\item pagetotal 
      Accumulated natural height of the current page.

\item pagedepth 
      Depth of the current page.

\item pagestretch 
      Accumulated zeroth-order stretch of the current page.

\item pagefilstretch 
      Accumulated first-order stretch of the current page.

\item pagefillstretch 
      Accumulated second-order stretch of the current page.

\item pagefilllstretch 
      Accumulated third-order stretch of the current page.

\item pageshrink 
      Accumulated shrink of the current page.

\item outputpenalty   
      Value of the penalty at the current page break,
      or $10\,000$ if the break was not at a penalty.

\item interlinepenalty 
      Penalty for breaking a page between lines of a paragraph. 
      Plain \TeX\ default:~\n{0}

\item clubpenalty 
      Additional penalty for breaking a page after 
      the first line of a paragraph. 
      Plain \TeX\ default:~\n{150}

\item widowpenalty 
      Additional penalty for breaking a page before 
      the last line of a paragraph. 
      Plain \TeX\ default:~\n{150}

\item displaywidowpenalty 
      Additional penalty for breaking a page before the last line 
      above a display formula. 
      Plain \TeX\ default:~\n{50}

\item brokenpenalty 
      Additional penalty for breaking a page after a hyphenated line. 
      Plain \TeX\ default:~\n{100}

\item penalty
      Place a penalty on the current list.
\item lastpenalty
      If the last item on the list was a penalty, the value of this.
\item unpenalty
      Remove the last item of the current list if this
      was a penalty.

\inventstop

\point The current page and the recent contributions

The main vertical list of \TeX\ is divided in two parts:
\term current page\par\term recent contributions\par
\term page builder\par
the `current page' and the list of `recent contributions'.
Any material that is added to the main vertical list is
appended to the recent contributions; the act of moving
the recent contributions to the current page is known
as `exercising the page builder'.

Every time something is moved to the current page, \TeX\
computes the cost of breaking the page at that point.
If it decides that it is past the optimal point,
the current page up to 
\altt 
the best break so far
is put in \cs{box255} and the remainder of
the current page is moved back on top of the recent contributions.
If the page is broken at a penalty,
\label[break:penalty]%
that value is recorded in \cs{outputpenalty}, and 
a penalty of size $10\,000$ is placed on top of the
recent contributions; otherwise, \cs{outputpenalty}
\csterm outputpenalty\par
is set to~$10\,000$.

If the current page is empty, discardable items that are moved
from the recent contributions are discarded. This is the mechanism
that lets glue disappear after a page break and at the top of
the first page. When the first non-discardable item is moved
to the current page, the \cs{topskip} glue is inserted;
see the previous chapter.

The workings of the page builder can be made visible by
setting \cs{tracingpages} to some positive value
(see Chapter~\ref[trace]).

\point Activating the page builder

The page builder comes into play  in the
following circumstances.
\itemlist\item Around paragraphs: after the \cs{everypar}
    tokens have been inserted, and after the paragraph has been
    added to the vertical list. See the end of this chapter for 
    an example.
\item Around display formulas: after the \cs{everydisplay}
    tokens have been inserted, and after the display has been
    added to the list.
\item After \cs{par} commands, boxes, insertions, 
    and explicit penalties in vertical mode. 
\item After an output routine has ended. \>
In these places the page builder moves the recent
contributions to the current page. Note that \TeX\ need not be 
in vertical mode when the page builder is exercised.
In horizontal mode, activating the page builder
serves to move preceding vertical glue (for example, \cs{parskip},
\cs{abovedisplayskip}) to the page.

The \cs{end} command \ldash which is only allowed in
external vertical mode \rdash  terminates a \TeX\ job, but only if the
main vertical list is empty and \cs{deadcycles}${}=0$.
If this is not the case the combination
\label[end:play]%
\disp\ver>\hbox{}\vfill\penalty>$-2^{30}$\dispstop
is appended, which forces the output routine to act.

\point Page length bookkeeping

The height and depth of the page box that reaches the output
\term page length\par
routine are determined by \cs{vsize}, \cs{topskip},
and~\cs{maxdepth} as described in the previous chapter.
\TeX\ places the \cs{topskip} glue
when the first box is placed on the current page; the
\cs{vsize} and \cs{maxdepth} are read when the first
box or insertion occurs on the page. Any subsequent changes to these
parameters will not be noticeable until the next page or,
more strictly, until after the output routine has been called.

After the first box, rule, or insertion on the current page
the \cs{vsize} is recorded in \cs{pagegoal},
and its value is not looked at  until \cs{output}
has been active.
Changing \cs{pagegoal} does have an effect on the current
page.
When the page is empty,
the pagegoal is \cs{maxdimen}, and \cs{pagetotal} is zero.

Accumulated dimensions and stretch are available in
the parameters \cs{pagetotal}, \cs{pagedepth}, 
\cs{pagestretch}, \cs{pagefilstretch}, \cs{pagefillstretch}, 
\cs{pageshrink},
and \cs{pagefilllstretch}.
\csterm pagetotal\par\csterm pagedepth\par
\csterm pagestretch\par\csterm pagefilstretch\par
\csterm pagefillstretch\par
\csterm pageshrink\par\csterm pagefilllstretch\par
They are set by the page builder. The stretch and
shrink parameters are updated every time glue is added
to the page. The depth parameter becomes zero
if the last item was kern or glue.

These parameters are \gr{special dimen}s; an assignment
to any of them is an \gr{intimate assignment},
and it is automatically global.

\point Breakpoints

\spoint Possible breakpoints

Page breaks can occur at the same kind of locations where
\term breakpoints in vertical lists\par
line breaks can occur:
\itemlist\item at glue that is preceded by a non-discardable
item;\item at a kern that is immediately followed by glue;
\item at a penalty.\itemliststop
\TeX\ inserts interline glue and various sorts of
interline penalties when the lines of a paragraph are
added to the vertical list, so there will usually be 
sufficient breakpoints on the page.

\spoint Breakpoint penalties

If \TeX\ decides to break a page at a penalty item, this
penalty will, most of the time, be one that
has been inserted automatically
between the lines of a paragraph.

If the last item on a list (not necessarily a vertical list)
\alt
is a penalty, the value of this is recorded
\csterm lastpenalty\par
in the parameter \cs{lastpenalty}. If the item is other than
a penalty, this parameter has the value zero.
The last penalty of a list can be removed with the command
\csterm unpenalty\par
\cs{unpenalty}. See Section~\ref[varioset] for an example.
\message{Spoint ref varioset}

Here is a list of such penalties\term penalties in vertical mode\par:
\invent
\item interlinepenalty 
      \csterm interlinepenalty\par
      Penalty for breaking a page between lines of a paragraph. 
      In plain \TeX\ this is zero, so no penalty is added in
      between lines. \TeX\ can then find a valid breakpoint at the
      \cs{baselineskip} glue.

\item clubpenalty 
      \csterm clubpenalty\par
      Extra penalty for breaking a page after the first line of a paragraph. 
      In plain \TeX\ this is~\n{150}.
      This amount, and the following penalties, are 
      added to the \cs{interlinepenalty}, and
      a penalty of the resulting size is inserted after the
      \cs{hbox} containing the first line of a paragraph
      instead of the \cs{interlinepenalty}.

\item widowpenalty 
      \csterm widowpenalty\par
      Extra penalty for breaking a page before the last line of a paragraph. 
      In plain \TeX\ this is~\n{150}.

\item displaywidowpenalty 
      \csterm displaywidowpenalty\par
      Extra penalty for breaking a page before the last line 
      above a display formula. The default value in plain \TeX\
      is~\n{50}.

\item brokenpenalty 
      \csterm brokenpenalty \par
      Extra penalty for breaking a page after a hyphenated line. 
      The default value in plain \TeX\ is~\n{100}.
\inventstop
If the resulting penalty is zero, it is not placed.

Penalties can also be inserted by the user. For instance,
the plain format has macros to encourage (possibly, force)
or prohibit page breaks\csterm penalty\par:
\Ver>
\def\break{\penalty-10000 }        % force break
\def\nobreak{\penalty10000 }       % prohibit break
\def\goodbreak{\par\penalty-500 }  % encourage page break<Rev
Also, \ver>\vadjust{\penalty ... }> is a way of getting
penalties in the vertical list. This can be used to
discourage or encourage page breaking after a certain
line of a paragraph.

\spoint Breakpoint computation

\advance\rightskip by 5.5cm

Whenever an item is moved to the current page, \TeX\
\term page breaking\par\term breakpoints, computation of\par
\vadjust{\advance\hsize by -5.5cm
  \hbox to \hsize{\hfil\rlap{\hskip.4cm\vtop to 0pt
  {\kern-2\baselineskip
    \SansSerif \PointSize:8 \Style:roman
   \parindent0pt \offinterlineskip
   \def\tbox#1{\hbox{\quad\quad #1%
               \vrule height 10pt depth3pt width0cm }}
   \hbox
   {\vrule width\lw \kern-\lw
    \vbox{\hsize=5cm
          \hrule height\lw \ \vskip0cm
           \kern40pt
           \tbox{underfull page}
           \tbox{$b=10\,000$}
           \kern40pt
          \hrule height\lw
           \kern8pt
           \tbox{feasible breakpoints}
           \tbox{$b<10\,000$}
           \kern8pt
          \hrule height\lw
           \kern8pt
           \tbox{overfull page}
           \tbox{$b=\infty$}
           \kern3pt
           \tbox{.\vrule height3.5pt depth1pt width0cm}
           \tbox{.\vrule height3.5pt depth1pt width0cm}
           \tbox{.\vrule height3.5pt depth1pt width0cm}
           \kern8pt
           }%
    \kern-\lw \vrule width\lw}%
    \vss}}}}
computes the penalty $p$ and the badness $b$ associated with
breaking the page at that place. From the penalty and
the badness the cost $c$ of breaking is computed.

The place of least cost is remembered, and when
the cost is infinite, that is, the page is overfull, or
when the penalty is $p\leq-10\,000$, the current page is broken
at  the (last remembered) place of least cost. 
The broken-off piece is then
put in \cs{box255} and the output routine token list
is inserted. Box 255 is always given a height of \cs{vsize},
regardless of how much material it has.

The badness calculation is based on the amount of stretching
or shrinking that is necessary to fit the page in
a box with height \cs{vsize} 
and maximum depth \cs{maxdepth}. This calculation is
the same as for line breaking (see Chapter~\ref[glue]).
Badness is a value $0\leq b\leq 10\,000$, except when
pages are overfull; then~$b=\infty$.

\advance\rightskip by -5.5cm

Some penalties are implicitly inserted by \TeX, 
for instance the \cs{interlinepenalty}
which is put in between every pair of lines of a paragraph.
Other penalties can
be explicitly inserted by the user or a user macro. 
A~penalty
value $p\geq10\,000$ inhibits breaking; a penalty
$p\leq-10\,000$ (in external vertical mode)
\alt
forces a page break, and immediately
activates the output routine. 

Cost calculation proceeds as follows:
\enumerate \item When a penalty is so low that it forces
a page break and immediate invocation of the output routine, 
but the page is not overfull, that is
\disp$b<\infty\quad\hbox{and}\quad p\leq-10\,000$\>
the cost is equal to the penalty:~$c=p$.

\item When penalties do not force anything, and the page is not
overfull, that is
\disp$b<\infty\quad\hbox{and}\quad |p|<10\,000$\dispstop
the cost is~$c=b+p$.

\item For pages that are very bad, that is
\disp$b=10\,000\quad\hbox{and}\quad |p|<10\,000$\dispstop
the cost is~$c=10\,000$.

\item An overfull page, that is
\disp$b=\infty\quad\hbox{and}\quad p<10\,000$\dispstop
gives infinite cost:~$c=\infty$.
In this case \TeX\ decides that the optimal break point
must have occurred earlier, and it invokes the output routine.
Values of \cs{insertpenalties} (see Chapter~\ref[insert])
that exceed $10\,000$
also give infinite cost.
\enumeratestop

The fact that a penalty $p\leq-10\,000$ activates
the output routine is used extensively
in the \LaTeX\ output routine: 
the excess $\mathopen|p\mathclose|-10\,000$ is
a code indicating the reason for calling the output routine;
see also the second example in the next chapter.

\point[vsplit] \cs{\vsplit}

The page-breaking operation is available to the user
\csterm vsplit\par
through the \cs{vsplit} operation.

\example \Ver>\setbox1 = \vsplit2 to \dimen3<Rev
assigns to box~1 the top part of size \cs{dimen3}
of box~2. This material is actually removed from box~2.
Compare this with splitting off a chunk of size \cs{vsize}
from the current page.\>

The extracted
result of \disp\cs{vsplit}\gr{8-bit number}\n{to}\gr{dimen}
\dispstop is a box with the following properties.
\itemlist \item Height equal to the specified \gr{dimen}; \TeX\ will
      go through the original box register (which must contain
      a vertical box) to find the best breakpoint. This may
      result in an underfull box.
\item Depth at most \cs{splitmaxdepth}; this is analogous to
      \csterm splitmaxdepth\par
      the \cs{maxdepth} for the page box, rather than the \cs{boxmaxdepth}
      that holds for any box.
\item A first and last mark in the \cs{splitfirstmark} and 
      \cs{splitbotmark} registers.
\itemliststop

The remainder of the \cs{vsplit} operation is a box where
\itemlist \item all discardables have been removed
      from the top;
\item glue of size \cs{splittopskip} has been inserted on top;
\csterm splittopskip\par
      if the box being split was box~255, it
      already had \cs{topskip} glue on top;
\item its depth has been forced to be at most \cs{splitmaxdepth}.
\>

The bottom of the original box is always a valid breakpoint
for the \cs{vsplit} operation. If this breakpoint is taken,
the remainder box register is void. The extracted box
can be empty; it is only void if the original box
was void, or not a vertical box.

Typically, the \cs{vsplit} operation is used to split off part
of \cs{box255}. By setting \cs{splitmaxdepth} equal to \cs{boxmaxdepth}
the result is something that could have been made by \TeX's page
builder. After pruning the top of \cs{box255}, the 
mark registers \cs{firstmark} and \cs{botmark} contain the first
and last marks on the remainder of box~255.
See the next chapter for more information on marks.

\point Examples of page breaking

\spoint Filling up a page

Suppose a certain vertical box is too large
to fit on the remainder of the page.
Then \Ver>\vfil\vbox{ ... }<Rev is the wrong way
to fill up the page and push the box to the next.
\TeX\ can only break at the start of the glue, and
the \cs{vfil} is discarded after the break: the result
is an underfull, or at least horribly stretched, page.
On the other hand,
\Ver>\vfil\penalty0 % or any other value
\vbox{ ... }<Rev is the correct way: \TeX\ will break
at the penalty, and the page will be filled.

\spoint Determining the breakpoint

In the following examples the \cs{vsplit} operation is
used, which has the same
mechanism as page breaking.

Let the macros and
parameter settings
\Ver>\offinterlineskip \showboxdepth=1
\def\High{\hbox{\vrule height5pt}}
\def\HighAndDeep{\hbox{\vrule height2.5pt depth2.5pt}}<Rev
be given.

First let us consider
an example where a vertical list is simply stretched
in order to reach a break point.
\Ver>\splitmaxdepth=4pt 
\setbox1=\vbox{\High \vfil \HighAndDeep}
\setbox2=\vsplit1 to 9pt<Rev
gives \Ver>> \box2=
\vbox(9.0+2.5)x0.4, glue set 1.5fil
.\hbox(5.0+0.0)x0.4 []
.\glue 0.0 plus 1.0fil
.\glue(\lineskip) 0.0
.\hbox(2.5+2.5)x0.4 []<Rev
The two boxes together have a height of \n{7.5pt},
so the glue has to stretch~\n{1.5pt}.

Next, we decrease the allowed depth of the resulting list.
\Ver>\splitmaxdepth=2pt 
\setbox1=\vbox{\High \vfil \HighAndDeep}
\setbox2=\vsplit1 to 9pt<Rev gives
\Ver>> \box2=
\vbox(9.0+2.0)x0.4, glue set 1.0fil
.\hbox(5.0+0.0)x0.4 []
.\glue 0.0 plus 1.0fil
.\glue(\lineskip) 0.0
.\hbox(2.5+2.5)x0.4 []<Rev
The reference point is moved down half a point,
and the stretch is correspondingly diminished,
\alt
but this motion cannot lead to a larger dimension
than was specified.

As an example of this,
\alt
consider the sequence \Ver>\splitmaxdepth=3pt 
\setbox1=\vbox{\High \kern1.5pt \HighAndDeep}
\setbox2=\vsplit1 to 9pt<Rev
This gives a box exactly 9 points high and 2.5 points deep.
Setting \ver>\splitmaxdepth=2pt> does not increase
the height by half a point; instead, an underfull box
results because an earlier break is taken.

Sometimes the timing of actions is important.
\TeX\ first locates a breakpoint that will lead
to the requested height, then checks whether accommodating
the \cs{maxdepth} or \cs{splitmaxdepth} will not
violate that height.

Consider an example of this timing:
\alt
in
\Ver>\splitmaxdepth=4pt 
\setbox1=\vbox{\High \vfil \HighAndDeep}
\setbox2=\vsplit1 to 7pt<Rev
the result is {\italic not\/} a box
of 7 points high and 3 points deep. Instead,
\Ver>> \box2=
\vbox(7.0+0.0)x0.4
.\hbox(5.0+0.0)x0.4 []<Rev
which is an underfull box.

\spoint[par:page:build] The page builder after a paragraph

After a paragraph, the page builder moves material
to the current page, but it does not decide whether a breakpoint
has been found yet.

\example\Ver>
\output{\interrupt \plainoutput}% show when you're active
\def\nl{\hfil\break}\vsize=22pt % make pages of two lines
a\nl b\nl c\par \showlists      % make a 3-line paragraph<Rev
will report
\Ver>### current page:
[...]
total height 34.0
 goal height 22.0
prevdepth 0.0, prevgraf 3 lines<Rev Even though more than enough
material has been gathered, \cs{output} is only invoked
when the next paragraph starts: typing a \n d gives
\Ver>! Undefined control sequence.
<output> {\interrupt 
                     \plainoutput }
<to be read again> 
                   d<Rev
when \cs{output} is inserted after \cs{everypar}.
\>



\subject[output]  Output Routines

The final stages of page processing are performed by the
output routine. The page builder cuts off a certain portion
of the main vertical list and hands it to the output routine
in \cs{box255}. This chapter treats the commands and parameters
that pertain to the output routine, and it explains how
output routines can receive information through marks.

\invent
\item output 
      Token list with instructions for shipping out pages.

\item shipout 
      Ship a box to the \n{dvi} file.


\item mark 
      Specify a mark text.

\item topmark 
      The last mark on the previous page.

\item botmark 
      The last mark on the current page.

\item firstmark 
      The first mark on the current page.

\item splitbotmark 
      The last mark on a split-off page.

\item splitfirstmark 
      The first mark on a split-off page.

\item deadcycles 
      Counter that keeps track of how many times 
      the output routine has been called without a \cs{shipout} 
      taking place. 

\item maxdeadcycles 
      The maximum number of times that the output routine is allowed to
      be called without a \cs{shipout} occurring.

\item outputpenalty 
      Value of the penalty at the current page break,
 \alt
      or $10\,000$ if the break was not at a penalty.

\inventstop


\point The \cs{\output} token list

Common parlance has it that
`the output routine is called' when \TeX\ has found a place
to break the main vertical list.
Actually, \cs{output} is not a macro but a token list that
is inserted into \TeX's command stream.

Insertion of the \cs{output} token list happens
\csterm output\par\term output routine\par
inside a group that is implicitly opened.
Also, \TeX\ enters internal vertical mode.
Because of the group, non-local assignments 
(to the page number, for instance)
have to be prefixed with \cs{global}.
The vertical mode implies that during the workings of the 
output routine 
spaces are mostly harmless.

The \cs{output} token list belongs
to the class of the
\gr{token parameter}s. These behave the same as
\cs{toks}$nnn$ token lists; see Chapter~\ref[token].
Assigning an output routine can therefore take the following
forms:
\disp\cs{output}\gr{equals}\gr{general text}\quad
or\quad 
\cs{output}\gr{equals}\gr{filler}\gr{token variable}
\>


\point[output255] Output and \cs{\box255}

\TeX's page builder breaks the current page at the optimal point,
and stores everything above that in \cs{box255};
then, the \cs{output} tokens are inserted into the input stream.
Any remaining material on the main vertical list
is pushed back to the recent 
contributions. 
If the page is broken at a penalty,
\alt
that value is recorded in \cs{outputpenalty}, and 
a penalty of size $10\,000$ is placed on top of the
recent contributions; otherwise, \cs{outputpenalty}
is set to~$10\,000$.
When the output routine is finished, \cs{box255} is
supposed to be empty.
If it is not, \TeX\ gives an error message.

Usually, the output routine will take the pagebox,
\csterm shipout\par
append a headline and/""or footline,
maybe merge in some insertions such as footnotes,
and ship the page to the \n{dvi} file:
\Ver>
\output={\setbox255=\vbox
           {\someheadline 
            \vbox to \vsize{\unvbox255 \unvbox\footins}
            \somefootline}
         \shipout\box255}<Rev
When box 255 reaches the output routine, its height has
been set to \cs{vsize}.
However, the material in it can have considerably
smaller height.
Thus, the above output routine may lead to underfull boxes.
This can be remedied with a \cs{vfil}.

The output routine is under no obligation to
\csterm deadcycles\par
do anything useful with \cs{box255}; it can empty it, or
unbox it to let \TeX\ have another go at finding a page
break. The number of times
that the output routing  postpones the \cs{shipout}
is recorded in \cs{deadcycles}: this parameter is set to~0
by \cs{shipout}, and increased by~1 just before 
every \cs{output}.

When  the number of dead cycles reaches
\csterm maxdeadcycles\par
\cs{maxdeadcycles}, \TeX\ gives an error message,
and performs the default output routine
\Ver>\shipout\box255<Rev instead of the routine it was about
to start.
The \LaTeX\ format has a much higher value for \cs{maxdeadcycles}
than plain \TeX, because the output routine in \LaTeX\
is often called for
intermediate handling of floats and marginal notes.

The \cs{shipout} command can send any \gr{box} to the \n{dvi} file;
this need not be box 255, or even a box
containing the current page.
It does not have to be called inside the output routine, either.

If the output routine produces any material, for instance
by calling \Ver>\unvbox255<Rev this is put on top
of the recent contributions.

After the output routine finishes, the page builder is
activated. In particular, because the current page
has been emptied, the \cs{vsize} is read again.
Changes made to this parameter inside the output
routine (using \cs{global}) will therefore take effect.

\point Marks

Information can be passed to the output routine through the
\term marks\par\csterm mark\par
mechanism of `marks'. The user can specify a token list
with \disp\cs{mark}\lb\gr{mark text}\rb\dispstop
which is put in a mark item on the current vertical list.
The mark text is subject to expansion as in \cs{edef}.

If the mark is given in horizontal mode it migrates to
the surrounding vertical lists like an insertion item
(see page~\pgref[migrate]);
however, if this is not the external vertical list, the
output routine will not find the mark.

Marks are the main mechanism through which the output routine
can obtain information about the contents of the currently
broken-off page, in particular its top and bottom.
\TeX\ sets three variables:
\description 
\item \cs{botmark}
   the last mark occurring on the current page\csterm botmark\par;
\item \cs{firstmark}
   the first mark occurring on the current page\csterm firstmark\par;
\item \cs{topmark}
   the last mark of the previous page\csterm topmark\par, 
   that is, the value of \cs{botmark}
   on the previous page.\descriptionstop
If no marks have occurred yet, all three are empty;
if no marks occurred on the current page, 
all three mark variables are equal
to the \cs{botmark} of the previous page.

For boxes generated by a \cs{vsplit} command (see previous chapter),
the \cs{splitbotmark} and \cs{splitfirstmark}
\csterm splitbotmark\par\csterm splitfirstmark\par
contain the marks of the split-off part; \cs{firstmark}
and \cs{botmark} reflect the state of what remains in the register.

\example Marks can be used to get a section heading into
\howto Do tricks with headlines\par
the headline or footline of the page.
\Ver>\def\section#1{ ... \mark{#1} ... }
\def\rightheadline{\hbox to \hsize
   {\headlinefont \botmark\hfil\pagenumber}}
\def\leftheadline{\hbox to \hsize
   {\headlinefont \pagenumber\hfil\firstmark}}<Rev
This places the title of the first section that starts on a 
left page in the left headline, and the title of the last section
that starts on the right page in the right headline.
Placing the headlines on the page is the job of the output routine;
see below.

It is important that no page breaks can occur in between the
mark and the box that places the title:
\Ver>\def\section#1{ ...
    \penalty\beforesectionpenalty
    \mark{#1}
    \hbox{ ... #1 ...}
    \nobreak
    \vskip\aftersectionskip
    \noindent}<Rev
\>

Let us consider
another example with headlines: often a page looks better if
the headline is omitted on pages where a chapter starts.
This can be implemented as follows:
\Ver>\def\chapter#1{ ... \def\chtitle{#1}\mark{1}\mark{0} ... }
\def\theheadline{\expandafter\ifx\firstmark1 
    \else \chapheadline \fi}<Rev
Only on the page where a chapter starts will the mark be~1,
and on all other pages a headline is placed.

\point Assorted remarks

\spoint Hazards in non-trivial output routines

If the final call to the output routine does not
perform a \cs{shipout}, \TeX\ will call the output
routine endlessly, since a run will only stop if both
the vertical list is empty, and \cs{deadcycles}
is zero. The output routine can set \cs{deadcycles}
to zero to prevent this.

\spoint Page numbering

The page number is not an intrinsic property of the output
\term page numbering\par
routine; in plain \TeX\ it is  the value of \cs{count0}.
The output routine is responsible for increasing the
page number when a shipout of a  page occurs.

Apart from   \cs{count0}, counter registers~1--9 are also used
for page identification: at shipout \TeX\ writes the values
of these ten counters to the \n{dvi} file (see Chapter~\ref[TeXcomm]).
Terminal and log file output display only the non-zero counters,
and the zero counters for which a non-zero counter with
a higher number exists, that is, if \cs{count0}${}=1$ and
\cs{count3}${}=5$ are the only non-zero counters, the
displayed list of counters is~\n{[1.0.0.5]}.

\spoint Headlines and footlines in plain \TeX\

Plain \TeX\ has token lists \cs{headline} and
\cs{footline}; these are used in the macros
\cs{makeheadline} and \cs{makefootline}.
The page is shipped out as (more or less)
\Ver>\vbox{\makeheadline\pagebody\makefootline}<Rev

Both headline and footline are inserted inside a \cs{line}.
For non-standard headers and footers it is easier to
redefine the macros \cs{makeheadline} and \cs{makefootline}
than to tinker with the token lists.

\spoint Example: no widow lines

Suppose that one does not want to allow widow lines,
but pages have in general no stretch or shrink,
for instance because they only contain plain text.
A~solution would be to increase the page length
by one line if a page turns out to be broken
at a widow line.

\TeX's output routine can perform this sort of
trick: if the \cs{widowpenalty} is set to
some recognizable value, the output routine
can see by the \cs{outputpenalty} if a widow
line occurred. In that case, the output routine
can temporarily increase the \cs{vsize}, and
let the page builder have another go at
finding a break point.

Here is the skeleton of such an output routine.
No headers or footers are provided for.\Ver>
\newif\ifLargePage \widowpenalty=147
\newdimen\oldvsize \oldvsize=\vsize
\output={
    \ifLargePage \shipout\box255 
          \global\LargePagefalse
          \global\vsize=\oldvsize
    \else \ifnum \outputpenalty=\widowpenalty
             \global\LargePagetrue
             \global\advance\vsize\baselineskip
             \unvbox255 \penalty\outputpenalty
          \else  \shipout\box255
    \fi   \fi}<Rev
The test \cs{ifLargePage} is set to true by the
output routine if the \cs{outputpenalty}
equals the \cs{widowpenalty}. The page box
is then \cs{unvbox}$\,$ed, so that the page builder
will tackle the same material once more.

\spoint Example: no indentation top of page

Some  output routines can be classified
\howto Prevent indentation on top of page\par
as abuse of the output routine mechanism.
The output routine in this section is a good example of this.

It is imaginable that one wishes paragraphs not to indent
if they start at the top of a page. (There are plenty of objections
to this layout, but occasionally it is used.)
This problem can be solved using the output routine to
investigate whether the page is still empty and, if so,
to give a signal that a paragraph should not indent.

Note that we cannot use the fact here
that the page builder comes into play after
the insertion of \cs{everypar}: even if we could
force the output routine to be activated here,
there is no way for it to remove the indentation box.

The solution given here lets the \cs{everypar}
terminate the paragraph immediately
with \Ver>\par\penalty-\specialpenalty<Rev
which activates the output routine.
Seeing whether the pagebox is empty (after removing
the empty line and any \cs{parskip} glue),
the output routine then can set a switch
signalling whether the retry of the paragraph
should indent.

There are some minor matters in the following
routines, the sense of which is left
for the reader to ponder.
\Ver>
\mathchardef\specialpenalty=10001
\newif\ifPreventSwitch 
\newbox\testbox 
\topskip=10pt

\everypar{\begingroup \par
    \penalty-\specialpenalty
    \everypar{\endgroup}\parskip0pt 
    \ifPreventSwitch \noindent \else \indent \fi
    \global\PreventSwitchfalse
    }
\output{
    \ifnum\outputpenalty=-\specialpenalty 
        \setbox\testbox\vbox{\unvbox255 
                  {\setbox0=\lastbox}\unskip}
        \ifdim\ht\testbox=0pt \global\PreventSwitchtrue
        \else \topskip=0pt \unvbox\testbox \fi
    \else \shipout\box255 \global\advance\pageno1 \fi}<Rev


\spoint More examples of output routines

A large number of examples of output routines
can be found in~\cite[Sal1] and~\cite[Sal2].

\subject[insert] Insertions

Insertions are \TeX's way of handling floating information.
\TeX's page builder calculates what insertions and how many
of them will fit on the page; these insertion items are then
placed in insertion boxes which are to be handled by the
output routine.



\invent
\item insert 
      Start an insertion item.

\item newinsert 
      Allocate a new insertion class. 

\item insertpenalties 
      Total of penalties for split insertions.
      Inside the output routine, the number of held-over insertions.

\item floatingpenalty 
      Penalty added when an insertion is split.

\item holdinginserts
      (\TeX3 only)
      If this is positive, insertions are not placed in their boxes 
      at output time.

\item footins
      Number of the footnote insertion class in plain \TeX.

\item topins
      Number of the top insertion class.

\item topinsert
      Plain \TeX\ macro to start a top insert.

\item pageinsert
      Plain \TeX\ macro to start an insert that will take
      up a whole page.

\item midinsert
      Plain \TeX\ macro that places its argument if there is space,
      and converts it into a top insert otherwise.

\item endinsert
      Plain \TeX\ macro to wind up an insertion item
      that started with \cs{topinsert}, \cs{midinsert},
      or \cs{pageinsert}.

\inventstop


\point Insertion items

Insertions contain floating information.
\term insertions\par
Handling insertions is a strange interplay between the
user, \TeX's internal workings, and the output routine.
First the user specifies an insertion, which is
a certain amount of vertical material; 
then \TeX's page builder decides what insertions should go
on the current page and puts these insertions in insertion boxes;
finally, the output routine has to do something with these boxes.

An insertion item looks like
\csterm insert\par
\disp\cs{insert}\gr{8-bit number}\lb\gr{vertical mode material}\rb
\dispstop where the 8-bit number should not be~255,
because \cs{box255} is used by \TeX\ for passing the page to the output
routine.

The braces around the vertical mode material in an insertion
item can be implicit; they imply a new level of grouping.
The vertical mode material is processed in internal
vertical mode.

Values of \cs{splittopskip}, \cs{splitmaxdepth}, 
and \cs{floatingpenalty} are relevant for split insertions
(see below); the values that are current just before
the end of the group are used.

Insertion items can appear in vertical mode, horizontal
mode, and math mode. For the latter two modes they have to
migrate to the surrounding vertical list
(see page~\pgref[migrate]).
After an insertion item is put on  the vertical list the
page builder is exercised.


\point Insertion class declaration

In the plain format 
the  number for a new insertion class
is allocated by \cs{newinsert}:
\csterm newinsert\par
\Ver>\newinsert\myinsert % new insertion class<Rev
which uses \cs{chardef} to assign a number to the control
sequence.

Insertion classes are allocated numbering from 254 downward.
As box~255 is used for output, this allocation scheme leaves
\cs{skip255}, \cs{dimen255}, and \cs{count255}
free for scratch use.

\point Insertion parameters

For each insertion class~$n$ four registers are allocated:
\itemlist
\item \cs{box}$\,n$ When the output routine is active this
 box contains the insertion items of class~$n$ that should
 be placed on the current page.
\item \cs{dimen}$\,n$ This is the maximum space allotted for
 insertions of class~$n$ per page. If this amount would
 be exceeded \TeX\ will split insertions.
\item \cs{skip}$\,n$ Glue of this size is added the first
 time an insertion item of class~$n$ is added to the
 current page. This is useful for such phenomena as a rule
 separating the footnotes from the text of the page.
\item \cs{count}$\,n$ Each insertion item is a vertical list,
 so it has a certain height. However, the effective height,
 the amount of influence it has on the text height of the
 page, may differ from this real height.
 The value of \cs{count}$\,n$
 is then 1000 times the factor by which the height should
 be multiplied to obtain the effective height.
 
 Consider the following examples:
 \itemlist\item Marginal notes do not affect
 the text height, so the factor should be~0. \item Footnotes
 set in double column mode affect the page by half of their height:
 the count value should by~500. \item Conversely, footnotes
 set at page width underneath a page in double column mode
 affect both columns, so \ldash provided that the double column mode
 is implemented by applying \cs{vsplit} to a double-height column \rdash 
 the count value should be~2000.\itemliststop
\itemliststop

\point Moving insertion items from the contributions list

The most complicated issue with insertions is the algorithm
that adds insertion items to the main vertical list,
and calculates breakpoints if necessary.

\TeX\ never changes the \cs{vsize}, but it diminishes the
\csterm pagegoal\par
\cs{pagegoal} by the (effective) heights of the insertion
items that will appear before a page break. Thus the output
routine will receive a \cs{box255} that has height \cs{pagegoal},
not necessarily \cs{vsize}.

\enumerate
\item When the first insertion of a certain class $n$ occurs
  on the current page \TeX\ has to account for the quantity
  \cs{skip}$\,n$. This step is executed only if no earlier
  insertion item of this class occurs on the vertical list
  \ldash this includes insertions that were split \rdash  but \cs{box}$\,n$
  need not be empty at this time.
  
  If \cs{box}$\,n$ is not empty, its height plus depth is multiplied
  by \cs{count}$\,n/1000$ and the result is subtracted
  from \cs{pagegoal}. Then the \cs{pagegoal} is diminished
  by the natural component of \cs{skip}$\,n$. Any stretch and
  shrink of \cs{skip}$\,n$ are incorporated in \cs{pagestretch}
  and \cs{pageshrink} respectively.
\item If there is a split insertion of class $n$ on the page
  \ldash this case and the previous step in the algorithm are
  mutually exclusive \rdash  the \cs{floatingpenalty} is added to
  \csterm floatingpenalty\par\csterm insertpenalties\par
  \cs{insertpenalties}. A~split insertion is an insertion item
  for which a breakpoint has been calculated as it will not
  fit on the current page in its entirety. Thus the insertion
  currently under consideration will certainly not wind up 
  on the current page.
\item After the preliminary action of the two previous points
  \TeX\ will place the actual insertion item on the main vertical
  list, at the end of the current contributions.
  First it will check whether the item will fit without being split.
  
  There are two conditions to be checked:\itemlist\item
  adding the insertion item (plus all previous insertions of that class)
  to \cs{box}$\,n$ should not let
  the height plus depth of that box exceed \cs{dimen}$\,n$, and
  \item either the effective height of the insertion is negative, or
  \cs{pagetotal} plus \cs{pagedepth} minus \cs{pageshrink}
  plus the effective size of the insertion should be less than
  \cs{pagegoal}.\itemliststop
  If these conditions are satisfied, \cs{pagegoal} is diminished
  by the effective size of the insertion item, that is,
  by the height plus depth, multiplied by \cs{count}$n/1000$.

\item Insertions that fail on one of the two conditions in the
  previous step of the algorithm will be considered for splitting.
  \TeX\ will calculate the size of the maximal portion to 
  be split off the insertion item, such that
  \enumerate\item adding this portion
  together with earlier insertions of this class to \cs{box}$\,n$
  will not let the size of the box exceed \cs{dimen}$\,n$,
  and \item the effective size of this portion,
  added to \cs{pagetotal} plus \cs{pagedepth}, will not
  exceed \cs{pagegoal}. Note that \cs{pageshrink} is not taken
  into account this time, as it was in the previous step.
  \>
  
  Once this maximal size to be split off has been determined,
  \TeX\ locates the least-cost breakpoint in the current 
  insertion item that will result in a box with a  height
  that is equal to this maximal size. The penalty associated
  with this breakpoint is added to \cs{insertpenalties},
  and \cs{pagegoal} is diminished by the effective height plus
  depth of the box to be split off the insertion item.

\enumeratestop



\point Insertions in the output routine

When the output routine comes into action \ldash more precisely:
when \TeX\ starts processing the tokens in the \cs{output}
token list \rdash  all insertions that should be placed on the
current page have been put in their boxes, and
it is the responsibility of the output routine
to put them somewhere in the box that is going to be shipped out.

\example The plain \TeX\ output routine
handles top inserts and footnotes by packaging the following
sequence:
\Ver>\ifvoid\topins \else \unvbox\topins \fi
\pagebody
\ifvoid\footins \else \unvbox\footins \fi<Rev
Unboxing the insertion boxes makes the glue on various parts
of the page stretch or shrink in a uniform manner.
\>

With \TeX3 the insertion mechanism has been extended slightly:
\csterm holdinginserts\par\term \TeX\ version 3\par
the parameter \cs{holdinginserts} can be used to specify that
insertions should not yet be placed in their boxes.
This is very useful if the output routine wants to
recalculate the \cs{vsize}, or if the output routine
is called to do other intermediate calculations instead of
ejecting a page.

During the output routine the parameter
\csterm insertpenalties\par
\cs{insertpenalties} holds the number of insertion items that
are being held over for the next page.
In the plain \TeX\ output routine this is used after the
last page:\Ver>
\def\dosupereject{\ifnum\insertpenalties>0 
    % something is being held over
  \line{}\kern-\topskip\nobreak\vfill\supereject\fi}<Rev

\point Plain \TeX\ insertions

The plain \TeX\ format has only two insertion classes:
the footnotes and the top inserts.
The macro \cs{pageinsert} generates
\csterm pageinsert\par
top inserts that are stretched to be exactly \cs{vsize} high.
The \cs{midinsert} macro tests whether the vertical material
\csterm midinsert\par
specified by the user fits on the page; if so, it is placed
there; if not, it is converted to a top insert.

Footnotes are allowed to be split, but once one has been
split no further footnotes should appear on the current
page. This effect is attained by setting 
\Ver>\floatingpenalty=20000<Rev 
The \cs{floatingpenalty} is added to \cs{insertpenalties}
if an insertion follows a split insertion of the same 
class. However, \cs{floatingpenalty}${}>10\,000$ has infinite
cost, so \TeX\ will take an earlier breakpoint for
splitting off the page from the vertical list.

Top inserts essentially contain only a vertical box
which holds whatever the user specified. Thus such an insert
cannot be split. However, the \cs{endinsert} macro
\csterm endinsert\par
puts a \cs{penalty100} on top of the box, so the
insertion can be split with an empty part before the split.
The effect is that the whole insertion is carried over to
the next page. As the \cs{floatingpenalty} for top inserts
is zero, arbitrarily many of these inserts can be moved forward
until there is a page with sufficient space.

Further examples of insertion macros can be found
in~\cite[Sal3].

%\message{Maybe spaceleft example?}

\endinput

