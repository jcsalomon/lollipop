\subject[par:start] Paragraph Start

At the start of a paragraph \TeX\ inserts a vertical skip
as a separation from the preceding paragraph, and a horizontal
skip as an indentation for the current paragraph.
This chapter explains the exact sequence
of actions,
and it discusses how \TeX's decisions can be altered.

\invent
\item indent 
      Switch to horizontal mode and insert a box of width \cs{parindent}.

\item noindent  
      Switch to horizontal mode with an empty horizontal list.

\item parskip 
      Amount of glue added to 
      the surrounding vertical list when a paragraph starts.
      Plain \TeX\ default:~\n{0pt plus 1pt}.

\item parindent  
      Size of the indentation box added in front of a paragraph.
      Plain \TeX\ default:~\n{20pt}.

\item everypar 
      Token list inserted in front of paragraph text; 

\item leavevmode 
      Macro to switch to horizontal mode if necessary.

\inventstop


\point When does a paragraph start

\TeX\ starts a paragraph whenever it switches from 
vertical mode to (unrestricted) horizontal mode. This switch can
be effected by one of the commands
\cs{indent} and 
\cs{noindent}, for example\Ver>{\bf And now~\dots}
\vskip3pt
\noindent It's~\dots<Rev
or by any \gram{horizontal command}.
Horizontal commands include characters, in-line formulas,
and horizontal skips, but not boxes.
Consider the following examples.
\alt
The character `I' is a horizontal command:
\Ver>\vskip3pt
It's~\dots<Rev A single \n\$ is a horizontal command:
\Ver>$x$ is supposed~\dots<Rev 
The control sequence \cs{hskip} is a horizontal command:
\Ver>\hskip .5\hsize Long indentation~\dots<Rev
The full list of horizontal commands is given on
page~\pgref[h:com:list].

Upon recognizing a horizontal command in vertical mode,
\TeX\ will perform an \cs{indent} command (and all the actions
associated with it; see below), 
and after that it will reexamine the horizontal command,
this time executing it.



\point What happens when a paragraph starts

The \cs{indent} and \cs{noindent}  commands 
\term paragraph start\par
\csterm indent\par\csterm noindent\par
cause a paragraph to be started.
An~\cs{indent} command can either be placed explicitly by
the user or a macro, or it can be inserted by \TeX\ when
a \gr{horizontal command} occurs in vertical mode;
a~\cs{noindent} command can only be placed explicitly.

After  either command is encountered,
\cs{parskip} glue is appended to the surrounding vertical
list\csterm parskip\par,
unless \TeX\ is in internal vertical mode 
and that list is empty
(for example, at the start of a \cs{vbox} or \cs{vtop}).
\TeX\ then switches to unrestricted horizontal mode
with an empty horizontal list.
In the case of \cs{indent} (which may be inserted
implicitly) an empty \cs{hbox} of width
\csterm parindent\par
\cs{parindent} is placed at the start of the horizontal list; 
after \cs{noindent} no indentation
box is inserted. 

The contents of the \cs{everypar} \gr{token parameter}
\csterm everypar\par
are then inserted into the input (see some applications below).
After that,
the page builder is exercised (see Chapter~\ref[page:break]).
Note that this happens in horizontal mode: this is to 
move the \cs{parskip} glue to the current page.

If an \cs{indent} command is given while \TeX\ is already in
horizontal mode, the indentation box is inserted just the same.
This is not very useful.

\point Assorted remarks

\spoint Starting a paragraph with a box

An \cs{hbox} does not imply horizontal mode, so 
an attempt to start a paragraph with a box, for instance
\Ver>\hbox to 0cm{\hss$\bullet$\hskip1em}Text ....<Rev
will make the text following the box
wind up one line below the box.
It is necessary to switch to horizontal mode
explicitly, using for instance \cs{noindent} or
\cs{leavevmode}. 
The latter is defined using \cs{unhbox},
which is a horizontal command.

\spoint Starting a paragraph with a group

If the first \gram{horizontal command} of a paragraph
is enclosed in braces, the \cs{everypar} is evaluated
inside the group. This may give unexpected results.
Consider this example:
\Ver>\everypar={\setbox0=\vbox\bgroup\def\par{\egroup}}
{\bf Start} a paragraph ... \par<Rev
The \gr{horizontal command} starting the paragraph is the
character~`S', so when \cs{everypar} has been inserted
the input is essentially
\Ver>{\bf \indent\setbox0=\vbox\bgroup
    \def\par{\egroup}Start} a paragraph ... \par<Rev
which is equivalent to
\Ver>{\bf \setbox0=\vbox{Start} a paragraph ... \par<Rev
The effect of this is rather different from what was intended.
\alt
Also, \TeX\ will probably end the job inside a group.

\point Examples

\spoint Stretchable indentation 

Considering that \cs{parindent} is a \gram{dimen}, not a \gram{glue},
it is not possible to declare
\Ver>\parindent=1cm plus 1fil<Rev in order to get
a variable indentation at the start of a paragraph.
This problem may be solved by putting
\Ver>\everypar={\nobreak\hskip 1cm plus 1fil\relax}<Rev
The \cs{nobreak} serves to prevent (in rare cases) a line break
at the stretchable glue.

\spoint Suppressing indentation

Inserting 
\ver.{\setbox0=\lastbox}. in the horizontal list
at the beginning of the paragraph
removes the indentation:
indentation consists of a box, which is available through
\cs{lastbox}. Assigning it effectively removes it from the list.

However, this command sequence
has to be inserted at a moment when \TeX\ has
already switched to horizontal mode, so explicit insertion
of these commands in front of the first \gram{horizontal
command} of the paragraph does not work. 
The moment of insertion of the \cs{everypar} tokens 
is a better candidate: specifying
\Ver>\everypar={{\setbox0=\lastbox}}<Rev
leads to unindented paragraphs, even if \cs{parindent} is 
not zero. 


\spoint[indent:scheme] An indentation scheme

The above idea of letting the indentation box be removed
\howto Control indentation systematically\par
by \cs{everypar} can be put to use in a systematic approach
to indentation, where two conditionals
\Ver>\newif\ifNeedIndent %as a rule
\newif\ifneedindent %special cases<Rev
control whether paragraphs should indent as a rule, and
whether in special cases indentation is needed.
This section is taken from~\cite[E3].

We take a fixed \cs{everypar}:
\Ver>\everypar={\ControlledIndentation}<Rev
which executes in some cases the macro \cs{RemoveIndentation}
\Ver>
\def\RemoveIndentation{{\setbox0=\lastbox}}<Rev
The implementation of \cs{ControlledIndentation} is:\Ver>
\def\ControlledIndentation
   {\ifNeedIndent \ifneedindent 
                  \else \RemoveIndentation\needindenttrue \fi
    \else \ifneedindent \needindentfalse
          \else         \RemoveIndentation
    \fi   \fi}<Rev
In order to regulate indentation for a whole document,
the user now once specifies, for instance,
\Ver>\NeedIndenttrue<Rev to indicate that, in principle,
all paragraphs should indent.
Macros such as \cs{section} can then prevent
indentation in individual cases:
\Ver>\def\section#1{ ... \needindentfalse}<Rev
    

\spoint[skip:scheme] A paragraph skip scheme

The use of \cs{everypar} to control indentation,
\howto Control vertical white space systematically\par
as was sketched above, can be extended to the
paragraph skip.

A visible white space between paragraphs can be
created by the \cs{parskip} parameter, but, once this
parameter has been set to some value, it is difficult
to prevent paragraph skip in certain places elegantly.
Usually, white space above and below environments
and section headings should be specifiable independently
of the paragraph skip. This section sketches an
approach where \cs{parskip} is set to zero directly
above and below certain constructs, while the \cs{everypar}
is used to restore former values. This section is
taken from~\cite[E4].

First of all, here are two tools. The control sequence
\cs{csarg} will be used only inside other macros;
a typical call will look like
\Ver>\csarg\vskip{#1Parskip}<Rev
Here is the definition:\Ver>
\def\csarg#1#2{\expandafter#1\csname#2\endcsname}<Rev
Next follows a generalization of \cs{vskip}: the macro
\cs{vspace} will not place its argument if the previous glue item
is larger; otherwise it will eliminate the preceding
glue, and place its argument.\Ver>
\newskip\tempskipa
\def\vspace#1{\tempskipa=#1\relax
    \ifvmode \ifdim\tempskipa<\lastskip 
             \else \vskip-\lastskip \vskip\tempskipa \fi
    \else    \vskip\tempskipa \fi}<Rev
    
Now assume that any construct \n{foo} 
with surrounding white space
starts and ends with macro calls \ver>\StartEnvironment{foo}> and
\ver>\EndEnvironment{foo}> respectively.
Furthermore, assume that to this environment there correspond
three glue registers:
the \cs{fooStartskip} (glue
above the environment), \cs{fooParskip} (the paragraph skip
inside the environment), and the \cs{fooEndskip} (glue below
the environment).

For restoring the value of the paragraph skip
a conditional and a glue register are needed:\Ver>
\newskip\TempParskip \newif\ifParskipNeedsRestoring<Rev
The basic sequence for the
starting and ending macros for the environments is then
\Ver>
\TempParskip=\parskip\parskip=0cm\relax
\ParskipNeedsRestoringtrue<Rev

The implementations can now be given as:\Ver>
\def\StartEnvironment#1{\csarg\vspace{#1Startskip}
    \begingroup % make changes local
    \csarg\TempParskip{#1Parskip} \parskip=0cm\relax
    \ParskipNeedsRestoringtrue}
\def\EndEnvironment#1{\csarg\vspace{#1Endskip}
    \endgroup % restore global values
    \ifParskipNeedsRestoring
    \else \TempParskip=\parskip \parskip=0cm\relax
          \ParskipNeedsRestoringtrue
    \fi}<Rev
The \cs{EndEnvironment} macro needs a little comment:
if an environment is used inside another one, and
it occurs before the first paragraph in that environment,
the value of the paragraph skip for the outer environment
has already been saved. Therefore no further actions are
required in that case.

Note that both macros start with a vertical skip. This prevents
the \cs{begingroup} and \cs{endgroup} statements from
occurring in a paragraph. 

We now come to the main point: if necessary, the
\cs{everypar} will restore the value of the paragraph skip.
\Ver>\everypar={\ControlledIndentation\ControlledParskip}
\def\ControlledParskip
   {\ifParskipNeedsRestoring
       \parskip=\TempParskip \ParskipNeedsRestoringfalse
    \fi}<Rev

\subject[par:end]  Paragraph End

\TeX's mechanism for ending a paragraph is ingenious and effective.
This chapter explains the mechanism, the role of \cs{par} in it,
and it gives a number of practical remarks.

\invent
\item par 
      Finish off a paragraph and go into vertical mode.

\item endgraf
      Synonym for \cs{par}: \ver>\let\endgraf=\par>

\item parfillskip 
      Glue that is placed between the last          
      element of the paragraph and the line end.
      Plain \TeX\ default:~\n{0pt plus 1fil}.
\inventstop

\point The way paragraphs end

A paragraph is terminated by the primitive \cs{par} command, 
\term paragraph end\par
which can
be explicitly typed by the user (or inserted by
a macro expansion):\Ver>... last words.\par
A new paragraph ...<Rev
It can be implicitly generated in the input processor of \TeX\
by an empty line (see Chapter~\ref[mouth]):\Ver>
... last words.

A new paragraph ...<Rev 
The \cs{par} can be inserted because a \gr{vertical command}
occurred in unrestricted horizontal mode:\Ver>
... last words.\vskip6pt
A new paragraph ...<Rev
Also, a paragraph ends if a closing brace is found
in horizontal mode inside \cs{vbox}, \cs{insert}, or \cs{output}.

After the \cs{par} command \TeX\ goes into vertical mode
and exercises the page builder (see page~\pgref[par:page:build]).
If the \cs{par} was inserted because a vertical command occurred in
horizontal mode, the vertical command is then examined anew.
The \cs{par} does not insert any vertical
glue or penalties itself. A~\cs{par} command also clears
the paragraph shape parameters (see Chapter~\ref[par:shape]).

\spoint The \cs{par} command and the \cs{par} token

It is important to distinguish between the \cs{par} token
and the primitive \cs{par} command that is the initial meaning of
that token. The \cs{par} token is inserted when the input
processor sees an empty
line, or when the execution processor finds a \gram{vertical command}
in horizontal mode;
the \cs{par} command is what actually closes off a paragraph.
Decoupling the token and the command is an important tool
for special effects in paragraphs (see some examples in
Chapters \ref[boxes] and~\ref[rules]).


\spoint Paragraph filling: \cs{parfillskip}

After the last element of the paragraph \TeX\ implicitly inserts
the equivalent of
\csterm parfillskip\par
\Ver>\unskip \penalty10000 \hskip\parfillskip<Rev
The \cs{unskip} serves to remove any spurious glue at the
paragraph end, such as the space generated by the
line end if the \cs{par} was inserted by the input processor.
For example:\message{check unsplit paragraph example}
\Ver>end.

\noindent Begin<Rev results in the tokens
\disp\n{end.\char32}\cs{par} \n{Begin}\>
With the sequence inserted by the \cs{par} this becomes
\disp\n{end.\char32}\ver>\unskip\penalty10000\hskip ...>\>
which in turn gives
\disp\ver>end.\penalty ...>\>

The \cs{parfillskip} is in plain \TeX\ first-order infinite
(\n{0pt plus 1fil}),
so ending a paragraph with \ver.\hfil$\bullet$\par.
will give a bullet halfway between the last word and the
line end; with \ver.\hfill$\bullet$\par. it will be
flush right.


\point Assorted remarks

\spoint Ending a paragraph and a group at the same time

If a paragraph is set in a group, 
it may be necessary to ensure that the \cs{par} ending
the paragraph occurs inside the group.
The parameters influencing the typesetting of the paragraph,
such as the \cs{leftskip} and the \cs{baselineskip},
are only looked at when the paragraph is finished.
Thus finishing off a paragraph with 
\Ver>... last words.}\par<Rev causes the values to be used
that prevail outside the group, instead of those inside.

Better ways to end the paragraph are
\Ver>... last words.\par}<Rev or
\Ver>... last words.\medskip}<Rev
In the second example the vertical command \cs{medskip}
causes the \cs{par} token to be inserted.

\spoint Ending a paragraph with \cs{hfill}\cs{break}

The sequence \ver.\hfill\break. is a way to force
a `newline' inside a paragraph. If you end a paragraph
with this, however, you will probably
get an \ver-Underfull \hbox- error.
Surprisingly, the underfull box is not the broken line
\ldash after all, that one was filled \rdash 
but a completely empty box following it (actually, it
does contain the \cs{leftskip} and \cs{rightskip}). 

What happens?
The paragraph ends with \Ver>\hfill\break\par<Rev
which turns into 
\Ver>\hfill\break\unskip\nobreak\hskip\parfillskip<Rev
The \cs{unskip} finds no preceding glue, so the \cs{break}
is followed by a penalty item and a glue item, both of
which disappear after the line break has been chosen at the
\cs{break}.
However, \TeX\ has already decided that there should be an extra
line, that is, an \ver.\hbox to \hsize.. And there is nothing
\alt
to fill it with, so an underfull box results.

\spoint Ending a paragraph with a rule

See page~\pgref[par:leaders:end] for paragraphs ending with 
rule leaders instead of the default \cs{parfillskip}
white space.

\spoint No page breaks in between paragraphs

The \cs{par} command does not insert any glue in the
\howto Prevent page breaks in between paragraphs\par
vertical list, so
in the sequence
\Ver> ... last words.\par \nobreak \medskip 
\noindent First words ...<Rev
no page breaks will occur  between the paragraphs.
The vertical list generated is
\Ver>
\hbox(6.94444+0.0)x ...       % last line of paragraph
\penalty 10000                % \nobreak
\glue 6.0 plus 2.0 minus 2.0  % \medskip
\glue(\parskip) 0.0 plus 1.0  % \parskip
\glue(\baselineskip) 5.05556  % interline glue
\hbox(6.94444+0.0)x ...       % first line of paragraph<Rev
\TeX\ will not break this vertical list above the \cs{medskip},
because the penalty value prohibits it; it will not break
at any other place, because it can only break at glue if
that glue is preceded by a  non-discardable item.

\spoint Finite \cs{parfillskip}

In plain \TeX, \cs{parfillskip} has a (first-order) infinite
stretch component. All other glue in the last line of a 
paragraph will then be set at natural width.
If the \cs{parfillskip} has only finite (or possibly zero)
stretch, other glue will be stretched or shrunk.
A display formula in a paragraph with such a last line
will be surrounded by \cs{abovedisplayskip} and \cs{belowdisplayskip},
even if \cs{abovedisplayshortskip} glue would be in order.

The reason for this is that glue setting is slightly
machine-dependent, and any such processes should be kept
out of \TeX's global decisions.

\spoint A precaution for paragraphs that do not indent

If you are setting a text with both the paragraph indentation
and the white space between paragraphs zero, you run the risk
that the start of a new paragraph may be indiscernible when
the last line of the previous paragraph ends almost
or completely flush right.
A~sensible precaution for this is to set the \cs{parfillskip}
to, for instance \Ver> \parfillskip=1cm plus 1fil<Rev
instead of the usual \n{0cm~plus~1fil}.

On the other hand, you may let yourself be convinced by
\cite[Tsch] that paragraphs should always indent.

\subject[par:shape] Paragraph Shape

This chapter treats the parameters and commands that influence the
\term paragraph shape\par
shape of a paragraph.

\invent
\item parindent 
      Width of the indentation box added in front of a paragraph.
      Plain \TeX\ default:~\n{20pt}.

\item hsize 
      Line width used for typesetting a paragraph.
      Plain \TeX\ default:~\n{6.5in}.

\item leftskip 
      Glue that is placed to the left of all lines of a paragraph.


\item rightskip 
      Glue that is placed to the right of all lines of a paragraph.


\item hangindent   
      If positive, this indicates indentation from the left margin; 
      if negative, this is the negative of the indentation 
      from the right margin.

\item hangafter   
      If positive, this denotes the number of lines 
      before indenting starts; 
      if negative, the absolute value of this is the number 
      of indented lines starting with the first line of the paragraph.
      Default:~\n1.

\item parshape
      Command for general paragraph shapes.

\inventstop


\point The width of text lines

When \TeX\ has finished absorbing a paragraph, 
\term line width\par
it has formed a horizontal list, starting with an indentation
box, and ending with \cs{parfillskip} glue.
This list is then broken into lines of length \cs{hsize}.
\csterm hsize\par\csterm leftskip\par\csterm rightskip\par
Each line of a paragraph is padded left and right with
certain amounts of glue, the \cs{leftskip} and \cs{rightskip},
which are taken into account in reaching \cs{hsize}.

The values of \cs{leftskip} and \cs{rightskip} are taken 
into account in the line-breaking algorithm.
Thus the main point about the \cs{raggedright} 
\csterm raggedright\par
macro in plain \TeX\ and the \LaTeX\ `flushleft'
environment is that they
set the \cs{rightskip} to  zero plus some stretch.

The commands \cs{parshape} and \cs{hangindent}
also affect line width. They work by altering the
\cs{hsize} and afterwards shifting the boxes 
containing the lines.

\point Shape parameters

\spoint Hanging indentation

\message{twolines?}
A simple, and frequently occurring, paragraph shape is that
\term hanging indentation\par
\csterm hangafter\par\csterm hangindent\par
with a number of starting or trailing lines indented.
\TeX\ can realize such shapes using two parameters:
\cs{hangafter} and \cs{hangindent}.
Both can assume positive and negative values.

The \cs{hangindent} controls the amount of indentation:
\itemlist\item \cs{hangindent}${}>0$: the paragraph
is indented at the left margin by this amount.
\item\cs{hangindent}${}<0$: the paragraph is indented
at the right margin by the absolute value of this amount.\>
\def\exnul{\leftskip=0pt \rightskip=0pt \relax}
For example (assume \cs{parindent=0pt}),
\disp\leavevmode\message{Check parshape example!}%
\hbox{\Distance:verbatimwhiteleft=0pt
$\vcenter{\snugbox{\Ver>
 a a a a a a a a a a a a ...

 \hangindent=10pt
 a a a a a a a a a a a a ...

 \hangindent=-10pt
 a a a a a a a a a a a a ...<Rev}}$\Spaces:2 gives \Spaces:2
$\vcenter{\parindent0pt \setbox0\hbox{a a a a a}\hsize\wd0
 \leftskip=0pt %\parskip6pt 
 a a a a a a a a a a a a \dots\par%\vskip\baselineskip
 \hangindent=10pt
 a a a a a a a a a a a a \dots\par%\vskip\baselineskip
 \hangindent=-10pt
 a a a a a a a a a a a a \dots\par}$
}\>
The default value of \cs{hangindent} is~\n{0pt}.

The \cs{hangafter} parameter determines the number of
lines that is indented:
\itemlist\item \cs{hangafter}${}\geq0$: 
after this number of lines the rest of the lines will be
indented; in other words, this many lines from the
start of the paragraph will not be indented.
\item \cs{hangafter}${}<0$: the absolute value of this
is the number of lines that will be indented starting
at the beginning of the paragraph.\>
For example,
\disp\leavevmode\hbox{\Distance:verbatimwhiteleft=0pt
$\vcenter{\snugbox{\Ver>
 a a a a a a a a a a a a ...

 \hangindent=10pt \hangafter=2
 a a a a a a a a a a a a ...

 \hangindent=10pt \hangafter=-2
 a a a a a a a a a a a a ...<Rev}}$ \Spaces:2 looks like \Spaces:2
$\vcenter{\parindent0pt \setbox0\hbox{a a a a a}\hsize\wd0
 \leftskip=0pt %\parskip6pt
 a a a a a a a a a a a a \dots\par%\vskip\baselineskip
 \hangindent=10pt \hangafter=2
 a a a a a a a a a a a a \dots\par%\vskip\baselineskip
 \hangindent=10pt \hangafter=-2
 a a a a a a a a a a a a \dots\par}$
}\>
The default value for \cs{hangafter} is~\n1.

With both parameters having the possibility to
be positive and negative,
four ways of hanging indentation result. See below
for hanging indentation into the margin (`outdent').

Hanging indentation is implemented as follows.
The amount of hanging indentation is subtracted
from the \cs{hsize} for the lines that indent;
after the paragraph has been broken into horizontal
boxes, the lines that should indent on the left are
shifted right.

Regular indentation of size \cs{parindent} is not
influenced by hanging indentation. Thus you should
start a paragraph with hanging indentation 
explicitly by~\cs{noindent} if the extra
indentation is unwanted.

The default values of \cs{hangindent} and \cs{hangafter} are
restored after every \cs{par} command.

\spoint General paragraph shapes: \cs{parshape}

Quite general paragraph shapes can be implemented
\csterm parshape\par
using \cs{parshape}. With this command line lengths and indentation
for the first $n$ lines
of a paragraph can be specified. Thus this command
takes $2n+1$ parameters: the number of lines $n$, followed
by $n$ pairs of an indentation   and a line length.
\disp \cs{parshape}\gr{equals}
    $n$ $i_1$ $\ell_1$ $\ldots$ $i_n$ $\ell_n$\>
The   specification for the last line is repeated if the
paragraph following has more than $n$ lines. If there are fewer
than $n$ lines the remaining specifications are ignored.
The default value is (naturally) \cs{parshape${}={}$0}.

A \cs{parshape} command takes precedence over a \cs{hangindent}
if both have been specified. 
%Regular \cs{parindent} indentation
%is suppressed if \cs{parshape} is in effect.
Regular \cs{parindent}, \cs{leftskip}, 
and \cs{rightskip} are still obeyed if \cs{parshape} is in effect.

The \cs{parshape} parameter is, like \cs{hangindent}, \cs{hangafter},
and \cs{looseness} (see Chapter~\ref[line:break]),
cleared after a \cs{par}
command. Since every empty line generates a \cs{par} token,
one should not leave an empty line
between a paragraph shape (or hanging indentation)
declaration and the following paragraph.

The control sequence
\alt
\cs{parshape} is an \gr{internal integer}:
its value is the number of lines $n$ with which
it was set.

\point Assorted remarks

\spoint Centred last lines

Equal stretch and shrink amounts for the \cs{leftskip} and 
\cs{rightskip}
give centred texts, in the sense that each line is
centred. 
For proper centring of the first
and last lines of a paragraph the \cs{parindent} and
\cs{parfillskip} have to be made zero.
However, the margins are ragged.

A surprising application of \cs{leftskip} and \cs{rightskip}
\howto Centre the first/""last line of a paragraph\par
leads to paragraphs with flush margins and a centred
last line.
\Ver>\leftskip=0cm plus 0.5fil \rightskip=0cm plus -0.5fil
\parfillskip=0cm plus 1fil<Rev

For all lines of a paragraph but the 
last one the stretch components
add up to zero so the \cs{leftskip} and \cs{rightskip}
inserted are zero.
On the last line the \cs{parfillskip} adds \hbox{\n{plus 1fil}}
of stretch; therefore there is a total of
\hbox{\n{plus 0.5fil}} of stretch at both the left and right
end of the line.

It would have been incorrect to specify
\Ver>\leftskip=0cm plus 0.5fil \rightskip=0cm minus 0.5fil<Rev
\TeX\ gives an error about this: it complains about
`infinite shrinkage'.

Centring not only the last line, but also the
first line of a paragraph can be done by
the parameter settings
\Ver>\parindent=0pt \everypar{\hskip 0pt plus -1fil}
\leftskip=0pt plus .5fil
\rightskip=0pt plus -.5fil<Rev
This time a horizontal skip inserted by \cs{everypar}
combines with the \cs{leftskip} to give the same
amount of stretchability on both sides of the
first line of the paragraph.

\spoint Indenting into the margin

Suppose you want a hanging indent of \n{1cm} {\sl into\/}
\howto Indent into the margin\par
the left margin after the first two lines of a paragraph. 
Specifying \ver/\hangindent=-1cm/ will give
a hanging indentation of one centimetre from the {\sl right\/}
margin, so another approach is necessary. The following does the
job:
\Ver> \leftskip=-1cm \hangindent=1cm \hangafter=-2<Rev
The only problem with this is that
the leftskip needs to be reset after the paragraph.
Suitable redefinition of \cs{par} removes this objection:
\Ver>
\def\hangintomargin{\bgroup
    \leftskip=-1cm \hangindent=1cm \hangafter=-2
    \def\par{\endgraf\egroup}}<Rev
The redefinition of \cs{par} is here local to the paragraph that
should be outdented.

Another, elegant, solution uses \cs{parshape}:
\Ver> 
\dimen0=\hsize \advance\dimen0 by 1cm
\parshape=3        % three lines:
    0cm\hsize      % first  line specification
    0cm\hsize      % second line specification
    -1cm\dimen0    % third  line specification<Rev

\spoint Hang a paragraph from an object

The \LaTeX\ format has a macro, \cs{@hangfrom}, to have
\howto Hang a paragraph from an object\par
one paragraph of text hanging from some object, usually a box
or a short line of text. 

\begingroup
\medskip
\def\hangobject{Example \ }
\setbox0=\hbox{\hangobject}
\hangindent \wd0 \noindent \hangobject
This paragraph is an example of the \cs{hangfrom} macro
defined below.
In the \LaTeX\ document
styles, the \cs{@hangfrom} macro (which is similar to this)
is used for multi-line section headings.\par
\endgroup

Consider then the macro \cs{hangfrom}:
\Ver> 
\def\hangfrom#1{\def\hangobject{#1}\setbox0=\hbox{\hangobject}%
    \hangindent \wd0 \noindent \hangobject \ignorespaces}<Rev
Because of the default \cs{hangafter=1}, this 
will produce one line of width \cs{hsize}, after which the
rest of the paragraph will be left indented  by the width of the
\cs{hangobject}.

\spoint Another approach to hanging indentation

Hanging indentation can also be attained by a combination
of shifting the left margin and outdenting.
Itemized lists can for instance be implemented in this manner:
\Ver>
\newdimen\listindent
\def\itemlist{\begingroup
    \advance\leftskip by \listindent
    \parindent=-\listindent}
\def\stopitemlist{\par\endgroup}
\def\item#1{\par\leavevmode
    \hbox to \listindent{#1\hfil}\ignorespaces
    }<Rev
If an item should encompass more than one paragraph, the
implementation could be
\Ver>
\newdimen\listindent \newdimen\listparindent
\def\itemlist{\begingroup
    \advance\leftskip by \listindent
    \parindent=\listparindent}
\def\stopitemlist{\par\endgroup}
\def\item#1{\par\noindent
    \hbox to 0cm{\kern-\listindent #1\hfil}\ignorespaces
    }<Rev
\example
\Ver>\itemlist\item{1.}First item\par
Is two paragraphs long.
\item{2.}Second item.\stopitemlist<Rev
gives
\disp
\def\itemlist{\begingroup
    \advance\leftskip by \parindent
    \parindent=1em\relax}
\def\stopitemlist{\par\endgroup}
\def\item#1{\par\noindent
    \hbox to 0cm{\kern-\parindent #1\hfil}\ignorespaces
    }
\itemlist\item{1.}First item\par
Is two paragraphs long.
\item{2.}Second item.\stopitemlist\>
\>

\spoint Hanging indentation versus \cs{leftskip} shifting

From the above examples it would seem that
hanging indentation and modifying the \cs{leftskip} and \cs{rightskip}
are interchangeable. They are, but only to a certain   extent.
\altt

Setting \cs{leftskip} to some positive value for a paragraph
means that the \cs{hsize} stays the same, but every line
starts with a glue item. Hanging indentation, on the other hand,
is implemented by decreasing the \cs{hsize} value for the
lines that hang, and shifting the finished 
horizontal boxes horizontally in the surrounding vertical list.

The difference between the two approaches becomes visible
mainly in the fact that display formulas are not shifted
when the \cs{leftskip} is altered.
See Chapter~\ref[rules] for an example showing how leaders
are affected by margin shifting.

\spoint More examples

Some more examples of paragraph shapes (effected by
various means) can be found in~\cite[E1]. One example
from that article appears on page~\pgref[varioset].

\subject[line:break]  Line Breaking

This chapter treats line breaking and the concept of `badness' that \TeX\
uses to decide how to break a paragraph into lines,
or where to break a page.
The various penalties contributing to the cost of line breaking
are treated here, as is hyphenation.
Page breaking is treated in Chapter~\ref[page:break].

\invent 
\item penalty 
      Specify desirability of not breaking at this point.

\item linepenalty 
      Penalty value associated with each line break. 
      Plain \TeX\ default:~\n{10}.

\item hyphenpenalty 
      Penalty associated with break at a discretionary item
      in the general case. 
      Plain \TeX\ default:~\n{50}.

\item exhyphenpenalty 
      Penalty for breaking a horizontal line at a discretionary
      item in the special case where the prebreak text is empty. 
      Plain \TeX\ default:~\n{50}.

\item adjdemerits 
      Penalty for adjacent visually incompatible lines. 
      Plain \TeX\ default:~\n{10$\,$000}.

\item doublehyphendemerits 
      Penalty for consecutive lines ending with a hyphen. 
      Plain \TeX\ default:~\n{10$\,$000}.

\item finalhyphendemerits 
      Penalty added when the penultimate line of a 
      paragraph ends with a hyphen. 
      Plain \TeX\ default:~\n{5000}.

\item allowbreak 
      Macro for creating a breakpoint by inserting a
      \cs{penalty0}.

\item pretolerance 
      Tolerance value for a paragraph without hyphenation. 
      Plain \TeX\ default:~\n{100}.

\item tolerance 
      Tolerance value for lines in a paragraph with hyphenation. 
      Plain \TeX\ default:~\n{200}.

\item emergencystretch
      (\TeX3 only)
      Assumed extra stretchability in lines of a paragraph.

\item looseness  
      Number of lines by which this paragraph has to be made longer 
      than it would be ideally.

\item prevgraf  
      The number of lines in the paragraph last
      added to the vertical list.

\item discretionary 
      Specify the way a character sequence is split up at a line break.

\item - 
      Discretionary hyphen; this is
      equivalent to \ver|\discretionary{-}{}{}|.

\item hyphenchar 
      Number of the hyphen character of a font.

\item defaulthyphenchar 
      Value of \cs{hyphenchar} when a font is loaded.
      Plain \TeX\ default:~\n{`\cs{-}}.

\item uchyph 
      Positive to allow hyphenation of words starting with a capital 
      letter.
      Plain \TeX\ default:~\n{1}.

\item lefthyphenmin 
      (\TeX3 only)
      Minimal number of characters before a hyphenation.
      Plain \TeX\ default:~\n{2}.

\item righthyphenmin 
      (\TeX3 only)
      Minimum number of characters after a hyphenation.
      Plain \TeX\ default:~\n{3}.

\item patterns 
      Define a list of hyphenation patterns for the current
      value of \cs{language};  allowed only in \IniTeX.

\item hyphenation 
      Define hyphenation exceptions for the current value of \cs{language}.

\item language 
      Choose a set of hyphenation patterns and exceptions.

\item setlanguage 
      Reset the current language.

\inventstop


\point Paragraph break cost calculation

A paragraph is broken such that the amount $d$ of {\em demerits\/}
associated with breaking it is minimized. 
The total amount of demerits for a paragraph is the sum
of those for the individual lines, plus possibly some extra
penalties. Considering a paragraph as a whole instead of 
breaking it on a line-by-line basis can lead to better
line breaking: \TeX\ can choose to take a slightly less beautiful
line in the beginning of the paragraph in order to avoid
bigger trouble later on.

For each line demerits are calculated from the {\em badness\/}~$b$
of stretching or shrinking the line to the break, and
the {\em penalty\/}~$p$ associated with the break.
The badness is not allowed to exceed a certain prescribed
tolerance.

In addition to the demerits for breaking individual lines,
\TeX\ assigns demerits for the way lines combine; see below.

The implementation of \TeX's paragraph"-breaking algorithm
is explained in~\cite[K:break].

\spoint Badness

From the ratio between the stretch or shrink present in a 
\term badness and line breaking\par
line, and the actual stretch or shrink taken,
the `badness' of breaking a line at a certain point is calculated.
This badness is an important
factor in the process of line breaking.
See page~\pgref[bad:form] for the formula for badness.

In this chapter
badness will only be discussed in the context of line breaking. 
Badness is also computed when a vertical list is stretched
or shrunk (see Chapter~\ref[page:break]). 

The following terminology is used to describe badness:
\description \item tight (3)
is any line that has shrunk with a badness~$b\geq13$, 
that is, by using at least one-half of its amount of shrink
(see page~\pgref[bad:form] for the computation).
\item decent (2)
is any line with a badness~$b\leq12$.
\item loose (1)
is any line that has stretched with a badness~$b\geq13$, 
that is, by using at least one-half of its amount of stretch.
\item very loose (0)
is any line that has stretched with a badness~$b\geq100$, 
that is, by using its full amount of stretch or more. Recall
that glue can stretch, but not shrink more than its
allowed amount.
\>
The numbering is used in trace output (Chapter~\ref[trace]), and
it is also used in the following definition:
if the classifications of two adjacent lines differ by more than~1,
the lines are said to be {\em visually incompatible\/}.
See below for the \cs{adjdemerits} parameter associated with this.

Overfull horizontal and vertical
boxes are passed unnoticed if their excess width
or height is less than \cs{hfuzz} or \cs{vfuzz} respectively; 
they are not reported if the badness is less than
\cs{hbadness} or \cs{vbadness} (see Chapter~\ref[boxes]).

\spoint Penalties and other break locations
 
Line breaks can occur at the following places in horizontal
\csterm penalty\par
\term breakpoints in horizontal lists\par
\term penalties in horizontal lists\par
lists:
\enumerate \item At a penalty. The penalty value is the
`aesthetic cost' of breaking the line at that place.
Negative penalties are considered as bonuses.
A~penalty of $10\,000$ or more inhibits, and a penalty
of $-10\,000$ or less forces, a~break.

Putting more than one penalty
in a row is equivalent to putting just the one with the
minimal value, because that one is the best candidate for line breaking.

Penalties in horizontal mode are inserted by the user (or a
user macro). The only exception is the \cs{nobreak}
inserted before the \cs{parfillskip} glue.

\item At a glue, if it is not part of a math formula, and
if it is preceded by a non-discardable item (see Chapter~\ref[hvmode]).
There is no penalty associated with breaking at glue.

The condition about the non-discardable precursor is necessary, 
because otherwise breaking in between  two pieces of glue would
be possible, which would cause ragged edges to the paragraph.

\item At a kern, if it is not part of a math formula
and if it is followed by  glue.
There is no penalty associated with breaking at a~kern.

\item At a math-off, if that is followed by glue.
Since math-off
(and math-on) act as kerns (see Chapter~\ref[math]),
this is very much like the previous case.
There is no penalty associated with breaking at a~math-off.

\item At a discretionary break. The penalty
is the \cs{hyphenpenalty} or the \cs{exhyphenpenalty}.
This is treated below.
\enumeratestop

Any discardable material following the break \ldash glue, kerns,
math-on/""off and penalties \rdash  is discarded. If one considers
a line break at glue (kern, math-on{/}off) to occur at the
front end of the glue item, this implies that that piece
of glue disappears in the break.

\spoint Demerits

From the badness of a line and the penalty, if any, the demerits
of the line are calculated. Let $l$ be the value of
\csterm linepenalty\par
\cs{linepenalty}, $b$~the badness of the line,
$p$~the penalty at the break; then the demerits $d$
\term demerits\par
are given by
\disp$\displaystyle d=\cases{(l+b)^2+p^2&if $0\leq p<10\,000$\cr
           (l+b)^2-p^2&if $-10\,000<p<0$\cr
           (l+b)^2    &if $p\leq-10\,000$\cr}$\>

Both this formula and the one for the badness are described
\alt
in \cite[K:break] as `quite arbitrary',
but they have been shown  to lead to
good results in practice.

The demerits for a paragraph are the sum of the demerits for
the lines, plus \itemlist
\item the \cs{adjdemerits} for any two
   \csterm adjdemerits\par
   adjacent lines that are not visually compatible (see above),
\item \cs{doublehyphendemerits} for any two
   \csterm doublehyphendemerits\par
   consecutive lines ending with a hyphen, and the
\item \cs{finalhyphendemerits}
   \csterm finalhyphendemerits\par
   if the penultimate line of a paragraph
   ends with a hyphen.\itemliststop

At the start of a paragraph \TeX\ acts as if
there was a preceding line which was `decent'.
Therefore \cs{adjdemerits} will be added if the first
line is `very loose'. Also, the last line
of a paragraph is ordinarily also `decent'
\ldash all spaces are set at natural width
owing to the infinite stretch in the \cs{parfillskip} \rdash 
so \cs{adjdemerits} are added if
the preceding line is `very loose'.

Note that the penalties at which a line break
is chosen weigh about as heavily as the badness of
the line, so they can be relatively small.
However, the three extra demerit parameters
have to be of the order of the square of 
penalties and badnesses to weigh equally heavily.

\spoint The number of lines of a paragraph

After a paragraph has been completed (or partially
completed prior to a display), the variable \cs{prevgraf}
\csterm prevgraf\par
records the number of lines in the paragraph.
By assigning to this variable \ldash and
because this is a \gr{special integer}
such an assignment is automatically global \rdash 
\TeX's decision processes can be influenced.
This may be useful in combination with hanging indentation
or \cs{parshape} specifications (see Chapter~\ref[par:shape]).

Some direct influence of the line"-breaking process
on the resulting number of lines exists. One factor
is the \cs{linepenalty} which is included in the demerits 
of each line. By increasing the line penalty \TeX\ can be
made to minimize the number of lines in a paragraph.

Deviations from the optimal number of lines, that is, the
\csterm looseness\par
number of lines stemming from the optimal way of breaking a
paragraph into lines, can be forced by the user by means
of the \cs{looseness} parameter. This parameter, which is
reset every time the shape parameters 
are cleared (see Chapter~\ref[par:shape]), 
indicates by how many lines the current
paragraph should be made longer than is optimal. A~negative
value of \cs{looseness} will attempt to make the paragraph shorter
by a number of lines that is the absolute value of the parameter.

\TeX\ will still observe the values
of \cs{pretolerance} and \cs{tolerance} (see below)
when lengthening or shortening a paragraph under influence
of \cs{looseness}.
Therefore,
\TeX\ will only lengthen or shorten a paragraph for as far
as is possible without exceeding these parameters.


\spoint[between:lines] Between the lines

\TeX's
paragraph mechanism packages lines into horizontal boxes
that are appended to the surrounding vertical list.
The resulting sequence of vertical items is then a 
repeating sequence of
\itemlist\item a box containing a line of text,
\item possibly migrated vertical material (see page~\pgref[migrate]),
\item a penalty item reflecting the cost of a page break
      at that point, which is normally the \cs{interlinepenalty}
      (see Chapter~\ref[page:break]), and
\item interline glue, which is calculated automatically
      on basis of the \cs{prevdepth} (see Chapter~\ref[baseline]).
\itemliststop

\point The process of breaking

\TeX\ tries to break paragraphs in such a way that 
\term paragraphs, breaking into lines\par
the badness of each line does not exceed a certain tolerance.
If there exists more than one solution to this, the one with
the fewest demerits is taken.

By setting \cs{tracingparagraphs} to a positive value,
\csterm tracingparagraphs\par
\TeX\ can be made to report the calculations of the
paragraph mechanism in the log file. Some implementations of \TeX\
may have this option disabled to make \TeX\ run faster.

\spoint Three passes

First an attempt is made to split the paragraph into lines
without hyphenating, that is, without inserting discretionary
hyphens. This attempt succeeds if none of the
\csterm pretolerance\par
lines has a badness exceeding \cs{pretolerance}.

Otherwise, a second pass is made, inserting discretionaries
and using \cs{tolerance}.
\csterm tolerance\par
If \cs{pretolerance} is negative, the first pass is omitted.

\TeX\ can be made to make a third pass if the first and
second pass fail.
\csterm emergencystretch\par
If \cs{emergency\-stretch} is a positive dimension, 
\TeX\ will assume this much extra stretchability 
in each line when badness and demerits are calculated.
Thus solutions that only slightly exceeded the given
tolerances will now become feasible.
However, no glue of size \cs{emergencystretch} is
actually present, so underfull box messages
may still occur.

\spoint Tolerance values

How much
trouble \TeX\ will have typesetting a piece of text
depends partly on the tolerance value.
Therefore it is sensible to have some idea of
what badness values mean in visual terms.

For lines that are stretched, the badness is
100 times the cube of the stretch ratio.
A~badness of 800 thus means that the stretch ratio
is~2.
If the space is,
\alt
as in the ten-point Computer Modern Font,
\Ver>3.33pt plus 1.67pt minus 1.11pt<Rev
a badness of 800 means that spaces have been stretched to
\disp \n{3.33pt}${}+2\times{}$\n{1.67pt}${}={}$\n{6.66pt}\>
that is, to exactly double their natural size.
It is up to you to decide whether this is too large.

\point Discretionaries

A discretionary item \ver-\discretionary{..}{..}{..}-
\term discretionary item\par\csterm discretionary\par
marks a place where a word can be broken.
Each of the three arguments is a \gr{general text}
(see Chapter~\ref[gramm]):
they are, in sequence,
\itemlist \item the {\em pre-break\/} text, which is appended
to the part of the word before the break,
\item the {\em post-break\/} text, which is prepended to the part
of the word after the break, and
\item the {\em no-break\/} text, which is used if the word
is not broken at the discretionary item.\itemliststop
For example: \ver>ab\discretionary{g}{h}{cd}ef>
is the word \hbox{\n{abcdef}}, but it can be hyphenated
\alt
with \n{abg} before the break and \n{hef} after.
Note that there is no automatic hyphen character.

All three texts may contain any sorts of tokens,
but any primitive commands and macros
should expand to boxes, kerns, and characters.

\spoint Hyphens and discretionaries

Internally, \TeX\ inserts the equivalent of
\csterm hyphenchar\par\term hyphen character\par
\Ver>\discretionary{\char\hyphenchar\font}{}{}<Rev
at every place where a word can be broken. No
such discretionary is inserted if \ver>\hyphenchar\font>
is not in the range 0--255, or if its position in the
font is not filled.
When a font is loaded, its \cs{hyphenchar} value
\csterm defaulthyphenchar\par
is set to \cs{defaulthyphenchar}. The \cs{hyphenchar}
value can be changed after this.

In plain \TeX\ the \cs{defaulthyphenchar} has the value~\ver>`\->, so
for all fonts character~45 (the \ascii\ hyphen character)
is the hyphen sign, unless
it is specified otherwise.

The primitive command \ver|\-| (called a `discretionary hyphen')
\csterm -\par\term discretionary hyphen\par
is equivalent to the above
\ver|\discretionary{\char\hyphenchar\font}{}{}|.
Breaking at such a discretionary, whether inserted implicitly
by \TeX\ or explicitly by the user, has
\csterm hyphenpenalty\par
a cost of \cs{hyphenpenalty}.


In unrestricted horizontal mode an empty discretionary
\cs{disc\-re\-tio\-na\-ry}\ver-{}{}{}-
is automatically inserted after characters
whose character code is the \cs{hyphenchar} value 
of the font, thus enabling hyphenation at that point.
The penalty for breaking a line at
such a discretionary with an empty pre-break text
\csterm exhyphenpenalty\par
is \cs{exhyphenpenalty}, that is, the `explicit hyphen' penalty.

If a word contains
discretionary breaks, for instance 
because of  explicit hyphen characters,
\TeX\ will not consider it for further hyphenation.
People have solved the ensuing problems by tricks
such as
\howto Enable hyphenation of a word containing a hyphen\par
\Ver>\def\={\penalty10000 \hskip0pt -\penalty0 \hskip0pt\relax}
... integro\=differential equations...<Rev
The skips before and after the hyphen lead \TeX\ into
treating the first and second half of the 
compound expression as separate words; the penalty
before the first skip inhibits breaking before the hyphen.

\spoint Examples of discretionaries

Languages such as German or Dutch have words that change
\term languages\par
spelling when hyphenated (German: `\hbox{backen}'
becomes `\hbox{bak-ken}'; Dutch: `\hbox{autootje}'
becomes `\hbox{auto-tje}'). This problem can be solved
with \TeX's discretionaries. 

For instance, for German (this is inspired by~\cite[Partl]):
\Ver>\catcode`\"=\active
\def"#1{\ifx#1k\discretionary{k-}{k}{ck}\fi}<Rev
which enables the user to write \ver>ba"ken>.

In Dutch there is a further problem which allows a nice
systematic solution. Umlaut characters (`trema' is the
Dutch term) should often
disappear in a break, for instance `\hbox{na"apen}'
hyphenates as `\hbox{na-apen}', and `\hbox{onbe"invloedbaar}'
hyphenates as `\hbox{onbe-invloedbaar}'. A solution
(inspired by~\cite[Babel]) is
\Ver>\catcode`\"=\active
\def"#1{\ifx#1i\discretionary{-}{i}{\"\i}%
        \else  \discretionary{-}{#1}{\"#1}\fi}<Rev
which enables the user to type \ver>na"apen> and
\ver>onbe"invloedbaar>.

\point Hyphenation

\TeX's hyphenation algorithm uses a list of patterns to
\term hyphenation\par
determine at what places a word that is a candidate for
hyphenation can be broken.
Those aspects of hyphenation connected with these
patterns are
treated in appendix~H of \TeXbook;
the method of generating hyphenation patterns automatically
is described in~\cite[Liang]. People have been known
to generate lists of patterns by hand; 
see for instance~\cite[Vas:add]. Such hand-generated lists
may be superior to automatically generated lists.

Here it will mainly be described how \TeX\ declares a word to
be a candidate for hyphenation. The  problem here is
how to cope with punctuation and things such as quotation marks
that can be attached to a word. Also, {\em implicit kerns\/},
that is, kerns inserted because of font information,
must  be handled properly.

\spoint Start of a word

\TeX\ starts at glue items (if they are not in math mode)
looking for a {\em starting letter\/} of a word:
a character with non-zero \cs{lccode}, or a ligature starting
with such a character (upper/""lowercase codes are explained
on page~\pgref[uc/lc]).
Looking for this starting letter,
\TeX\ bypasses any implicit kerns, and
characters with zero \cs{lccode} (this includes,
for instance, punctuation and quotation marks), 
or ligatures starting with
such a character.

If no suitable starting letter turns up, that is, if
something is found that is not a character or ligature,
\TeX\ skips to the next glue, and starts this algorithm anew.
Otherwise a trial word is collected consisting of
all following characters with non-zero \cs{lccode}
from the same font as the starting letter, or ligatures consisting
completely of such characters. Implicit kerns are allowed
between the characters and ligatures.

If the starting letter is from a font for which the value
of \cs{hyphenchar} is invalid, or for which this character
does not exist, hyphenation is abandoned for this word.
If the starting letter is an uppercase letter (that is,
it is not equal to its own \cs{lccode}), \TeX\ will
\csterm uchyph\par
abandon hyphenation unless \cs{uchyph} is positive.
The default value for this parameter is~1  in
plain \TeX,
implying that capitalized words are subject to hyphenation.

\spoint End of a word


Following the trial word can be characters (from another
font, or with zero \cs{lccode}), ligatures or implicit kerns.
After these items, if any, must follow
\itemlist\item glue or an explicit kern,
\item a penalty,
\item a whatsit, or
\item a \cs{mark}, \cs{insert}, or \cs{vadjust} item.
\itemliststop
In particular, the word will not be hyphenated if it is
followed by a \itemlist\item box, \item rule, \item math
formula, or \item discretionary item.\itemliststop

Since discretionaries are inserted after the \cs{hyphenchar}
of the font, occurrence of this character inhibits further
hyphenation. Also, placement of accents is implemented using
explicit kerns (see Chapter~\ref[char]), so any \cs{accent}
command is considered to be the end of a word, and inhibits
hyphenation of the word.

\spoint \TeX2 versus \TeX3

There is a noticeable difference in the treatment of
\term \TeX\ version 3\par
hyphenated fragments between \TeX2 and \TeX3. 
\TeX2 insists that the part before the break should be
at least two characters, and the part after the break three
characters, long.
Typographically this is a sound decision: this way
there are no two-character pieces of a word stranded at the
end or beginning of the line. Both before and after the break
there are at least three characters.

In \TeX3 two integer parameters have been introduced to control
the length of these fragments:
\csterm lefthyphenmin\par\csterm righthyphenmin\par
\cs{lefthyphenmin} and \cs{righthyphenmin}. These are
set to 2 and~3 respectively in the plain format for \TeX3.
If the sum of these two is 63 or more, all hyphenation is
suppressed.

Another addition in \TeX3,
the possibility to have several sets of hyphenation patterns,
is treated below.

\spoint Patterns and exceptions

The statements \disp\cs{patterns}\gr{general text}\nl
\cs{hyphenation}\gr{general text}\dispstop
are \gr{hyphenation assignment}s, which are
\csterm hyphenation\par\csterm patterns\par
\gr{global assignment}s.
The \cs{patterns} command, which specifies a list
of hyphenation patterns, is allowed only in \IniTeX\
(see Chapter~\ref[TeXcomm]),
and all patterns must be specified before the first
paragraph is typeset.

Hyphenation exceptions can be specified at any time
\howto Specify exceptional hyphenations\par
with statements such as
\Ver>\hyphenation{oxy-mo-ron gar-goyle}<Rev
which specify locations where a word may be hyphenated.
Subsequent \cs{hyphenation} statements are cumulative.

In \TeX3 these statements are taken to hold for the
language that is the current value of the \cs{language}
parameter.

\point Switching hyphenation patterns

When typesetting paragraphs, \TeX\ (version~3) can use several
\alt
\term languages\par
sets of patterns and hyphenation exceptions, for at most 256
languages. 

If a \cs{patterns} or \cs{hyphenation}
command is given (see above), \TeX\ stores the patterns or exceptions
\csterm language\par
under the current value of the \cs{language} parameter.
The \cs{patterns} command is only allowed in \IniTeX, and
patterns must be specified before any typesetting is done.
Hyphenation exceptions, however, can
be specified cumulatively, and not only in \IniTeX.

In addition to the \cs{language} parameter, 
\term current language\par
which can be set by the user, \TeX\ has internally a `current
language'. This is set to zero at the start of every paragraph.
For every character that is added to a paragraph
the current language is compared with the value of \cs{language},
and if they differ  a whatsit element is added to the horizontal
list, resetting the current language to the value of \cs{language}.

At the start of a paragraph, this whatsit is inserted
\altt
after the \cs{everypar} tokens, but \cs{lastbox}
can still access the indentation box.

As an example, suppose that a format has been created such that
language~0 is English, and language~1 is Dutch. English hyphenations
will then be used if the user does not specify otherwise;
if a job starts with \Ver>\language=1<Rev the whole document
will be set using Dutch hyphenations, because \TeX\ will insert
a command changing the current language at the start of
every paragraph. For example:
\Ver>\language=1
T...<Rev gives
\Ver>
.\hbox(0.0+0.0)x20.0           % indentation
.\setlanguage1 (hyphenmin 2,3) % language whatsit
.\tenrm T                      % start of text<Rev

The whatsit can be inserted explicitly, without changing
\csterm setlanguage\par
the value of \cs{language}, by specifying
\disp\cs{setlanguage}\gr{number}\dispstop
However, this will hardly ever be needed.
One case where it may be necessary is when the contents of
a horizontal box are unboxed to a paragraph: inside the box no
whatsits are added automatically, since inside such a box
no hyphenation can take place.
See page~\pgref[wide:vbox] for another problem with text
in horizontal boxes.

\endinput






