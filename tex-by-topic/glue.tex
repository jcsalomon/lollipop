\subject[glue]  Dimensions and Glue

In \TeX\ vertical and horizontal white space
can have a possibility to adjust itself through `stretching' or
\term glue\par
`shrinking'. An~adjustable white space is called `glue'.
This chapter treats all technical concepts related to
dimensions and glue, and it explains how the badness of stretching or shrinking
a  certain amount is calculated.


\invent
\item dimen 
      Dimension register prefix.

\item dimendef 
      Define a control sequence to be a synonym for
      a~\cs{dimen} register.

\item newdimen 
      Allocate an unused dimen register. 

\item skip 
      Skip register prefix.

\item skipdef 
      Define a control sequence to be a synonym for
      a~\cs{skip} register.

\item newskip
      Allocate an unused skip register.

\item ifdim 
      Compare two dimensions. 

\item hskip  
      Insert in horizontal mode a glue item.

\item hfil 
      Equivalent to 
      \csterm hfil\par
      \ver-\hskip 0cm plus 1fil-.

\item hfilneg 
      Equivalent to 
      \csterm hfilneg\par
      \ver-\hskip 0cm minus 1fil-.

\item hfill 
      Equivalent to 
      \csterm hfill\par
      \ver-\hskip 0cm plus 1fill-.

\item hss 
      Equivalent to 
      \csterm hss\par
      \ver-\hskip 0cm plus 1fil minus 1fil-.

\item vskip  
      Insert in vertical mode a glue item.

\item vfil 
      Equivalent to 
      \csterm vfil\par
      \ver-\vskip 0cm plus 1fil-.

\item vfill 
      Equivalent to 
      \csterm vfill\par
      \ver-\vskip 0cm plus 1fill-.

\item vfilneg 
      Equivalent to 
      \csterm vfilneg\par
      \ver-\vskip 0cm minus 1fil-.

\item vss 
      Equivalent to 
      \csterm vss\par
      \ver-\vskip 0cm plus 1fil minus 1fil-.

\item kern  
      Add a kern item to the current horizontal or vertical list.

\item lastkern 
      If the last item on the current list was a kern, the size of it.

\item lastskip 
      If the last item on the current list was a~glue, the size of it.

\item unkern 
      If the last item of the current list was a~kern, remove it.

\item unskip 
      If the last item of the current list was a~glue, remove it.

\item removelastskip
      Macro to append the negative of the \cs{lastskip}.

\item advance 
      Arithmetic command to add to or subtract from
      a~\gr{numeric variable}.

\item multiply 
      Arithmetic command to multiply a~\gr{numeric variable}.

\item divide 
      Arithmetic command to divide a~\gr{numeric variable}.


\inventstop



\point Definition of \gr{glue} and \gr{dimen}

This section gives
the syntax of the quantities
\gr{dimen} and \gr{glue}. 
In the next section the practical aspects of glue are treated.

Unfortunately the terminology for glue is slightly confusing.
The syntactical quantity~\gr{glue} is a dimension (a distance) with
possibly a stretch and/""or shrink component.
In order to add a glob of `glue' (a white space) to a list one has to
let a \gr{glue} be preceded by commands such as \cs{vskip}.


\spoint Definition of dimensions

A~\gr{dimen} is what \TeX\ expects to see when
it needs to indicate a dimension; it can be positive or negative.
\disp\gr{dimen} $\longrightarrow$ \gr{optional signs}%
     \gr{unsigned dimen}\dispstop
The unsigned part of a \gr{dimen} can be
\disp\gr{unsigned dimen} $\longrightarrow$ \gr{normal dimen}
     $|$ \gr{coerced dimen}\nl
     \gr{normal dimen} $\longrightarrow$ \gr{internal dimen}
     $|$ \gr{factor}\gr{unit of measure}\nl
     \gr{coerced dimen} $\longrightarrow$ \gr{internal glue}
     \dispstop
That is, we have the following three cases:
\itemlist \item an \gr{internal dimen}; this is
 any register or parameter of \TeX\ that has a \gr{dimen} value:
 \disp\PopIndentLevel\gr{internal dimen} $\longrightarrow$
      \gr{dimen parameter}\nl
      \indent $|$ \gr{special dimen} $|$ \cs{lastkern}\nl
      \indent $|$ \gr{dimendef token} $|$ \cs{dimen}\gr{8-bit number}\nl
      \indent $|$ \cs{fontdimen}\gr{number}\gr{font}\nl
      \indent $|$ \gr{box dimension}\gr{8-bit number}\nl
      \gr{dimen parameter} $\longrightarrow$ \cs{boxmaxdepth}\nl
      \indent $|$ \cs{delimitershortfall} $|$ \cs{displayindent}\nl
      \indent $|$ \cs{displaywidth} $|$ \cs{hangindent}\nl
      \indent $|$ \cs{hfuzz} $|$ \cs{hoffset} $|$ \cs{hsize}\nl
      \indent $|$ \cs{lineskiplimit} $|$ \cs{mathsurround}\nl
      \indent $|$ \cs{maxdepth} $|$ \cs{nulldelimiterspace}\nl
      \indent $|$ \cs{overfullrule} $|$ \cs{parindent}\nl
      \indent $|$ \cs{predisplaysize} $|$ \cs{scriptspace}\nl
      \indent $|$ \cs{splitmaxdepth} $|$ \cs{vfuzz}\nl
      \indent $|$ \cs{voffset} $|$ \cs{vsize}
      \>
\item  a dimension denotation, 
 consisting of \gr{factor}\gr{unit of measure},
 for example \ver>0.7\vsize>; or
\item an \gr{internal glue} (see below) 
 coerced to a dimension by omitting
 the stretch and shrink components, for example \cs{parfillskip}.
\itemliststop

A dimension denotation is a somewhat complicated entity:
\itemlist \item a \gr{factor} is an integer denotation,
 a decimal constant denotation (a number with an integral and
 a fractional part),
 or an \gr{internal integer}
 \disp\PopIndentLevel
      \gr{factor} $\longrightarrow$ \gr{normal integer} 
      $|$ \gr{decimal constant}\nl
      \gr{normal integer} $\longrightarrow$ \gr{integer denotation}\nl
      \indent $|$ \gr{internal integer}\nl
      \gr{decimal constant} $\longrightarrow$ \n{.$_{12}$}
      $|$ \n{,$_{12}$}\nl
      \indent $|$ \gr{digit}\gr{decimal constant}\nl
      \indent $|$ \gr{decimal constant}\gr{digit}
      \>
 An internal integer is a parameter that is `really' an
\alt
 integer (for instance, \cs{count0}), and not coerced from a dimension or glue.
 See Chapter~\ref[number]
 for the definition of various kinds of integers.
\item a \gr{unit of measure} can be 
 a \gr{physical unit}, that is, an ordinary unit such as~\n{cm} 
 (possibly preceded by \n{true}),
 an internal unit such as~\n{em}, but also an \gr{internal integer}
 (by conversion to scaled points),
 an \gr{internal dimen}, or an \gr{internal glue}.
 \disp\PopIndentLevel
      \gr{unit of measure} $\longrightarrow$
      \gr{optional spaces}\gr{internal unit}\nl
      \indent $|$ 
      \gr{optional \n{true}}\gr{physical unit}\gr{one optional space}\nl 
      \gr{internal unit} $\longrightarrow$ 
      \n{em}\gr{one optional space}\nl
      \indent $|$ \n{ex}\gr{one optional space}
              $|$ \gr{internal integer}\nl
      \indent $|$ \gr{internal dimen} $|$ \gr{internal glue}
      \dispstop
\itemliststop

Some \gr{dimen}s are called \gr{special dimen}s:\label[special:dimen:list]
\disp\gr{special dimen} $\longrightarrow$ \cs{prevdepth}\nl
     \indent $|$ \cs{pagegoal} $|$ \cs{pagetotal} $|$ \cs{pagestretch}\nl
     \indent $|$ \cs{pagefilstretch} $|$ \cs{pagefillstretch}\nl
     \indent $|$ \cs{pagefilllstretch} $|$ \cs{pageshrink} $|$ \cs{pagedepth}
     \dispstop
An assignment to any of these is
called an \gr{intimate assignment}, and it is automatically
global (see Chapter~\ref[group]). The meaning of these 
dimensions is explained in Chapter \ref[page:break], with the
exception of \cs{prevdepth} which is treated in
Chapter~\ref[baseline].

\spoint Definition of glue

A \gr{glue} is either some form of glue variable, or
a glue denotation with explicitly indicated stretch and
shrink. Specifically,
\disp\gr{glue} $\longrightarrow$ \gr{optional signs}\gr{internal glue}
     $|$ \gr{dimen}\gr{stretch}\gr{shrink}\nl
     \gr{internal glue} $\longrightarrow$ \gr{glue parameter}
     $|$ \cs{lastskip}\nl 
     \indent $|$ \gr{skipdef token} $|$ \cs{skip}\gr{8-bit number}\nl
     \gr{glue parameter} $\longrightarrow$ \cs{abovedisplayshortskip}\nl
     \indent $|$ \cs{abovedisplayskip} $|$ \cs{baselineskip}\nl
     \indent $|$ \cs{belowdisplayshortskip} $|$ \cs{belowdisplayskip}\nl
     \indent $|$ \cs{leftskip} $|$ \cs{lineskip} $|$ \cs{parfillskip}
             $|$ \cs{parskip}\nl
     \indent $|$ \cs{rightskip} $|$ \cs{spaceskip}
             $|$ \cs{splittopskip} $|$ \cs{tabskip}\nl
     \indent $|$ \cs{topskip} $|$ \cs{xspaceskip}
     \>
The stretch and shrink components in a glue denotation
are optional, but when both are specified they have to
be given in sequence; they are defined as
\disp
\gr{stretch} $\longrightarrow$ \n{plus} \gr{dimen}
      $|$ \n{plus}\gr{fil dimen} $|$ \gr{optional spaces}\nl
\gr{shrink} $\longrightarrow$ \n{minus} \gr{dimen}
      $|$ \n{minus}\gr{fil dimen} $|$ \gr{optional spaces}\nl
\gr{fil dimen} $\longrightarrow$ \gr{optional signs}\gr{factor}%
     \gr{fil unit}\gr{optional spaces}\nl
\gr{fil unit} $\longrightarrow$ \n{ $|$ fil $|$ fill $|$ filll}
\>
The actual definition of \gr{fil unit} is recursive
(see Chapter~\ref[gramm]), but these are the only valid
possibilities.

\spoint Conversion of \gr{glue} to \gr{dimen}

The grammar rule
\disp\gr{dimen} $\longrightarrow$
     \gr{factor}\gr{unit of measure}
\>
has some noteworthy consequences, caused by the fact
that a \gr{unit of measure} need not look like a `unit of measure'
at all (see the list above).

For instance, from this definition we conclude that the statement
\Ver>\dimen0=\lastpenalty\lastpenalty<Rev is
syntactically correct because \cs{lastpenalty} can function
both as an integer and as \gr{unit of measure} by taking
its value in scaled points.
After \ver>\penalty8> the \cs{dimen0} thus defined will
have a size of~\n{64sp}.

More importantly, consider the case where the \gr{unit of measure} is
an \gr{internal glue}, that is, any sort of glue parameter.
Prefixing such a glue with a number (the \gr{factor})
makes it a valid \gr{dimen} specification.
Thus \Ver>\skip0=\skip1<Rev is very different
from \Ver>\skip0=1\skip1<Rev The first statement makes
\cs{skip0} equal to \cs{skip1}, the second converts
the \cs{skip1} to a \gr{dimen} before assigning it.
In other words, the \cs{skip0} defined by the second statement
has no stretch or shrink.


\spoint Registers for \cs{dimen} and \cs{skip}

\TeX\ has registers for storing \gr{dimen} and \gr{glue}
\csterm dimen\par\csterm skip\par
values: the \cs{dimen} and \cs{skip} registers
respectively. These are accessible by the expressions
\disp\cs{dimen}\gr{number}\> and
\disp\cs{skip}\gr{number}\>
As with all registers of \TeX, these registers are
numbered~0--255.

Synonyms for registers can be made with the \cs{dimendef} and
\csterm dimendef\par\csterm skipdef\par
\cs{skipdef} commands. Their syntax is
\Disp\cs{dimendef}\gr{control sequence}\gr{equals}\gr{8-bit number}
\Dispstop
and 
\Disp\cs{skipdef}\gr{control sequence}\gr{equals}\gr{8-bit number}\Dispstop
For example, after \ver-\skipdef\foo=13- using \cs{foo}
is equivalent to using \cs{skip13}.

Macros \cs{newdimen} and \cs{newskip} exist in plain \TeX
\csterm newdimen\par\csterm newskip\par
for allocating an unused dimen or skip register.
These macros are defined to be \cs{outer} in the plain format.

\spoint Arithmetic: addition

As for integer variables, arithmetic operations exist for
\csterm advance\par\term arithmetic on glue\par
dimen, glue, and muglue (mathematical glue; see page~\pgref[muglue])
variables.

The expressions
\Disp\cs{advance}\gr{dimen variable}\gr{optional \n{by}}%
     \gr{dimen}\nl
     \cs{advance}\gr{glue variable}\gr{optional \n{by}}%
     \gr{glue}\nl
     \cs{advance}\gr{muglue variable}\gr{optional \n{by}}%
     \gr{muglue}\Dispstop
add to the size of a dimen, glue, or muglue.

Advancing a \gr{glue variable} by \gr{glue} is done by
adding the natural sizes, and the stretch and shrink components.
Because \TeX\ converts between \gr{glue} and \gr{dimen},
it is possible to write for instance
\Ver>\advance\skip1 by \dimen1<Rev or
\Ver>\advance\dimen1 by \skip1<Rev
In the first case  \cs{dimen1} is coerced to \gr{glue} without
stretch or shrink; in the second case the \cs{skip1} is coerced
to a \gr{dimen} by taking its natural size.

\spoint Arithmetic: multiplication and division

Multiplication and division operations exist for glue
\csterm multiply\par\csterm divide\par
and dimensions. One may for instance write
\Ver>\multiply\skip1 by 2<Rev
which multiplies the natural size, and the stretch and shrink
components of \cs{skip1} by~2.

The second operand of a \cs{multiply} or \cs{divide}
operation can only be a \gr{number}, that is, an integer.
Introducing the notion of \gr{numeric variable}:
\disp\gr{numeric variable} $\longrightarrow$ \gr{integer variable}
     $|$ \gr{dimen variable} \nl
     \indent $|$ \gr{glue variable} $|$ \gr{muglue variable}\dispstop
these operations take the form
\Disp\cs{multiply}\gr{numeric variable}\gr{optional \n{by}}%
\gr{number}\Dispstop 
and
\Disp\cs{divide}\gr{numeric variable}\gr{optional \n{by}}%
\gr{number}\Dispstop

Glue and dimen can be multiplied by 
non-integer quantities:
\Ver>\skip1=2.5\skip2
\dimen1=.78\dimen2<Rev
However, in the first line the \cs{skip2} is first coerced
to a \gr{dimen} value by omitting its stretch and shrink.

\point More about dimensions

\spoint Units of measurement

In \TeX\ dimensions can be indicated in
\term units of measurement\par
\description \item centimetre
    denoted \n{cm} or 
\item millimetre
	denoted \n{mm}; these are SI~units ({\italic Syst\`eme International
	d'Unit\'es}, the
	international system of standard units of measurements).
\item inch
\n{in}; more common in the Anglo-American world.
One inch is 2.54~centimetres.
\item pica
    denoted \n{pc}; one pica is 12~points.
\item point
    denoted \n{pt}; the common system
for Anglo-American printers. One inch is 72.27 points.
\item didot point
    denoted \n{dd}; the common system for continental European printers.
    Furthermore, 1157 didot points are 1238~points.
\item cicero
    denoted \n{cc}; one cicero is 12~didot points.
\item big point
    denoted \n{bp}; one inch is 72 big points.
\item scaled point
    denoted \n{sp}; this is the smallest unit in \TeX, and all measurements
    are integral multiples of one scaled point.
    There are $65\,536$ scaled points in a~point.
\descriptionstop

Decimal fractions can be written using both the
Anglo-American system with the decimal point
(for example, \n{1in}=\n{72.27pt})
and the continental European system with a decimal
comma; \n{1in}=\n{72,27pt}.

Internally \TeX\ works with multiples of a smallest 
dimension: the  scaled point.
Dimensions larger (in absolute value) than $2^{30}-1$\n{sp},
which is about 5.75~metres or 18.9~feet, are illegal.

Both the pica system and the didot system are of French
origin: in 1737 the type founder Pierre Simon Fournier
introduced typographical points based on the French foot.
Although at first he introduced a system based on lines and
points, he later took the point as unit:
there are 72 points in an inch,
which is one-twelfth of a foot. 
About 1770 another founder, Fran\c{c}ois Ambroise Didot, introduced
points based on the more common, and slightly longer,
`pied du roi'.

\spoint Dimension testing

Dimensions and natural sizes of glue can be compared with
the \cs{ifdim} test. This takes the form
\disp\cs{ifdim}\gr{dimen$_1$}\gr{relation}\gr{dimen$_2$}\dispstop
where the relation can be an \n>, \n<, or~\n= token, 
all of category~12.

\spoint Defined dimensions

\invent
\item z@
 \n{0pt}

\item maxdimen 
      \n{16383.99999pt}; the largest legal dimension.

\>
These \gr{dimen}s are predefined in the plain format;
for instance \Ver>\newdimen\z@ \z@=0pt<Rev
Using such abbreviations for commonly used dimensions
has at least two advantages. First of all it saves main memory
if such a dimension occurs in a macro: a control sequence
is one token, whereas a string such as \n{0pt} takes three.
Secondly, it saves time in processing, as \TeX\ does not need
to perform conversions to arrive at the correct type of
object.

Control sequences such as \cs{z@}
are only available to a user who changes the
category code of the `at' sign. Ordinarily, these control sequences
appear only in the macros defined in packages such as the
plain format.

\point More about glue

Glue items can be added to a vertical list with one of the
\alt
commands \cs{vskip}\gr{glue}, \cs{vfil}, \cs{vfill}, \cs{vss} or
\csterm vskip\par
\cs{vfilneg}; 
glue items can be added to a horizontal list with one of the
commands \cs{hskip}\gr{glue}, \cs{hfil}, \cs{hfill}, \cs{hss} or
\csterm hskip\par
\cs{hfilneg}. We will now treat the properties of glue.

\spoint Stretch and shrink

In the syntax given above, \gr{glue} was defined as having
\term stretch\par\term shrink\par
\itemlist\item a `natural size', which is a \gr{dimen}, and optionally
\item a `stretch' and `shrink' component built out of a \gr{fil dimen}.
\itemliststop

Each list that \TeX\ builds has amounts of stretch and shrink
(possibly zero),
which are the sum of the
stretch and shrink components of individual pieces of glue in the list. 
Stretch and shrink are used if the context in which the list
appears requires it to assume a size that is different from
its natural size.

There is an important difference in behaviour between stretch
and shrink components when they are finite \ldash that is,
when the \gr{fildimen} is not \n{fil}(\n{l}(\n{l})). 
A~finite amount of shrink is indeed the maximum shrink
that \TeX\ will take: the amount of glue specified
as \Ver>5pt minus 3pt<Rev can shrink to \n{2pt}, but not further.
In contrast to this, a finite amount of stretch 
can be stretched arbitrarily far. 
Such arbitrary stretching
has a large `badness', however.
Badness calculation is treated below.

\examples
The sequence with natural size \n{20pt}
\Ver>\hskip 10pt plus 2pt \hskip 10pt plus 3pt<Rev
has \n{5pt} of stretch, but it has no shrink. In
\Ver>\hskip 10pt minus 2pt \hskip 10pt plus 3pt<Rev
there is \n{3pt} of stretch, and \n{2pt} of shrink,
so its minimal size is~\n{18pt}. 

Positive shrink is not the same as negative stretch:
\Ver>\hskip 10pt plus -2pt \hskip 10pt plus 3pt<Rev
looks a lot like the previous example, but it cannot
be shrunk as there are no \hbox{\n{minus}\gr{dimen}}
specifications. It does have \n{1pt} of stretch, however.

This is another example of negative amounts of shrink and stretch.
It is not possible to stretch
glue (in the informal sense) by shrinking it (in the technical
sense): \Ver>\hbox to 5cm{a\hskip 0cm minus -1fil}<Rev
is an underfull box, because \TeX\ looks for a \n{plus}~\gr{dimen}
specification when it needs to stretch the contents.

Finally, \Ver>\hskip 10pt plus -3pt \hskip 10pt plus 3pt<Rev
can neither stretch nor shrink.
The fact that there is only stretch
available means that the sequence cannot
shrink. However, the stretch components cancel out: the 
total stretch is zero. Another way of looking at this
is to consider that for each point that the second glue item would
stretch, the first one would `stretch back' one point.
\>

Any amount of infinite stretch or shrink overpowers all
finite stretch or shrink available:
\Ver>\hbox to 5cm{\hskip 0cm plus 16384pt 
              text\hskip 0cm plus 0.0001fil}<Rev
has the \n{text} at the extreme left of the box.
There are three orders of `infinity', each  one infinitely
stronger than the previous one:
\Ver>\hbox to 5cm{\hskip 0cm plus 16384fil
              text\hskip 0cm plus 0.0001fill}<Rev
and
\Ver>\hbox to 5cm{\hskip 0cm plus 16384fill
              text\hskip 0cm plus 0.0001filll}<Rev
both have the \n{text} at the left end of the box.



\spoint Glue setting

In the process of `glue setting', the desired width (or height)
\term glue setting\par
of a box is compared with the natural dimension of its contents,
which is the sum of all natural dimensions of boxes and globs of glue.
If the two differ, any available stretchability or shrinkability is used
to bridge the gap.
To attain the desired dimension of the box
only the glue of the highest available order is set:
each piece of glue of that order is stretched or shrunk by the
same ratio.

For example, in
\Ver>\hbox to 6pt{\hskip 0pt plus 3pt \hskip 0pt plus 9pt}<Rev
the natural size of the box is~\n{0pt}, and
the total stretch is~\n{12pt}. In order to obtain a box
of~\n{6pt} each glue item is set with a stretch ratio
of~$1/2$. Thus the result is equivalent to
\Ver>\hbox {\hskip 1.5pt \hskip 4.5pt}<Rev
Only the highest order of stretch or shrink is used:
in \Ver>\hbox to 6pt{\hskip 0pt plus 1fil \hskip 0pt plus 9pt}<Rev
the second glue  will assume its natural size of~\n{0pt},
and only the first   glue will be stretched.

\TeX\ will never exceed the maximum value of a finite
amount of shrink.
A~box that cannot be shrunk enough is called `overfull'.
Finite stretchability can be exceeded to provide an
escape in difficult situations; however, \TeX\ is likely 
to give an \ver-Underfull \hbox- message about this
(see page~\pgref[over/underfull]).
For an example of infinite shrink see page~\pgref[rlap].

\spoint Badness

When stretching or shrinking a list \TeX\ calculates 
\term badness calculation\par
badness based on the
ratio between actual stretch and the amount of stretch
present in the line. See Chapter~\ref[line:break]
for the application  of badness to the paragraph algorithm.

%\tracingmacros=2 \tracingcommands\tracingmacros
The formula for badness of a list that is stretched (shrunk) is
\label[bad:form]\message{Check roman min}
\disp $\displaystyle b=\hbox{min}\left(10\,000,
100\times \left({\hbox{actual amount stretched (shrunk)}
\over\hbox{possible amount of stretch (shrink)}}\right)^3\right)$\>
In reality \TeX\ uses a slightly different formula that is
easier to calculate, but behaves the same. Since glue setting is
one of the main activities of \TeX, this must be performed
as efficiently as possible.

This formula lets the badness be a reasonably small number
if the glue set ratio (the fraction in the above expression)
is reasonably small, but will let it grow rapidly once
the ratio is more than~1. Badness is infinite if the
glue would have to shrink more than the allotted amount;
stretching glue beyond its maximum is possible, so this
provides an  escape for very difficult lines of text or pages.

In \TeX3, the \cs{badness} parameter records the badness
of the most recently formed box.

\spoint Glue and breaking

\TeX\ can break lines and pages in several kinds of places.
One of these places is before a glue item. 
The glue is then discarded. For line breaks this is treated
in Chapter~\ref[line:break], 
for page breaks see Chapter~\ref[page:break].

There are two macros in plain \TeX, \cs{hglue} and \cs{vglue},
\csterm hglue\par\csterm vglue\par
that give non-disappearing glue in horizontal and
vertical mode respectively. For the horizontal case this is
accomplished by
placing:
\Ver>\vrule width 0pt \nobreak \hskip ...<Rev
Because \TeX\ breaks at the front end of glue,
this glue will always stay attached to the rule,
and will therefore never disappear.
The actual macro definitions are somewhat more complicated,
because they take care to preserve the \cs{spacefactor} and the
\cs{prevdepth}.

\spoint \cs{kern}

The \cs{kern} command specifies
\csterm kern\par
a~kern item in whatever mode \TeX\ is currently
in. A~kern item is much like a glue item without
stretch or shrink.
It differs from glue in that it is
in general not a legal breakpoint. Thus in
\Ver>.. text .. \hbox{a}\kern0pt\hbox{b}<Rev
\TeX\ will not break lines in between the boxes; in
\Ver>.. text .. \hbox{a}\hskip0pt\hbox{b}<Rev
a line can be broken in between the boxes.

However, if a kern is followed by glue, \TeX\ can break at the
kern (provided that it is not in math mode). 
In horizontal mode
both the kern and the glue then disappear in the break.
In vertical mode they are discarded when they are moved to
the (empty) current page after the material before
the break has been disposed of by the output routine 
(see Chapter~\ref[page:break]).

\spoint Glue and modes

All horizontal skip commands are \gr{horizontal command}s and
all vertical skip commands are \gr{vertical commands}s.
This means that, for instance, an \cs{hskip} command
makes \TeX\ start a paragraph if it is given in vertical mode.
The \cs{kern} command can be given in both modes.

\spoint The last  glue item in a list: backspacing

The last glue item in a list can be measured, and
it can be removed in all modes but external vertical mode.
The internal variables
\cs{lastskip} and  \cs{lastkern} can be used
\csterm lastskip\par\csterm lastkern\par
to measure the last glob of glue in all modes;
if the last glue was not a skip or kern respectively
they give~\n{0pt}.
In math mode the \cs{lastskip}
functions as \gr{internal muglue}, but in general
it classifies as \gr{internal glue}.
The \cs{lastskip} and \cs{lastkern}
are also \n{0pt} if that was the size of the last glue or
kern item on the list.

The operations\label[unskip]
\cs{unskip} and \cs{unkern} remove the last item of a list,
\csterm unskip\par\csterm unkern\par
if this is a glue or kern respectively. They have no effect
in external vertical mode; in that case the
best substitute is 
\ver=\vskip-\lastskip= 
and~\ver=\kern-\lastkern=.

In the process of paragraph building \TeX\ itself performs
an important \cs{unskip}: a~paragraph ending with a
white line will have a space token inserted by \TeX's input processor.
This is removed by an \cs{unskip} before the \cs{parfillskip} glue
(see Chapter~\ref[par:end]) is inserted.

Glue is treated by \TeX\ as a special case of leaders,
which becomes apparent when \cs{unskip} is applied to
leaders: they are removed.

\spoint Examples of backspacing

The plain \TeX\ macro \cs{removelastskip} is defined
\csterm removelastskip\par
as \Ver>\ifdim\lastskip=0pt \else \vskip-\lastskip \fi<Rev
If the last item on the list was a glue, this macro will
backspace by its value, provided its natural size was not zero.
In all other cases, nothing is added to the list.

Sometimes an intelligent version of commands such as \cs{vskip}
is necessary, in the sense that two subsequent skip commands
should result only in the larger of the two glue amounts.
On page~\pgref[skip:scheme] such a macro is used:
\Ver>\newskip\tempskipa
\def\vspace#1{\tempskipa=#1\relax
    \ifvmode \ifdim\tempskipa<\lastskip 
             \else \vskip-\lastskip \vskip\tempskipa
             \fi
    \else \vskip\tempskipa \fi}<Rev
First of all, this tests whether the mode is vertical; 
if not, the argument can safely be placed.
Copying the argument into a skip register is necessary
because \cs{v\-space}\ver>{2pt plus 3pt}> would lead to 
problems in an \ver>\ifdim#1<\lastskip> test.

If the surrounding mode was vertical, the argument
should only be placed if it is not less than what is
already there. The macro would be incorrect
if the test  read
\Ver>        \ifdim\tempskipa>\lastskip 
            \vskip-\lastskip \vskip\tempskipa
        \fi<Rev
In this case the sequence
\Ver>... last word.\par \vspace{0pt plus 1fil}<Rev
would not place any glue, because after
the \cs{par} we are in vertical mode and
\cs{lastskip} has a value of \n{0pt}.

\spoint Glue in trace output

If the workings of \TeX\ are traced by setting
\cs{tracingoutput} positive, or if \TeX\ 
writes a box to the log file 
(because of a \cs{showbox} command, or because it
is overfull or underfull),
glue  is denoted by the control sequence \cs{glue}.
This is not a \TeX\ command; it merely indicates the presence
of glue in the current list.

The box representation that \TeX\ generated from,
\alt
for instance, \cs{showbox}
inserts a space after every explicit \cs{kern},
but no space is inserted after an implicit
kern that was inserted by the kerning information in the font
\n{tfm} file. Thus \hbox{\ver-\kern 2.0pt-} denotes a kern
that was inserted by the user or by a macro, and
\ver-\kern2.0pt- denotes an implicit kern.

Glue that is inserted automatically (\cs{topskip}, \cs{baselineskip},
et cetera) is denoted by name in \TeX's trace output.
For example, the box
\Ver>\vbox{\hbox{Vo}\hbox{b}}<Rev
looks like
\Ver>
\vbox(18.83331+0.0)x11.66669
.\hbox(6.83331+0.0)x11.66669
..\tenrm V
..\kern-0.83334
..\tenrm o
.\glue(\baselineskip) 5.05556
.\hbox(6.94444+0.0)x5.55557
..\tenrm b<Rev
Note the implicit kern inserted between `V' and~`o'.
\endinput



