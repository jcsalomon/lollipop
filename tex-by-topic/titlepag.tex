%\tracingmacros=0 \tracingcommands\tracingmacros
%\Trace:yes
\emptypage
\HasNum:no

%%%%%%%%%%%%%%%% Half Title
\setbox0=\hbox{\def\krn{\kern8pt}\SansSerif\PointSize:24 \Style:roman
    \LetterSpace[4pt]{T{\lower.5ex\hbox{E}}X}\Spaces:2
    \LetterSpace[8pt]{BY}\Spaces:2 \LetterSpace[8pt]{TOPIC}}
\setbox2=\hbox{\PointSize:18 \Style:roman
    A \TeX nician's Reference}
\setbox0=\vbox{\baselineskip=24pt
    \moveright36pt\box0 
    \moveright36pt\box2
    \kern 36pt
    \hrule height 1pt width 30pc}
\setbox0=\hbox{\vrule width 1pt height 16pc \box0}
\setbox0=\vbox{\moveright 36pt \box0
    \nointerlineskip \kern 36pt 
    \hrule height 1pt width 30pc}
\setbox0=\hbox{\vrule width 1pt height 16pc
    \box0}
\box0

\EjectPage 

%%%%%%%%%%%%%%%% ii verso of half title empty

\hbox{}\vfil\hbox{}\eject

%%%%%%%%%%%%%%%% iii Title page

\setbox0=\hbox{\def\krn{\kern8pt}\SansSerif \PointSize:30 \Style:roman
    \LetterSpace[4pt]{T{\lower.5ex\hbox{E}}X}\Spaces:2
    \LetterSpace[8pt]{BY}\Spaces:2 \LetterSpace[8pt]{TOPIC}}
\setbox2=\hbox{\PointSize:24 \Style:roman
    A \TeX nician's Reference}
\setbox0=\vbox{\baselineskip=36pt
    \moveright36pt\box0 
    \moveright36pt\box2
    \kern 36pt
    \hrule height 1pt width 30pc}
\setbox0=\hbox{\vrule width 1pt height 16pc \box0}
\setbox0=\vbox{\moveright 36pt \box0
    \nointerlineskip \kern 36pt 
    \hrule height 1pt width 30pc}
\setbox0=\hbox{\vrule width 1pt height 16pc
    \box0}
\box0

\dimen4=74pt
{\baselineskip=6pc
\moveright\dimen4\hbox{\PointSize:18 \SetFont Victor Eijkhout}
\baselineskip=18pt
\moveright\dimen4\hbox{\PointSize:9 \Style:italic 
    University of Tennessee, Knoxville}
}
\vfill

{\advance\leftskip\dimen4
\PointSize:10 
ADDISON-WESLEY PUBLISHING COMPANY

\vskip12pt %\def\b{\unskip\nobreak\Spaces:0.8 $\bullet$\Spaces:0.8 }
\def\b{\unskip\nobreak\ $\bullet$\ }
\PointSize:9 \Style:roman 
\baselineskip=14pt
\NoHyphenation %\flushright:no \rightmarginstretch=0pt plus 2em
\noindent Wokingham, England \b Reading, Massachusetts \nl
Menlo Park, California \b New York \b Don Mills, Ontario \nl
Amsterdam \b Bonn \b Sydney \b Singapore \b Tokyo \b Madrid \nl
San Juan \b Milan \b Paris \b Mexico City \b Seoul \b Taipei
\par
}

\EjectPage 

%%%%%%%%%%%%%%%% iv  title verso
\begingroup

\parindent=0pt \FlushRight:no

\noindent \copyright\Spaces:2 1992 Addison-Wesley Publishers Ltd.\par
\noindent \copyright\Spaces:2 1992 Addison-Wesley Publishing Company Inc.\par

\bigskip

All rights reserved. No part of this publication may be reproduced,
stored in a retrieval system, or transmitted
in any form or by any means, electronic, mechanical, photocopying, 
recording or otherwise,
without prior written permission of the publisher.

\bigskip

The programs in this book have been included for their
instructional value. They have been tested with care but are not
guaranteed for any particular purpose. The publisher does
not offer any warranties or representations, 
nor does it accept any liabilities with respect to the programs.

\bigskip

Many of the designations used by manufacturers and sellers
to distinguish their products are claimed as trademarks.
Addison-Wesley has made every attempt to supply trademark information
about manufacturers and their products mentioned in this book.
A~list of the trademark designations and their owners
appears on page~viii.



\bigskip

Cover designed by Chris Eley and\nl
printed by The Riverside Printing Co. (Reading) Ltd.\nl
Typeset by \CorporateLogo\ using Baskerville and Gill Sans typefaces;\nl
typographic design by Merry Obrecht.\nl
Printed in Great Britain by ...

\bigskip

First printed 1991.

\bigskip

{\Style:bold British Library Cataloguing in Publication Data}\nl
A catalogue record for this book is available from the British
Library.

\bigskip

{\Style:bold Library of Congress Cataloging in Publication Data}
available.

\endgroup

\EjectPage

%%%%%%%%%%%%%%%% v  preface, spill over to vi
\def\subjectTitle{PREFACE}
\asubject Preface\par

%\hbox{}\vskip0pt minus.5\baselineskip\hbox{}
% watchit!
%\tmc
\ToExternalFile:contents={Preface}

To the casual observer, \TeX\
is not a state-of-the-art typesetting system.
No flashy multilevel menus and interactive manipulation
of text and graphics dazzle the onlooker.
    On a less superficial level, however, \TeX\ is a very sophisticated
program, first of all because of the ingeniousness of its
built-in algorithms for such things as paragraph breaking
and make-up of mathematical formulas, and
second because of its almost complete programmability.
The combination of these factors makes it possible for \TeX\
to realize almost every imaginable layout in a highly automated
fashion.

Unfortunately, it also means that \TeX\ has an
unusually large number of commands and parameters,
and that programming \TeX\ can be far from easy.
Anyone wanting to program in \TeX, and maybe
even the ordinary user, would seem to need two books:
a~tutorial that gives a first glimpse of the many
nuts and bolts of \TeX, and after that
a~systematic, complete reference manual.
This book tries to fulfil the latter function.
A~\TeX er who has already made a start
(using any of a number of introductory books
on the market)
should be able to use this book indefinitely thereafter.

In this volume the universe of \TeX\ is presented as
about forty different subjects, each in a separate
chapter.\message{Number of chapters?}
Each chapter starts out with a list of control sequences
relevant to the topic of that chapter
and proceeds to treat the 
theory of the topic. 
Most chapters conclude with remarks and examples.

Globally, the chapters are ordered as follows. 
The chapters on basic mechanisms are first,
the chapters on text treatment and mathematics are next,
and finally there are some
chapters on output and aspects of \TeX's connections to
the outside world.

The book also contains a glossary of \TeX\
commands, tables,
and indexes by example, by control sequence, and by subject.
The subject index refers for most concepts to
only one page, where most of the information
on that topic can be found, as well as references
to the locations of related information.

This book does not treat any specific \TeX\ macro package.
Any parts of the plain format that are treated are those
parts that belong to the `core' of plain \TeX: they
are also present in, for instance, \LaTeX.
Therefore, most remarks about the plain format
are true for \LaTeX, as well as most other formats.
Putting it differently,
if the text refers to the plain format, this should be taken
as a contrast to pure \IniTeX, not to \LaTeX.
By way of illustration, occasionally macros from plain \TeX\
are explained that do not belong to the core.

\vskip1\baselineskip
\Indent:no
{\SansSerif\Style:roman Acknowledgment}\nl
I am indebted to Barbara Beeton, Karl Berry, and Nico Poppelier,
who read previous versions of this book. Their comments
helped to improve the presentation.
Also I~would like to thank the participants of
the discussion lists \TeX hax, \TeX-nl, and {\tt comp.text.tex}.
Their questions and answers gave me much food for thought.
Finally, any acknowledgement in a book about \TeX\ ought to
include Donald Knuth for inventing \TeX\ in the
first place. This book is no exception.
\hfil\break
\hbox{}\hfill \Style:italic Victor Eijkhout\break
\hbox{}\hfill Urbana, Illinois, August 1991\par

\vfil\eject

%%%%%%%%%%%%%%%% vii  table of contents
\parskip=0pt 
\PointSize:10 \Style:roman
\HasNum:no
\asubject Contents

\LoadExternalFile:contents

\eject

%%%%%%%%%%%%%%%% trademarks
\emptypage
\hbox{}\vfill
{\bold Trademark notice}\nl
IBM\TM\ is a trademark of International Business Machines Corporation.\nl
Metafont\TM\ is a trademark of Addison-Wesley Publishing Company.\nl
PostScript\TM\ is a trademark of Adobe Systems Incorporated.\nl
\TeX\TM\ and \AmsTeX\TM\ are 
    trademarks of the American Mathematical Society.\nl
UNIX\TM\ is a trademark of AT\&T.

%\eject

\endinput
