\subject[char] Characters

Internally, \TeX\ represents characters by their (integer) 
character code. This chapter treats those codes, and the
commands that have access to them.

\invent
\item char
      Explicit denotation of a character to be typeset. 

\item chardef 
      Define a control sequence to be a synonym for
      a~character code.

\item accent 
      Command to place accent characters.

\item if
      Test equality of character codes. 

\item ifx
      Test equality of both character and category codes.

\item let
      Define a control sequence to be a synonym of a token.

\item uccode 
      Query or set
      the character code that is the uppercase variant of a given code.

\item lccode
      Query or set
      the character code that is the lowercase variant of a given code.

\item uppercase
      Convert the \gr{general text} argument to its uppercase form.

\item lowercase 
      Convert the \gr{general text} argument to its lowercase form.

\item string
      Convert a token to a string of one or more characters.
\item escapechar
      Number of the character that is to be used 
      for the escape character
      when control sequences are being converted
      into character tokens. \IniTeX\ default:~92~(\cs{}).

\inventstop

\point[char:code] Character codes

Conceptually it is easiest to think that \TeX\ works with
\term character codes\par
characters internally, but in fact
\TeX\ works with integers: the `character codes'. 

The way characters are encoded in a computer may differ
from system to system.
Therefore \TeX\ uses its own scheme of character codes.
Any character that is read from a file (or from the user terminal)
is converted to a character code according to the
character code table.
A~category code is then assigned based on this (see Chapter~\ref[mouth]).
The character code table is based on the 7-bit \ascii{} table
for numbers under~128 (see Chapter~\ref[table]).

There is an explicit conversion between characters
(better:  character tokens)
and  character codes  using the left quote (grave, back quote)
character~\n{`{}}:
at all places where \TeX\ expects a \gram{number} you
can use the left quote followed by a character
token or
a single-character control sequence.
Thus both \ver.\count`a. and \ver.\count`\a. are synonyms
\awp
for \ver.\count97.. See also Chapter~\ref[number].

The possibility of a single-character control
sequence is necessary in certain cases such as
\disp\ver>\catcode`\%=11>\quad or\quad \ver>\def\CommentSign{\char`\%}>\>
which would be misunderstood if the backslash were left out.
For instance \Ver>\catcode`%=11<Rev would consider
the \n{=11} to be a comment.
Single-character
control sequences can be formed from characters with any
category code.

After the conversion to character codes any connection
with external representations has disappeared. Of course,
for most characters  the visible output will `equal' the input
(that is, an `\n{a}' causes an~`a').
There are exceptions, however, even among the common symbols.
In the Computer Modern
roman fonts there are no `less than' and `greater than'
\message{Check <>! Dammit!}%
signs, so the input `\ver.<>.' will give `{\MathRMx<>}' in the output.

In order to make \TeX\ machine independent at the output
side, the character codes are also used in the \n{dvi} file:
opcodes $n=0\ldots127$ denote simply the instruction `take
character $n$ from the current font'. The complete definition
of the opcodes in a \n{dvi} file can be found in~\cite[K:program].


\point Control sequences for characters

There are a number of ways in which a control sequence can denote
a character. The \cs{char} command specifies a character to be
typeset; the \cs{let} command introduces
a synonym for a character token, that is,
the combination of character code and category code.

\point Denoting characters to be typeset: \cs\char

Characters can be denoted numerically by, for example,
\ver.\char98.\csterm char\par.
This command tells \TeX\ to add character number~98 of the
current font to the horizontal list currently under construction.

Instead of decimal notation, it is often more convenient to
use octal or hexadecimal notation. For octal the single quote is used:
\ver.\char'142.; hexadecimal uses the double quote: \ver.\char"62..
Note that \ver.\char''62. is incorrect; the process that replaces
two quotes by a double quote works at a later stage of processing
(the visual processor) than number scanning (the execution processor).

Because of the explicit conversion to character codes by the
back quote character it is also possible to get a `b' \ldash provided
that you are using a font organized a bit like the \ascii{} table \rdash
with \ver.\char`b.  or \ver.\char`\b..

The \cs{char} command looks superficially a bit like
the \ver-^^- substitution mechanism (Chapter~\ref[mouth]).
Both mechanisms access characters without directly denoting them.
However, the \ver-^^- mechanism operates in a very early stage of
processing (in the input processor of \TeX,
but before category code
assignment); the \cs{char} command, on the other hand,
comes in the final stages of processing. 
In effect it says `typeset character number
so-and-so'.
\awp

There is a construction to let a control sequence stand
for some character code: the \cstoidx chardef\par\ command.
The syntax of this is \label[chardef]
\disp\cs{chardef}\gram{control sequence}\gr{equals}\gram{number}, 
\dispstop
where the number can be an explicit
representation or a counter value, but it can also be
a character code
obtained using the left quote command (see above; 
the full definition of \gr{number} is given in Chapter~\ref[number]). 
In the plain format 
the latter possibility is used in
definitions such as \Ver>\chardef\%=`\%<Rev
which could have been given equivalently as
\Ver>\chardef\%=37<Rev
After this command, the control symbol \ver>\%>
used on its own is a synonym for \ver>\char37>,
that is, the command to typeset character~37
(usually the per cent character).

A control sequence that has been defined with a \cs{chardef}
command can also be used as a \gr{number}.
This fact is used in  allocation commands such as 
\cs{newbox} (see Chapters~\ref[number] and~\ref[alloc]).
Tokens defined with \cs{mathchardef} can also be used this
way.

\spoint Implicit character tokens: \cs{\let}

Another construction defining a control sequence
\term implicit characters\par
to stand for (among other things)
a character is~\cs{let}\csterm let\par:
\disp\cs{let}\gr{control sequence}\gr{equals}\gr{token}\dispstop
with a character token on the right hand side of the (optional)
equals sign. The result is called an implicit character token.
(See page~\pgref[let] for a further discussion of~\cs{let}.)

In the
plain format there are for instance synonyms for
the open and close brace:
\Ver>\let\bgroup={ \let\egroup=}<Rev
The resulting control sequences are called `implicit braces'
(see Chapter~\ref[group]).

Assigning characters by \cs{let}
is different from defining control sequences by \cs{chardef}, 
in the sense that \cs{let}
makes the control sequence stand for the combination
of a character code and category code. 

As an example
\Ver>\catcode`|=2 % make the bar an end of group
\let\b=|  % make \b a bar character
{\def\m{...}\b \m<Rev
gives an `undefined control sequence \cs{m}'
because the \cs{b} closed the group inside which \cs{m}
was defined. On the other hand,
\Ver>\let\b=| % make \b a bar character
\catcode`|=2  % make the bar character end of group
{\def\m{...}\b \m<Rev
leaves one group open, and it prints a vertical bar
(or whatever is in position 124 of the current font).
The first of these examples
implies that even when the braces have been redefined
(for instance into active characters for macros that
format C code) the beginning-of-group and end-of-group
functionality is available through the control sequences
\cs{bgroup} and~\cs{egroup}.

Here is
another example to show
that implicit character tokens are hard to distinguish
from real character tokens. After the above sequence
\Ver>\catcode`|=2 \let\b=|<Rev
the tests \Ver>\if\b|<Rev and \Ver>\ifcat\b}<Rev
are both true.

Yet another example can be found in the plain format:
the commands
\Ver>\let\sp=^ \let\sb=_ <Rev allow people without an
underscore or circumflex on their keyboard to 
make sub- and superscripts in mathematics.
For instance:
\disp\ver>x\sp2\sb{ij}>\quad gives\quad $x\sp2\sb{ij}$\>
If a person typing in the format itself does not have
these keys, some further tricks are needed:\label[spsb:truc]
\Ver>{\lccode`,=94 \lccode`.=95 \catcode`,=7 \catcode`.=8
\lowercase{\global\let\sp=, \global\let\sb=.}}<Rev
will do the job; see below for an explanation of lowercase codes.
The \ver>^^> method as it was in \TeX\ version~2
(see page~\pgref[hathat]) cannot be used here,
as it would require typing two characters that can ordinarily
not be input.
With the extension in \TeX\ version~3 it would also be possible
to write \Ver>{\catcode`\,=7
\global\let\sp=,,5e \global\let\sb=,,5f}<Rev
denoting the codes 94 and 95 hexadecimally.

Finding out just what a control sequence has been defined to be with
\cs{let} can be done using \cs{meaning}:
the sequence \Ver>\let\x=3 \meaning\x<Rev gives
`\n{the character 3}'.\awp

\point Accents

Accents can be placed by the
\gr{horizontal command}~\cstoidx accent\par\term accents\par
\label[character]:
\disp\cs{accent}\gr{8-bit number}\gr{optional assignments}%
     \gr{character}\dispstop
where \gr{character} is a character of category 11 or~12,
 a~\cs{char}\gr{8-bit number} command,
or a~\cs{chardef} token. If none of these
four types of \gr{character} follows, the accent is taken to be a
\cs{char} command itself; this gives an accent `suspended
in mid-air'. Otherwise the accent is placed
on top of the following character.
Font changes between the accent and the character can be effected
by the \gr{optional assignments}.

An unpleasant implication of the fact that an \cs{accent} command
has to be followed by a \gr{character} is that it is not
possible to place an accent on a ligature, or
two accents on top of each other.
In some languages, such as Hindi or Vietnamese,
such double accents do occur.
Positioning accents on top of each other is possible,
however, in math mode.

The width of a character with an accent is the same as that of
the unaccented character. \TeX\ assumes that the 
accent as it appears in the font file
is properly positioned for a character that is as high
as the x-height of the font; for characters with other heights
it correspondingly lowers or raises the accent.

No genuine under-accents exist in \TeX. They are
implemented as low placed over-accents. A~way of handling
them more correctly would be to write a macro that
measures the following character, and raises or drops
the accent accordingly.
The cedilla macro, \cs{c}\csterm c\par,
in plain \TeX\ does something along these lines. However,
it does not drop the accent for characters with descenders.

The horizontal positioning of an accent is controlled by
\cs{fontdimen1}, slant per point. Kerns are used
for the horizontal movement. Note that, although they
are inserted automatically, these kerns are classified
as {\italic explicit\/} kerns. Therefore they inhibit hyphenation
in the parts of the word before and after the kern.

As an example of kerning for accents, 
here follows the dump of a horizontal list.
\message{maybe italic correction for extra line}
\Ver>\setbox0=\hbox{\it \`l}
\showbox0<Rev
gives\Ver>
\hbox(9.58334+0.0)x2.55554
.\kern -0.61803 (for accent)
.\hbox(6.94444+0.0)x5.11108, shifted -2.6389
..\tenit ^^R
.\kern -4.49306 (for accent)
.\tenit l<Rev
Note that the accent is placed first, so afterwards the italic
correction of the last character is still available.
\awp

\point Testing characters

Equality of character codes is tested by \cs{if}:
\disp\cs{if}\gr{token$_1$}\gr{token$_2$}\dispstop
Tokens following this conditional are expanded until two
unexpandable tokens are left. The condition is then true
if those tokens are character tokens with the same character
code, regardless of category code. 

An unexpandable control
sequence is considered to have character code 256 and
category code~16 (so that it is unequal to anything except
another control sequence), except in the case
where it had been \cs{let} to a non-active character token.
In that case it is considered to have the character code
and category code of that character. This was mentioned above.

The test \cs{ifcat} for category codes was mentioned
in Chapter~\ref[mouth]; the test
\disp\cs{ifx}\gr{token$_1$}\gr{token$_2$}\dispstop
can be used to test for category code and character code
simultaneously.
The tokens following this test are not expanded.
However, if they are macros, \TeX\
tests their expansions for equality.

Quantities defined by \cs{chardef} can be tested with
\cs{ifnum}:
\Ver>\chardef\a=`x \chardef\b=`y \ifnum\a=\b % is false <Rev
based on the fact (see Chapter~\ref[number]) that
\gr{chardef token}s can be used as numbers.

\point Uppercase and lowercase

\spoint[uc/lc] Uppercase and lowercase codes

To each of the character codes correspond
\term uppercase\par\term lowercase\par
\csterm lccode\par\csterm uccode\par
an uppercase code and a lowercase code (for still more codes see below).
These can be assigned
by 
\Disp\cs{uccode}\gram{number}\gr{equals}\gram{number}\Dispstop
and 
\Disp\cs{lccode}\gram{number}\gr{equals}\gram{number}.\Dispstop
In \IniTeX\ codes \ver-`a..`z-, \ver-`A..`Z- have uppercase code
\label[ini:uclc]
\ver-`A..`Z- and lowercase code \ver-`a..`z-.
All other character codes have both uppercase and lowercase
code zero.

\spoint[upcase] Uppercase and lowercase commands

The commands \ver-\uppercase{...}- and \ver-\lowercase{...}-
\csterm uppercase\par\csterm lowercase\par
go through their argument lists, replacing all character 
codes of explicit character tokens
by their uppercase and lowercase code respectively
if these are non-zero,
without changing the category codes. 
\awp

The argument of \cs{uppercase} and \cs{lowercase}
is a \gr{general text}, which is defined as
\Disp \gr{general text} $\longrightarrow$ \gr{filler}\lb
      \gr{balanced text}\gr{right brace}\Dispstop
(for the definition of \gr{filler} see Chapter~\ref[gramm])
meaning that the left brace can be implicit, but the closing
right brace must be an explicit character token with category
code~2. \TeX\ performs expansion to find the opening
brace.

Uppercasing and lowercasing are executed in the execution processor;
they are not `macro expansion' activities
like \cs{number} or \cs{string}.
The sequence (attempting to produce~\cs{A})
\Ver>\expandafter\csname\uppercase{a}\endcsname<Rev
gives an error (\TeX\ inserts an \cs{endcsname} before   the
\cs{uppercase} because \cs{uppercase} is unexpandable), but
\Ver>\uppercase{\csname a\endcsname}<Rev works.

As an example of the correct use of \cs{uppercase}, here
is a macro that tests if a character is uppercase:
\Ver>\def\ifIsUppercase#1{\uppercase{\if#1}#1}<Rev
The same test can be
performed by \ver>\ifnum`#1=\uccode`#1>.

Hyphenation of words starting with an uppercase character,
that is, a character not equal to its own \cs{lccode},
is subject to the \cs{uchyph} parameter: if this
is positive, hyphenation of capitalized words is allowed.
See also Chapter~\ref[line:break].

\spoint Uppercase and lowercase forms of keywords

Each character in \TeX\ keywords, such as \n{pt}, can be
given in uppercase or lowercase form. 
For instance, \n{pT}, \n{Pt}, \n{pt}, and~\n{PT} all have
the same meaning. \TeX\ does not use
the \cs{uccode} and \cs{lccode} tables here to
determine the lowercase form. Instead it
converts uppercase characters to lowercase by adding~32
\ldash the \ascii{} difference between uppercase and lowercase
characters \rdash to their character code. This has some implications
for implementations of \TeX\ for non-roman alphabets;
see page 370 of \TeXbook, \cite[K:book].

\spoint Creative use of \cs{\uppercase} and \cs{\lowercase}

The fact that \cs{uppercase} and \cs{lowercase} do not change
category codes can sometimes be used to create certain
character-code--category-code combinations that would
otherwise be difficult to produce. See for instance the
explanation of the \cs{newif} macro in Chapter~\ref[if],
and another example on page~\pgref[spsb:truc].

For a slightly different application, consider the
problem (solved by Rainer Sch"opf) of,
given a counter \ver-\newcount\mycount-, writing character
number \ver-\mycount- to the terminal.
Here is a solution:
%\Ver>\lccode`a=\mycount \chardef\terminal=16
%\lowercase{\write\terminal{a}}<Rev
\Ver>\lccode`a=\mycount \chardef\terminal=16<Rev
\awp
\Ver>\lowercase{\write\terminal{a}}<Rev
The \cs{lowercase} command effectively changes the 
argument of the \cs{write} command from~`\n a'
into whatever it should be.

\point[codename] Codes of a character

Each character code has a number of \gr{codename}s associated
\term codenames\par
with it. These are integers in various ranges that determine
how the character is treated in various contexts, or
how the occurrence of that character changes the workings
of \TeX\ in certain contexts.

The code names are as follows:
\description\item \cs{catcode}
\gr{4-bit number} (0--15); the category to which a character belongs.
This is treated in Chapter~\ref[mouth].
\item \cs{mathcode}
\gr{15-bit number} (0--\ver-"7FFF-) or \ver-"8000-;
determines how a character is treated
in math mode. See Chapter~\ref[mathchar].
\item \cs{delcode}
\gr{27-bit number} (0--\n{\hex7$\,$FFF$\,$FFF});
determines how a character is treated after
\cs{left} or \cs{right} in math mode.
See page~\pgref[delcodes].
\item \cs{sfcode}
integer; determines how spacing is affected after this character.
See Chapter~\ref[space].
\item \cs{lccode}, \cs{uccode}
\gr{8-bit number} (0-255); lowercase and
uppercase codes \rdash these were treated above.
\descriptionstop

\point Converting tokens into character strings

The command \cs{string} takes the next token and expands it
\csterm string\par
into a string of separate characters. Thus
\Ver>\tt\string\control<Rev will give \cs{control} in the
output, and
\Ver>\tt\string$<Rev will give~\ver-$-, but, noting that the string 
operation comes after the tokenizing,
\Ver>\tt\string%<Rev will {\em not\/} give~\ver$%$,
because the comment
sign is removed by \TeX's input processor.
Therefore, this command will `string' the first token on the next line.

The \cs{string} command is executed by the expansion processor, thus
it is expanded unless explicitly inhibited (see Chapter~\ref[expand]).

\spoint Output of control sequences

In the above examples the typewriter font was selected, because
\csterm escapechar\par
the Computer Modern roman font does not have a backslash character.
\awp
However,
\TeX\ need not have used the backslash character to display
a control sequence: it uses character number \cs{escapechar}.
This same value is also used when a control sequence is
output with \cs{write}, \cs{message}, or \cs{errmessage},
and it is used in the output of \cs{show}, \cs{showthe} and \cs{meaning}.
If \cs{escapechar} is negative or more than~255,
the escape character is not
output; the default value (set in \IniTeX) is~92, the number
of the backslash character.

For use in a  \cs{write} statement the \cs{string} can 
in some circumstances be
replaced  by \cs{noexpand} (see page~\pgref[expand:write]).

\spoint Category codes of a \cs{\string}

The characters that are the result of a \cs{string} command have 
category code~12, except for any spaces in 
a stringed control sequence;
they have category code~10. Since inside a control
sequence there are no category codes, 
any spaces resulting from \cs{string} are
of necessity only space {\em characters}, that is,
characters with code~32.
However, \TeX's input processor converts
all space tokens that have a character code other than~32
into character tokens with character code~32, 
so the chances are pretty slim that
`funny spaces' wind up in control sequences.

Other commands with the same behaviour with respect to 
category codes as \cs{string}, are
\cs{number},
\cs{romannumeral}, \cs{jobname}, \cs{fontname}, \cs{meaning},
and \cs{the}.




\endinput
