\subject[if] Conditionals

Conditionals are an indispensible tool for powerful macros.
\term conditionals\par
\TeX\ has a large repertoire of conditionals for querying
such things as category codes or processing modes.
This chapter gives an inventory of the various conditionals,
and it treats the evaluation of
conditionals in detail.

\invent
\item if 
      Test equality of character codes. 

\item ifcat 
      Test equality of category codes. 

\item ifx 
      Test equality of macro expansion, or equality of character code and
      category code.

\item ifcase 
      Enumerated case statement.

\item ifnum 
      Test relations between numbers.

\item ifodd 
      Test whether a number is odd.

\item ifhmode 
      Test whether the current mode is (possibly restricted) horizontal mode.

\item ifvmode 
      Test whether the current mode is (possibly internal) vertical mode.

\item ifmmode 
     Test whether the current mode is (possibly display) math mode.

\item ifinner 
      Test whether the current mode is an internal mode.

\item ifdim 
      Compare two dimensions. 

\item ifvoid 
      Test whether a box register is empty.

\item ifhbox 
      Test whether a box register contains a horizontal box.

\item ifvbox 
      Test whether a box register contains a vertical box. 

\item ifeof 
      Test for end of input stream or non-existence of file.

\item iftrue 
      A test that is always true.
\item iffalse 
      A test that is always false.
\item fi 
      Closing delimiter for all conditionals.

\item else 
      Select \gr{false text} of a conditional 
      or default case of \cs{ifcase}.

\item or 
      Separator for entries of an \cs{ifcase}.

\item newif 
     Create a new test.

\inventstop

\point The shape of conditionals

Conditionals in \TeX\ have one of the following two forms
\csterm else\par\csterm fi\par
\disp\cs{if...}\gr{test tokens}\gr{true text}\cs{fi}\nl
     \cs{if...}\gr{test tokens}\gr{true text}\cs{else}%
     \gr{false text}\cs{fi}\dispstop
where the \gr{test tokens} are zero or more tokens, depending on
the particular conditional; the \gr{true text} is a series of tokens
to be processed if the test turns out true, and the \gr{false text}
is a series of tokens to be processed if the test turns out false.
Both the \gr{true text} and the \gr{false text} can be empty.

The exact process of how \TeX\ expands conditionals is treated
below.

\point Character and control sequence tests

Three tests exist for testing character tokens and
control sequence tokens.

\spoint \cs{if}

Equality of character codes can be tested by
\csterm if\par
\Disp\cs{if}\gr{token$_1$}\gr{token$_2$}\Dispstop
In order to allow the tokens to be control sequences,
\TeX\ assigns character code~256 to control sequences,
the lowest positive number that is not the character code of a
character token (remember that the legal character codes
are~0--255).

Thus all control sequences are equal as far as \cs{if} is
concerned, and they are unequal to all character tokens.
As an example, this fact can be used to define
\howto Test whether a token is a control sequence\par
\Ver>\def\ifIsControlSequence#1{\if\noexpand#1\relax}<Rev
which tests whether a token is a control sequence token
instead of a character token (its result is unpredictable
if the argument is a \ver>{...}> group).

After \cs{if} \TeX\ will expand until two unexpandable
tokens are obtained, so it is necessary to prefix
expandable control sequences and active characters
with \cs{noexpand} when testing them with~\cs{if}.

\examples After
\Ver>\catcode`\b=13 \catcode`\c=13 \def b{a} \def c{a} \let\d=a<Rev
we find that
\tdisp\ver-\if bc- is true, because both \n b and \n c expand to \n a,\nl
\ver-\if\noexpand b\noexpand c- is false, and\nl
\ver-\if b\d- is true because \n{b} expands to the character~\n{a},
   and \cs{d} is an implicit character token~\n{a}.
\>
\>
\spoint \cs{ifcat}

The \cs{if} test ignores category codes; these can be tested
\csterm ifcat\par
by \Disp\cs{ifcat}\gr{token$_1$}\gr{token$_2$}\Dispstop

This test is a lot like \cs{if}: \TeX\ expands after it
until unexpandable tokens remain. For this test
control sequences
are considered to have category code~16
(ordinarily, category codes are in the range~0--15), which makes them
all equal to each other, and different from all character
tokens.

\spoint \cs{ifx}

Equality of tokens is tested in a stronger sense than
\csterm ifx\par
the above by \Disp\cs{ifx}\gr{token$_1$}\gr{token$_2$}\Dispstop

\itemlist\item Character tokens are equal for \cs{ifx} if
they have the same character code and category code.
    \item Control sequence tokens are equal if they represent the
same \TeX\ primitive, or have been similarly defined by
\cs{font}, \cs{countdef}, or some such. For example,
\Ver>\let\boxhor=\hbox \ifx\boxhor\hbox %is true
\font\a=cmr10 \font\b=cmr10 \ifx\a\b %is true<Rev
\item Control sequences are also equal if they are
macros with the same parameter text and replacement text,
and the same status with respect to \cs{outer} and~\cs{long}.
For example,
\Ver>\def\a{z} \def\b{z} \def\c1{z} \def\d{\a}
\ifx\a\b %is true
\ifx\a\c %is false
\ifx\a\d %is false<Rev\itemliststop

Tokens following this test are not expanded.

By way of example of the use of \cs{ifx} consider string testing.
A simple implementation of string testing in \TeX\ is as follows:
\Ver>
\def\ifEqString#1#2{\def\testa{#1}\def\testb{#2}%
    \ifx\testa\testb}<Rev
The two strings are used as the replacement text of two macros,
and equality of these macros is tested.
This is about as efficient as string testing can get:
\TeX\ will traverse the definition texts of the
macros \cs{testa} and \cs{testb}, which has precisely the
right effect.

As another example, one can test whether a control sequence is defined
by\howto Test whether a control sequence is (un)defined\par
\Ver>
\def\ifUnDefinedCs#1{\expandafter
    \ifx\csname#1\endcsname\relax}
\ifUnDefinedCs{parindent} %is not true
\ifUnDefinedCs{undefined} %is (one hopes) true<Rev
This uses the fact that a \ver>\csname...\endcsname> command
is equivalent to \cs{relax} if the control sequence
has not been defined before. Unfortunately, this test also
turns out true if a control sequence has been \cs{let} to
\cs{relax}.

\point Mode tests

In order to determine in which of the six modes 
(see Chapter~\ref[hvmode]) \TeX\ 
\csterm ifhmode\par\csterm ifvmode\par\csterm ifmmode\par
\csterm ifinner\par
is currently operating, the tests \cs{ifhmode}, 
\cs{ifvmode}, \cs{ifmmode}, and~\cs{ifinner}
are available.

\itemlist\item\cs{ifhmode} is true if \TeX\ is in horizontal mode
or restricted horizontal mode.
\item\cs{ifvmode} is true if \TeX\ is in vertical mode or
internal vertical mode.
\item\cs{ifmmode} is true if \TeX\ is in math mode or display
math mode.\itemliststop

The \cs{ifinner} test is true if \TeX\ is in any of the three
internal modes: restricted horizontal mode, internal vertical
mode, and non-display math mode.

\point Numerical tests

Numerical relations between \gr{number}s can be tested
\csterm ifnum\par
with \disp\cs{ifnum}\gr{number$_1$}\gr{relation}%
\gr{number$_2$}\dispstop
where the relation is a character \n{<}, \n{=}, or~\n{>},
of category~12.

Quantities such as glue can be used as a number here
through the conversion to scaled points, and \TeX\
will expand in order to arrive at the two \gr{number}s.

Testing for odd or even numbers can be done with \cs{ifodd}:
\csterm ifodd\par
the test\disp\cs{ifodd}\gr{number}\dispstop
is true if the \gr{number} is odd.

\point Other tests

\spoint Dimension testing

Relations between \gr{dimen} values (Chapter~\ref[glue])
\csterm ifdim\par
can be tested with
\cs{ifdim} using the same three relations as in \cs{ifnum}.

\spoint Box tests

Contents of box registers (Chapter~\ref[boxes]) can be tested with
\csterm ifhbox\par\csterm ifvbox\par\csterm ifvoid\par
\disp\cs{ifvoid}\gr{8-bit number}\>
which is true if the register contains no box,
\disp\cs{ifhbox}\gr{8-bit number}\>
which is true if the register contains a horizontal box, and
\disp\cs{ifvbox}\gr{8-bit number}\>
which is true if the register contains a vertical box.

\spoint I{/}O tests

The status of input streams (Chapter~\ref[io]) can be tested with
\csterm ifeof\par
the end-of-file test
\cs{ifeof}\gr{number}, which is only false
if the number is in the range 0--15, and the corresponding
stream is open and not fully read. In particular, this test
is true if the file name connected
to this stream (through \cs{openin})
does not correspond to an existing file.
See the example on page~\pgref[ex:eof].

\spoint Case statement

The \TeX\ case statement is called \cs{ifcase};
\csterm ifcase\par\csterm or\par
its syntax is\disp\cs{ifcase}\gr{number}\gr{case$_0$}\cs{or}%
\n{...}\cs{or}\gr{case$_n$}\cs{else}\gr{other cases}\cs{fi}
\dispstop where for $n$ cases there are $n-1$ \cs{or}
control sequences. Each of the \gr{case$_i$}
parts can be empty,
and the \cs{else}\gr{other cases} part is optional.

\spoint Special tests

The tests \cs{iftrue} and \cs{iffalse} are always
\csterm iftrue\par\csterm iffalse\par
true and false respectively.
They are mainly useful as tools in macros.

For instance, the sequences \Ver>\iftrue{\else}\fi<Rev
and \Ver>\iffalse{\else}\fi<Rev yield a left and right
brace respectively, but they have balanced braces, so they
can be used inside a macro replacement text.

The \cs{newif} macro, treated below,
provides another use of \cs{iftrue} and \cs{iffalse}.
On page 260 of \TeXbook\ these control sequences
are also used in an interesting manner.

\point[newif:def] The \cs{\newif} macro

The plain format defines an (outer) macro \cs{newif} by
\csterm newif\par
which the user can define new conditionals.
If the user defines \Ver>\newif\iffoo<Rev
\TeX\ defines three new control sequences,
\cs{footrue} and \cs{foofalse} with which the user can set
the condition, and \cs{iffoo} which tests the `foo' condition.

The macro call \ver-\newif\iffoo- expands to
\Ver>\def\footrue{\let\iffoo=\iftrue} \def\foofalse{\let\iffoo=\iffalse}
\foofalse<Rev
The actual definition, especially the part that ensures that
the \cs{iffoo} indeed starts with \cs{if}, is a pretty hack.
An explanation follows here.
This uses concepts from Chapters~\ref[macro]
and~\ref[expand].

The macro \cs{newif} starts as follows:
\Ver>\outer\def\newif#1{\count@\escapechar \escapechar\m@ne<Rev
This saves the current escape character in \cs{count@}, and
sets the value of \cs{escapechar} to~\n{-1}.
The latter action has the
effect that no escape character is used in the output
of \cs{string}\gr{control sequence}.

An auxiliary macro \ver>\if@> is defined by
\Ver>{\uccode`1=`i \uccode`2=`f \uppercase{\gdef\if@12{}}}<Rev
Since the uppercase command changes only character codes, and
not category codes, the macro \cs{if@} now has
to be followed by the characters \n{if} of category~12.
Ordinarily, these characters have category code~11.
In effect this
macro then eats these two characters, and \TeX\ complains if
they are not present.

Next there is a macro \ver>\@if> defined by
\Ver>\def\@if#1#2{\csname\expandafter\if@\string#1#2\endcsname}<Rev
which will be called like \ver>\@if\iffoo{true}> and
\ver>\@if\iffoo{false}>. 

Let us examine the call \ver>\@if\iffoo{true}>.
\itemlist\item The \cs{expandafter} reaches over the \ver>\if@>
to expand \cs{string} first. The part \ver>\string\iffoo>
expands to \n{iffoo} because the escape character is not printed,
and all characters have category~12.
\item The \ver>\if@> eats the first two characters
\n i$_{12}$\n f$_{12}$ of this.
\item As a result, the final expansion of \ver>\@if\iffoo{true}>
is then \Ver>\csname footrue\endcsname<Rev\itemliststop

Now we can treat the relevant parts of \cs{newif} itself:
\Ver>
\expandafter\expandafter\expandafter
   \edef\@if#1{true}{\let\noexpand#1=\noexpand\iftrue}%<Rev

The three \cs{expandafter} commands may look intimidating, so let us
take one step at a time.
\itemlist\item One \cs{expandafter} is necessary to reach over the \cs{edef},
such that \ver>\@if> will expand:
\Ver>\expandafter\edef\@if\iffoo{true}<Rev gives
\Ver>\edef\csname footrue\endcsname<Rev
\item Then another \cs{expandafter} is necessary to activate
\altt
the \cs{csname}:
\Ver>
\expandafter \expandafter \expandafter \edef \@if ...
%   new          old          new<Rev
\item This makes the final expansion 
\Ver>\edef\footrue{\let\noexpand\iffoo=\noexpand\iftrue}<Rev
\itemliststop

After this follows a similar statement for the \n{false} case:
\Ver>   \expandafter\expandafter\expandafter
   \edef\@if#1{false}{\let\noexpand#1=\noexpand\iffalse}%<Rev
The conditional starts out false, and the escape character
has to be reset:
\Ver>  \@if#1{false}\escapechar\count@} <Rev


\point Evaluation of conditionals

\TeX's conditionals behave differently from those
\term conditionals, evaluation\par
in ordinary programming languages. In many instances
one may not notice the difference, but in certain contexts
it is important to know precisely what happens.

When \TeX\ evaluates a conditional, it first determines
what is to be tested. This in itself may involve some
expansion; as we saw in the previous chapter, 
only after an \cs{ifx} test
does \TeX\  not expand. After all other tests \TeX\ will
expand tokens until the extent of the test and the tokens
to be tested have been determined. On the basis of the outcome
of this test the \gr{true text} and the \gr{false text}
are either expanded or skipped.

For the processing of the parts of the conditional
let us consider some cases separately.
\itemlist
\item \ver>\if... ... \fi> and the result of the test is false.
 After the test \TeX\ will start skipping material
 without expansion, without counting braces, but balancing
 nested conditionals, until a \cs{fi} token is encountered.
 If the \cs{fi} is not found an error message results
 at the end of the file:
 \disp\tt Incomplete \cs{if...}; all text was ignored after line \n{...}
 \dispstop where the line number indicated is that of the line
 where \TeX\ started skipping, that is, where the conditional
 occurred.

\item \ver>\if... \else ... \fi> and the result of the test is false.
 Any material in between the condition and the \cs{else} is skipped
 without expansion, without counting braces, but balancing nested
 conditionals.

 The \cs{fi} token can be the result of expansion; if it never
 turns up \TeX\ will give a diagnostic message
 \disp\tt \cs{end} occurred when \cs{if...} on line \n{...}
      was incomplete\dispstop
 This sort of error is not visible in the output.

 This point plus the previous may jointly be described as follows:
 after a false condition \TeX\ skips until an \cs{else} or \cs{fi}
 is found; any material in between \cs{else} and \cs{fi} is processed.
 
\item \ver>\if... ... \fi> and the result of the test is true.
 \TeX\ will start processing the material following the condition.
 As above, the \cs{fi} token may be inserted by expansion of
 a macro.
 
\item \ver>\if... \else ... \fi> and the result of the test is true.
 Any material following the condition is processed until the \cs{else}
 is found; then \TeX\ skips everything until the matching \cs{fi}
 is found.
 
 This point plus the previous may be described as follows:
 after a true test \TeX\ starts processing material until
 an \cs{else} or \cs{fi} is found; if an \cs{else} is found
 \TeX\ skips until it finds the matching \cs{fi}.
\itemliststop


\point Assorted remarks

\spoint The test gobbles up tokens

A common mistake is to write the following:
\Ver>\ifnum\x>0\someaction \else\anotheraction \fi<Rev
which has the effect that the \ver.\someaction. is expanded,
regardless of whether the test succeeds or not. 
The reason for this is that \TeX\ evaluates the input stream until
it is certain that it has found the arguments to be tested.
In this case it is perfectly possible for the \ver.\someaction.
to yield a digit, so it is expanded. The remedy is to insert
\altt
a space or a \cs{relax} control sequence
after the last digit of the number to be tested.

\spoint The test wants to gobble up the \cs{else} or \cs{fi}

The same mechanism that underlies the phenomenon in the previous
point can lead to even more surprising effects if \TeX\
bumps into an \ver.\else., \ver.\or., or \ver.\fi. 
while still busy determining the extent of the test itself.

Recall that \ver.\pageno. is a synomym for \ver.\count0., and
consider the following examples:
\Ver>\newcount\nct \nct=1\ifodd\pageno\else 2\fi 1<Rev
and
\Ver>\newcount\nct \nct=1\ifodd\count0\else 2\fi 1<Rev
The first example will assign either 11 or~121 to \cs{nct}, 
but the second one will assign 1 or~121. 
The explanation is that
in cases like the second, where
\altt
an \ver.\else. is encountered while the
test still has not been delimited, a \ver.\relax. is inserted.
In the case that \ver.\count0. is odd the result will thus be \ver.\relax.,
and the example will yield \Ver>\nct=1\relax2<Rev
which will assign~1 to \cs{nct}, and print~2.


\spoint[after:cond] Macros and conditionals; the use of \cs{expandafter}

Consider the following example:
\Ver>\def\bold#1{{\bf #1}} \def\slant#1{{\sl #1}}
\ifnum1>0 \bold \else \slant \fi {some text} ...<Rev
This will make not only `some text', 
but {\sl all\/} subsequent text bold.
Also, at the end of the job there will be a notice that
`end occurred inside a group at level~1'.
Switching on \cs{tracingmacros} reveals that the argument
of \ver.\bold. was \ver.\else..
This means that, after expansion of \ver.\bold., 
the input stream looked like
\Ver>\ifnum1>0 {\bf \else }\fi {some text} rest of the text <Rev
so the closing brace was skipped as part of the \gram{false text}.
Effectively, then, the resulting stream is
\Ver>{\bf {some text} rest of the text<Rev
which is unbalanced.

One solution to this sort of problem would be to write
\Ver>\ifnum1>0 \let\next=\bold \else \let\next=\slant \fi \next<Rev
but  a solution using \cs{expandafter} is also possible:
\Ver>\ifnum1>0 \expandafter \bold \else \expandafter \slant \fi<Rev
This works, because the \cs{expandafter} commands let \TeX\ determine
the boundaries of the \gram{true text} and the \gram{false text}.

In fact, the second solution may be preferred over the first,
since conditionals are handled by the expansion processor,
and the \cs{let} statements are tackled only by the execution
processor; that is, they are not expandable.
Thus the second solution will (and the first will not)
work, for instance,
inside an~\cs{edef}.

Another example with \cs{expandafter} is the sequence
\Ver>\def\get#1\get{ ... }
\expandafter \get \ifodd1 \ifodd3 5\fi \fi \get<Rev
This gives\Ver>#1<- \ifodd3 5\fi \fi<Rev and
\Ver>\expandafter \get \ifodd2 \ifodd3 5\fi\fi \get<Rev
gives\Ver>#1<-<Rev
This illustrates again that the result of evaluating a 
conditional is not the final expansion, but the start
of the expansion of the \gr{true text} or \gr{false text},
depending on the outcome of the test.

A detail should be noted: with \cs{expandafter}
it is possible that the \ver.\else. is encountered
before the \gram{true text} has been expanded completely.
This raises the question as to the exact timing of expansion
and skipping.
In the example
\Ver>\def\hello{\message{Hello!}}
\ifnum1>0 \expandafter \hello \else \message{goodbye} \bye<Rev
the error message caused by the missing \ver.\fi. is given
without \ver.\hello. ever having been expanded.
The conclusion must be that the \gram{false text} is
skipped as soon as it has been located, even if this is at a time
when the \gram{true text} has not been expanded completely.

\spoint Incorrect matching

\TeX's matching of \ver.\if., \ver.\else., and \ver.\fi.
is easily upset. For instance, \TeXbook\ warns you that
you should not say \Ver> \let\ifabc=\iftrue<Rev inside a 
conditional, because if this text is skipped \TeX\ sees
at least one \ver.\if. to be matched. 

The reason for this is that when \TeX\ is skipping
it recognizes all \cs{if...}, \cs{or}, \cs{else}, and \cs{fi}
tokens, and everything that has been declared a synonym of
such a token by \cs{let}. In \ver>\let\ifabc=\iftrue>
\TeX\ will therefore at least see the \cs{iftrue} as
the opening of a conditional, and, if the current meaning
of \cs{ifabc} was for instance \cs{iffalse}, it will also
be considered as the opening of a conditional statement.

As another example, if
\Ver> \csname if\sometest\endcsname \someaction \fi<Rev
is skipped as part of conditional text,
the \ver.\fi. will unintentionally close the
outer conditional.

It does not help to enclose such potentially dangerous
constructs inside a group, because grouping is independent of
conditional structure. Burying such commands inside macros is
the safest approach.

Sometimes another solution is possible, however.
The \cs{loop} macro of plain \TeX\ (see page~\pgref[loop:ex])
is used as \Ver>\loop ... \if ... \repeat<Rev
where the \cs{repeat} is not an actually executable
command, but is merely a delimiter:
\Ver>\def\loop#1\repeat{ ... }<Rev
Therefore,
by declaring \Ver>\let\repeat\fi<Rev
the \cs{repeat} balances the \cs{if...} that terminates
the loop, and it becomes possible to have loops in
skipped conditional text.

\spoint Conditionals and grouping

It has already been mentioned above that group nesting in \TeX\
is independent of conditional nesting.
The reason for this is that conditionals are handled by the
expansion part of \TeX; in that stage braces are just
unexpandable tokens that require no special treatment.
Grouping is only performed in the later stage of execution
processing.

An example of this independence is now given.
One may write a macro that yields part of
a conditional:
\Ver>\def\elsepart{\else \dosomething \fi}<Rev
The other way around, the following macros
yield a left brace and a right brace respectively:
\Ver>\def\leftbrace{\iftrue{\else}\fi}
\def\rightbrace{\iffalse{\else}\fi}<Rev
Note that braces in these definitions are properly nested.

\spoint A trick

In some contexts it may be hard to get rid of
\cs{else} or \cs{fi} tokens in a proper
manner. The above approach with \cs{expandafter}
works only if there is a limited number of tokens involved.
In other cases the following trick may provide a way out:
\Ver>\def\hop#1\fi{\fi #1}<Rev Using this as
\disp\ver>\if... \hop >\gr{lots of tokens}\ver>\fi>\dispstop
will place the tokens outside the conditional.
This is for instance used in~\cite[E2].

As a further example of this sort of trick,
consider the problem (suggested to me and solved by
Alan Jeffrey) of implementing a conditional 
\ver-\ifLessThan#1#2#3#4-
such that the arguments corresponding to \ver-#3- or
\ver-#4- result, depending on whether \ver-#1- is
less than \ver-#2- or not. 

The problem here is how to get rid of the \cs{else} and the~\cs{fi}.
The \ldash or at least, one \rdash solution is to scoop them up
as delimiters for macros:
\Ver>\def\ifLessThan#1#2{\ifnum#1<#2\relax\taketrue \else \takefalse \fi}
\def\takefalse\fi#1#2{\fi#2}
\def\taketrue\else\takefalse\fi#1#2{\fi#1}<Rev
Note that \cs{ifLessThan} has only two parameters
(the things to be tested); however, its
result is a macro that
chooses between the next two arguments.

\spoint More examples of expansion in conditionals

Above, the macro \cs{ifEqString} was given
\alt
that compares two strings:
\howto Compare two strings\par
\Ver>\def\ifEqString#1#2%
   {\def\csa{#1}\def\csb{#2}\ifx\csa\csb }<Rev
However, this macro relies on \cs{def},  which is not an
expandable command. If we need a string tester that will
work, for instance, inside an  \cs{edef}, we need some
more ingenuity (this solution was taken from~\cite[E2]).
The basic principle of this solution is to compare the strings
one character at a time. Macro delimiting by \cs{fi} is used;
this was explained above.

First of all, the \cs{ifEqString} call is replaced by a
sequence \ver>\ifAllChars ...\Are ...\TheSame>, and both
strings are delimited by a dollar sign, which is not supposed
to appear in the strings themselves.
\Ver>
\def\ifEqString
    #1#2{\ifAllChars#1$\Are#2$\TheSame}<Rev
The test for equality of characters first determines
whether either string has ended. If both have ended, the original
strings were equal; if only one has ended, they were of unequal
length, hence unequal. If neither string has ended, we test
whether the first characters are equal, and if so, we make a recursive
call to test the remainder of the string.
\Ver>
\def\ifAllChars#1#2\Are#3#4\TheSame
   {\if#1$\if#3$\say{true}%
          \else \say{false}\fi
    \else \if#1#3\ifRest#2\TheSame#4\else
                 \say{false}\fi\fi}
\def\ifRest#1\TheSame#2\else#3\fi\fi
   {\fi\fi \ifAllChars#1\Are#2\TheSame}<Rev
The \cs{say} macro is supposed to  give \cs{iftrue} for
\ver>\say{true}> and \cs{iffalse} for \ver>\say{false}>.
Observing that all  calls to this macro occur two conditionals deep,
we use the `hop' trick explained above as follows.
\Ver>
\def\say#1#2\fi\fi
   {\fi\fi\csname if#1\endcsname}<Rev

Similar to the above example, let us write a macro
that will test lexicographic (`dictionary') precedence
of two strings:
\howto Compare two strings lexicographically\par
\Ver>
\let\ex=\expandafter
\def\ifbefore
    #1#2{\ifallchars#1$\are#2$\before}
\def\ifallchars#1#2\are#3#4\before
   {\if#1$\say{true\ex}\else
     \if#3$\say{false\ex\ex\ex}\else
      \ifnum`#1>`#3 \say{false%
         \ex\ex\ex\ex\ex\ex\ex}\else
       \ifnum`#1<`#3 \say{true%
         \ex\ex\ex\ex\ex\ex\ex
         \ex\ex\ex\ex\ex\ex\ex\ex}\else
       \ifrest#2\before#4\fi\fi\fi\fi}
\def\ifrest#1\before#2\fi\fi\fi\fi
   {\fi\fi\fi\fi
    \ifallchars#1\are#2\before}
\def\say#1{\csname if#1\endcsname}<Rev
In this macro a slightly
different implementation of \cs{say} is used.

Simplified, a call to \cs{ifbefore} will eventually lead to a situation
that looks (in the `true' case) like
\Ver>
\ifbefore{...}{...}
       \if... %% some comparison that turns out true
          \csname iftrue\expandafter\endcsname
       \else .... \fi
   ... %% commands for the `before' case
\else
   ... %% commands for the `not-before' case
\fi<Rev
When the comparison has turned out true, \TeX\ will start processing
the \gr{true text}, and make a mental note to remove any
\ver>\else ... \fi> part once an \cs{else} token is seen.
Thus, the sequence
\Ver>\csname iftrue\expandafter\endcsname \else ... \fi<Rev
is replaced by \Ver> \csname iftrue\endcsname<Rev
as the \cs{else} is seen while \TeX\ is still processing
\ver>\csname...\endcsname>.

Calls to \cs{say} occur inside nested conditionals, so
the number of \cs{expandafter} commands necessary may be 
\alt
larger than~1: for level two it is~3, for level three
it is~7, and for level~4 it is 15. Slightly more compact
implementations of this macro do exist.

\endinput
