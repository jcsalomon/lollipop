\subject[io]  File Input and Output

This chapter treats the various ways in which \TeX\ can read from
\term I/""O\par
and write to external files.

\invent
\item input 
      Read a specified file as \TeX\ input. 

\item endinput 
      Terminate inputting the current file after the current line.

\item pausing 
      Specify that \TeX\ should pause after each line that is 
      read from a file.

\item inputlineno 
      Number of the current input line.


\item write   
      Write a \gr{general text} to the terminal or to a file. 

\item read   
      Read a line from a stream into a control sequence.

\item newread \cs{newwrite}
      Macro for allocating a new input/""output stream.

\item openin \cs{closein}
      Open/close an input stream.

\item openout \cs{closeout}
      Open/close an output stream.

\item ifeof  
      Test whether a file has been fully read, or does not exist.

\item immediate  
      Prefix to have output operations executed right away.

\item escapechar 
      Number of the character that is  used 
      when control sequences are being converted
      into character tokens.
      \IniTeX\ default:~92.

\item newlinechar 
      Number of the character that triggers a new line in
      \cs{write} statements.

\inventstop


\point Including files: \cs{\input} and \cs{\endinput}

Large documents can be segmented in \TeX\ by putting 
\term input files\par\csterm input\par
parts in separate files, and loading these with \cs{input}
into the master file. The exact syntax for
file names is implementation dependent; most of the
time a \n{.tex} file extension is assumed if no explicit
extension is given.
File names can be delimited with a space or with \cs{relax}.
The \cs{input} command is expandable.

If \TeX\ encounters in an input file the
\csterm endinput\par
\cs{endinput} statement, it acts as if the file
ends after the line on which the statement occurs.
Any statements on the same line as \cs{endinput} are
still executed.
The \cs{endinput} statement is expandable.

\point File I{/}O

\TeX\ supports input and output streams for reading and writing
\altt
files one line at a time.

\spoint Opening and closing streams

\TeX\ supports up to 16 simultaneous input and 16 output streams. 
\term streams\par
The plain \TeX\ macros
\csterm newread\par\csterm newwrite\par
\cs{newread} and \cs{newwrite} give the number of an unused
stream. This number is assigned by a \cs{chardef} command. 
Input streams are completely independent of output
streams. 

Input streams are opened by
\csterm openin\par
\disp\cs{openin}\gr{4-bit number}\gr{equals}\gr{filename}\dispstop
and closed by 
\csterm closein\par
\disp\cs{closein}\gr{4-bit number}\dispstop

Output streams are opened by
\csterm openout\par
\disp\cs{openout}\gr{4-bit number}\gr{equals}\gr{filename}\dispstop
and closed by
\csterm closeout\par
\disp\cs{closeout}\gr{4-bit number}\dispstop

If an output file does not yet exist, it is created
by \cs{openout}; if it did exist, an \cs{openout} will
cause it to be overwritten.

The output operations \cs{openout}, \cs{closeout},
and \cs{write} can all three be prefixed by \cs{immediate};
see below.

\spoint Input with \cs{read}

In addition to the \cs{input} command, which reads a whole
\csterm read\par
file, \TeX\ has the \cs{read} operation, which
reads one line from a file (or from the user terminal).
The syntax of the read command is
  \disp\cs{read}\gr{number}\n{to}\gr{control sequence}\dispstop
The effect of this statement is that one input line
is read from the designated stream, and the control
sequence is defined as a macro without parameters, having
that line as replacement text.

If the input line is not balanced with respect to braces,
\TeX\ will read more than one line, continuing for as long
as is necessary to get a balanced token list.
\TeX\ implicitly appends an empty line to each input stream,
\alt
so the last \cs{read} operation on a stream will always
yield a single \cs{par} token.

Read operations from any stream outside the range 0--15 \ldash or 
streams not associated with an open file, or on which the file
end has been reached \rdash 
read from the terminal. If the stream number is positive
the user is prompted with the name of the control sequence
being defined by the \cs{read} statement.

\example\Ver>\read16 to \data<Rev
displays a prompt \Ver>\data=<Rev and typing `my name'
in response makes the read statement equivalent
to \Ver>\def\data{my name }<Rev
The space at the end of the input derives from the line end;
to prevent this one could write
\Ver>{\endlinechar=-1 \global\read16 to \data}<Rev
\>

\spoint Output with \cs{write}

\TeX's \cs{write} command
\csterm write\par
  \disp\cs{write}\gr{number}\gr{general text}\dispstop
writes a balanced token list to a file which has been opened
by \cs{openout}, to the log file, or to the terminal.

Write operations to a stream outside 0--15 \ldash or to a
stream that is not associated with an open file \rdash  go to the log file;
if the stream number is positive they 
go to the terminal as well as to the log file.

The token list argument of \cs{write}, defined as
  \disp\gr{general text} $\longrightarrow$ \gr{filler}%
       \lb\gr{balanced text}\gr{right brace}\dispstop
can have an implicit opening brace.
This argument is expanded as if it were the replacement
text of an \cs{edef}, so, for instance, 
any macros and conditionals appearing are expanded.
No commands are executed, however. 
This expansion occurs
at the time of shipping out; see below.
Until that time the argument token list is stored
in a whatsit item on the current list.
See further Chapter~\ref[expand] for
a discussion of expansion during writing.

A control sequence output by \cs{write} (or \cs{message})
is represented with  a trailing space, and using
character number \cs{escapechar}
for the escape character.
The \IniTeX\ default for this is~92,
the code for the backslash.
The trailing space can be prevented by prefixing the control
sequence with \cs{string}.

\point Whatsits

There is an essential difference 
\term whatsits\par
in execution between input and output:
operations concerning output 
(\cs{openout}, \cs{closeout}, \cs{write})
are not executed immediately; instead, they are saved until
the box in which they appear is shipped out 
to the \n{dvi} file.

Writes and the other two output operations are placed
in `whatsit' items on whichever list is currently being built.
The actual operation occurs when the part of the page
that has the item is shipped out to the \n{dvi} file.
This delayed output is made necessary by \TeX's
asynchronous output routine behaviour.
See a worked-out example on page~\pgref[expand:write].

An \ver.\immediate\write. \ldash or any other \cs{immediate} output
\csterm immediate\par
operation \rdash  is executed on the spot, and
does not place a whatsit item on the current list.

The argument of a \cs{special} command
(see page~\pgref[special]) is also placed in a whatsit.

Whatsit items in leader boxes are ignored.

\point Assorted remarks

\spoint Inspecting input

\TeX\ records the current line number in the current input file
\csterm inputlineno\par
in the \gr{internal integer} parameter \cs{inputlineno}
(in \TeX3).

If the parameter \cs{pausing} is positive, \TeX\ shows
\csterm pausing\par
every line that is input on the terminal screen,
and gives the user the opportunity
to insert commands. These can for instance be \cs{show} commands.
Inserted commands are treated as if they were directly
in the source file: it is for instance not necessary
to prefix them with~`i', as would be necessary when 
\TeX\ pauses for an error.

\spoint Testing for existence of files

\TeX\ is not the friendliest of systems when you
\howto Test whether a file exists\par
ask it to input a non-existing file. Therefore the following
sequence of commands can be used to prevent trouble\label[ex:eof]:
\Ver>\newread\instream \openin\instream= fname.tex
\ifeof\instream \message{File 'fname' does not exist!}
\else \closein\instream \input fname.tex
\fi<Rev
Here an input stream is opened with the given file name.
The end-of-file test is also true
if an input stream does not correspond to a physical file,
so if this conditional is not true,
the file exists and an \cs{input} command can safely be given.

\spoint Timing problems

The synchronization between write operations on the
one hand, and opening/closing operations
of files on the other hand,
can be a crucial point. Auxiliary files, such as are
used by various formats to implement cross-references,
are a good illustration of this.

Suppose that during a run of \TeX\ the auxiliary file is written, and
\howto Input a file that was created in the same run of \TeX\par
at the end of the run it has to be input again for a variety
of purposes (such as seeing whether references have changed).
An \cs{input} command is executed right away, so
the file must have been closed with an \ver=\immediate\closeout=.
However, now it becomes possible that the file is closed 
before all writes to it have been performed.
The following sequence remedies this:
\Ver>\par\vfil\penalty -10000 \immediate\closeout\auxfile<Rev
The first three commands activate the output routine
in order to close off the last page,
so all writes will indeed have been performed before the
file is closed.

\spoint \cs{message} versus \cs{immediate}\cs{write}16

Messages to the user can be given using 
\csterm message\par
\cs{message}\gr{general text}, which writes to the  terminal.
Messages are appended to one another; 
the line is wrapped when the line
length (a~\TeX\ compile-time constant) has been reached.
In \TeX\ version2,
a~maximum of 1000 characters is written per message;
this is not a compile-time constant, but is hard-wired 
into the \TeX\ program.

Each message given with \ver=\immediate\write=
starts on a new line; the user can force a new line
in the message by including the character with
\csterm newlinechar\par
number~\cs{newlinechar}. This parameter also works
in \cs{message}.

\spoint Write inside a vertical box

Since a write operation winds up on the vertical list in a whatsit,
issuing one at the start of a \cs{vtop} 
will probably influence the height of that box
(see  Chapter~\ref[boxes]). As an example,
\Ver>
have the \vtop{\write\terminal{Hello!}\hbox{more text}}
dangling from<Rev
will have the~\vtop{\write-1{vtop gezien}\hbox{more text}}~dangling 
from the baseline (and when this book is \TeX ed the
message `Hello!' appears on the screen).

\spoint Expansion and spaces in \cs{write} and \cs{message}

Both \cs{write} and \cs{message} expand their argument
as if it were the replacement text of an \cs{edef}.
Therefore \Ver>\def\a{b}\message{\a}<Rev will
write out~`\n b'.

Unexpandable control sequences are displayed with a trailing
space (and prefixed with the \cs{escapechar}):
\Ver>\message{\hbox\vbox!}<Rev will write out
`\ver>\hbox \vbox !>'. Undefined control sequences give an error here.

Expandable control sequences can be written out with some
care:\Ver>\message{\noexpand\ifx}
\message{\string\ifx}
{\let\ifx\relax \message{\ifx}}<Rev
all write out `\ver>\ifx>'.

Note, however, that spaces after expandable control sequences
are removed in the input processor, which goes into state~$S$
after a control sequence. Therefore
\Ver>\def\a{b}\def\c{d}
\message{\a \c}<Rev writes out `\n{bd}'.
Inserting a space can be done as follows:
\Ver>\def\space{ } % in plain TeX
\message{\a\space\c}<Rev displays `\n{b d}'.
Note that\Ver> 
\message{\a{ }\c}<Rev does not work: it displays `\ver=b{ }d='
since braces are unexpandable character tokens.

\endinput
