\subject[expand]  Expansion

Expansion in \TeX\ is rather different from procedure calls
\term expansion\par
in most programming languages. This chapter treats the
commands connected with expansion, and gives a number of
(non-trivial) examples.

\invent
\item relax 
     Do nothing.


\item expandafter  
      Take the next two tokens and place the expansion of the
      second after the first.

\item noexpand   
      Do not expand the next token.


\item edef 
      Start a macro definition; 
      the replacement text is expanded at definition time.

 
\item aftergroup  
      Save the next token for insertion after the current group.

\item afterassignment   
      Save the next token for execution after the next assignment.


\item the 
      Expand the value of various quantities in \TeX\ into a string
      of character tokens.

\inventstop


\point Introduction

\TeX's expansion processor accepts a stream of tokens
coming out of the input processor, and its result is
again a stream of tokens, which it feeds to the execution
processor. For the input processor there are two
kinds of tokens: expandable and unexpandable ones.
The latter category is passed untouched, and it contains
largely assignments and typesettable material;
the former category
is expanded, and the result of that expansion is examined anew.

\point Ordinary expansion

The following list gives those constructs
that are expanded, unless
expansion is inhibited:
\itemlist
\item macros\label[expand:lijst]\term expandable control sequences\par
\item conditionals
\item \cs{number}, \cs{romannumeral}
\item \cs{string}, \cs{fontname}, \cs{jobname}, 
      \cs{meaning}, \cs{the}
\item \ver,\csname ... \endcsname,
\item \cs{expandafter}, \cs{noexpand}
\item \cs{topmark}, \cs{botmark}, \cs{firstmark}, 
      \cs{splitfirstmark}, \cs{splitbotmark}
\item \cs{input}, \cs{endinput}
\itemliststop

This is the list of all instances where
expansion is inhibited:
\itemlist\label[noexp:list]
\item when \TeX\ is reading a token to be defined by
      \itemlist \item a \gr{let assignment}, that is,
           by \cs{let} or \cs{futurelet},
        \item a \gr{shorthand definition}, that is,
           by \cs{chardef} or \cs{mathchardef}, or a
           \gr{register def}, that is, \cs{countdef},
           \cs{dimendef}, \cs{skipdef}, \cs{muskipdef},
           or~\cs{toksdef},
        \item a \gr{definition}, that is a macro definition
           with \cs{def}, \cs{gdef}, \cs{edef}, or~\cs{xdef},
        \item the \gr{simple assignment}s \cs{read} and \cs{font};
      \itemliststop
\item when a \gr{parameter text} or macro arguments
      are being read; also when  the replacement text of a 
      control sequence
      being defined by \cs{def}, \cs{gdef}, or \cs{read}
      is being read;
\item when the token list for a \gr{token variable} or
      \cs{uppercase}, \cs{lowercase}, or \cs{write}
      is being read; however, the token list for \cs{write}
      will be  expanded later when it is shipped out;
\item when tokens are being deleted during error recovery;
\item when part of a conditional is being skipped;
\item in two instances when \TeX\ has to know what follows
      \itemlist\item after a left quote in a context where
         that is used to denote an integer (thus in 
         \ver-\catcode`\a- the \cs{a} is not expanded), or
        \item after a math shift character that begins math mode
         to see whether another math shift character follows (in which case
         a display opens);
        \itemliststop
\item when an alignment preamble is being scanned; however,
      in this case a~token
      preceded by \cs{span} and the tokens in a \cs{tabskip} 
      assignment are still expanded.
\itemliststop

\point Reversing expansion order

Every once in a while you need to change the normal order of
expansion of tokens. \TeX\ provides several mechanisms for
this. Some of the control sequences in this section are
not strictly concerned with expansion.

\spoint One step expansion: \cs{expandafter}

The most obvious tool for reversed expansion order is
\csterm expandafter\par
\cs{expandafter}. The sequence
\disp\cs{expandafter}\gr{token$_1$}\gr{token$_2$}\dispstop
expands to \disp\gr{token$_1$}\gr{\italic the expansion of token$_2$}
\dispstop
Note the following.
\itemlist \item If \gr{token$_2$} is a macro, it is replaced
by its replacement text, not by its final expansion.
Thus, if 
\Ver>\def\tokentwo{\ifsomecondition this \else that \fi}
\def\tokenone#1{ ... }<Rev
the call \Ver>\expandafter\tokenone\tokentwo<Rev
will give \cs{ifsomecondition} as the parameter
to \cs{tokenone}:
\Ver>\tokenone #1-> ...
#1<-\ifsomecondition<Rev
\item If the \cs{tokentwo} is a macro with one or more
parameters, sufficiently many subsequent tokens will be absorbed
to form the replacement text.\itemliststop

\spoint[expand:edef] Total expansion: \cs{edef}

Macros are usually defined by \cs{def}, but for the cases where
\csterm edef\par
one wants the replacement text to reflect current conditions
(as opposed to conditions at the time of the call),
there is an `expanding define', \cs{edef}, which expands
everything in the replacement text, before assigning it to the
control sequence. 

\example\Ver>
\edef\modedef{This macro was defined in
    `\ifvmode vertical\else \ifmmode math
     \else horizontal\fi\fi' mode}<Rev
The mode tests will be executed at definition time, so the 
replacement text will be a single string.

As a more useful example, suppose that in a file that will be
\cs{input} the category code of the~\n@ will be changed.
One could then write
\Ver>\edef\restorecat{\catcode`@=\the\catcode`@}<Rev
at the start, and 
\Ver>\restorecat<Rev at the end. See page~\pgref[store:cat]
for a fully worked-out version of this.
\>

Contrary to the `one step expansion' of
\cs{expandafter}, the expansion inside an \cs{edef} is complete:
it goes on
until only unexpandable character and control sequence
tokens remain.
There are two exceptions to this total expansion:
\itemlist \item any control sequence preceded by \cs{noexpand}
is not expanded, and,
\item if \cs{sometokenlist} is a token list, the expression
\Ver>\the\sometokenlist <Rev is expanded to the contents
of the list, but the contents are not expanded
any further (see Chapter~\ref[token] for examples).\itemliststop

On certain occasions the \cs{edef} can conveniently be
abused, in the sense that one is not interested in defining
a control sequence, but only in the result of the  expansion.
For example, with the definitions
\alt
\Ver>\def\othermacro{\ifnum1>0 {this}\else {that}\fi}
\def\somemacro#1{ ... }<Rev the call\Ver>
\expandafter\somemacro\othermacro <Rev
gives the parameter assignment
\Ver>#1<-\ifnum<Rev
This can be repaired by calling
\Ver>\edef\next{\noexpand\somemacro\othermacro}\next<Rev
Conditionals are completely expanded inside an \cs{edef},
so the replacement text of \cs{next} will consist of the sequence
\Ver>\somemacro{this}<Rev 
and a~subsequent call to \cs{next} executes this statement.


\spoint \cs{afterassignment}

The \cstoidx afterassignment\par\ command
takes one token and sets it aside for insertion
in the token stream
after the next assignment or macro definition.
If the first assignment is of a~box
 to a box register,
the token will be inserted right after the opening
\alt
brace of the box (see page~\pgref[every:box:assign]).

Only one token can be saved this way; a subsequent token 
saved by \cs{afterassignment} will override the first.

Let us consider an example of the use of \cs{afterassignment}.
It is often desirable to have a macro that will
\itemlist \item assign the argument to some variable, and then
\item do a little calculation, based on the new value
of the variable.\itemliststop
The following example illustrates the
straightforward approach:
\Ver>\def\setfontsize#1{\thefontsize=#1pt\relax
    \baselineskip=1.2\thefontsize\relax}
\setfontsize{10}<Rev
A more elegant solution is possible using \cs{afterassignment}:
\Ver>\def\setbaselineskip
   {\baselineskip=1.2\thefontsize\relax}
\def\fontsize{\afterassignment\setbaselineskip
    \thefontsize}
\fontsize=10pt<Rev
Now the macro looks like an assignment: the equals sign
is even optional. In reality its expansion
ends with a variable to be assigned to. The control sequence
\cs{setbaselineskip} is saved for execution after
the assignment to \cs{thefontsize}.

Examples of \cs{afterassignment} in plain \TeX\ are
the \cs{magnification} and \cs{hglue} macros.
See \cite[Maus] for another creative application of
this command.

\spoint \cs{aftergroup}

Several tokens can be saved for insertion after the current
\csterm aftergroup\par
group with an \disp\cs{aftergroup}\gr{token}\> command.
The tokens are inserted after the group in the sequence
the \cs{aftergroup} commands were given in.
The group can be delimited either by implicit or explicit
braces, or by \cs{begingroup} and \cs{endgroup}.

\example\Ver>{\aftergroup\a \aftergroup\b}<Rev
is equivalent to \Ver>\a \b<Rev
\>

This command has many applications. One can be found
\alt
in the \cs{textvcenter} macro on page~\pgref[text:vcenter];
another one is provided
by the footnote mechanism of plain \TeX.

The footnote command of plain \TeX\ has the layout
\label[footnote:ex]
\disp\cs{footnote}\gr{footnote symbol}\lb\gr{footnote text}\rb
\dispstop which looks like a macro with two arguments.
However, it is undesirable to scoop up the footnote text,
since this precludes for
instance category code changes in the footnote.

What happens in the plain footnote macro is (globally) the following.
\itemlist\item The \cs{footnote} command opens
an insert, \Ver>\def\footnote#1{ ...#1... %treat the footnote sign
    \insert\footins\bgroup<Rev
\item In the insert box a group is opened,
and an \cs{aftergroup} command
is given to close off the insert properly:
\Ver>    \bgroup\aftergroup\@foot<Rev
This command is meant to wind up after the closing brace of
the text that the user typed to end the footnote text;
the opening brace of the user's footnote text must
be removed by 
\Ver>    \let\next=}%end of definition \footnote<Rev
which assigns the next token, the brace, to \cs{next}.
\item The footnote text is set as ordinary text
in this insert box.
\item After the footnote the command \cs{@foot}
defined by \Ver>\def\@foot{\strut\egroup}<Rev
will be executed.\itemliststop


\point Preventing expansion

Sometimes it is necessary to prevent expansion in a place
\csterm noexpand\par
where it normally occurs. For this purpose the control
sequences \cs{string} and \cs{noexpand} are available.

The use of \cs{string} is rather limited, since it converts
a control sequence token into a string of characters, with
the value of \cs{escapechar} used for the character of
category code~0. It is eminently suitable for use in a 
\cs{write}, in order to output a control sequence name
(see also Chapter~\ref[io]); for another application see
the explanation of \cs{newif} in Chapter~\ref[if].

All characters resulting from \cs{string} have category
code~12, `other', except for space characters; they receive
code~10. See also Chapter~\ref[char].

\spoint \cs{noexpand}

The \cs{noexpand} command is expandable, and its expansion
is the following token. The meaning of that token is
made temporarily equal to \cs{relax}, so that it cannot
be expanded further.

For \cs{noexpand} the most important application is probably
in \cs{edef} commands (but in write statements it can often
replace \cs{string}). Consider as an example
\Ver> \edef\one{\def\noexpand\two{\the\prevdepth}}<Rev
Without the \cs{noexpand}, \TeX\ would try to expand
\cs{two}, thus giving an `undefined control sequence' error.

A  (rather pointless)
illustration of the fact that \cs{noexpand} makes the following
token effectively into a \cs{relax} is
\Ver>\def\a{b}
\noexpand\a<Rev This will not produce any output, because the
effect of the \cs{noexpand} is to make the control sequence
\cs{a} temporarily equal to \cs{relax}.

\spoint \cs{noexpand} and active characters

The combination \cs{noexpand}\gr{token} is
\term active characters, and \cs{noexpand}\par
equivalent to \cs{relax}, even if the token
is an active character. Thus,
\Ver>\csname\noexpand~\endcsname<Rev
will not be the same as~\ver>\char`\~>.
Instead it will give an error message, because
unexpandable commands \ldash such as \cs{relax} \rdash  are not allowed to appear
in between \cs{csname} and \cs{endcsname}.
The solution is to use \cs{string} instead; see page~\pgref[store:cat]
for an example.

In another context, however, the sequence
\cs{noexpand}\gr{active character} is equivalent
to the character, but in unexpandable form. This is
when the conditionals \cs{if} and \cs{ifcat} are used
(for an explanation of these, see Chapter~\ref[if]).
Compare
\Ver>\if\noexpand~\relax % is false<Rev
where the character code of the tilde is tested, with
\Ver>\def\a{ ... } \if\noexpand\a\relax % is true<Rev
where two control sequences are tested.

\point \cs{\relax}

The control sequence \cs{relax} cannot be expanded, but
\csterm relax\par
when it is executed nothing happens.

This statement sounds a bit paradoxical, so  consider
an example. Let  counters \Ver>\newcount\MyCount
\newcount\MyOtherCount \MyOtherCount=2<Rev be given.
In the assignment \Ver>\MyCount=1\number\MyOtherCount3\relax4<Rev
the command \cs{number} is expandable, and \cs{relax} is not.
When \TeX\ constructs the number that is to be assigned
it will expand all commands, either until a non-digit is 
found, or until an unexpandable command is encountered.
Thus it reads the~\n1; it expands the sequence \ver>\number\MyOtherCount>,
which gives~\n2; it reads the~\n3; it sees the \cs{relax}, and
as this is unexpandable it halts. The number to be assigned
is then \n{123}, and the whole call has been expanded into
\Ver>\MyCount=123\relax4<Rev
Since the \cs{relax} token has no effect when it is executed,
the result of this line is that \n{123} is assigned to
\cs{MyCount}, and the digit 4 is printed.

Another example of how \cs{relax} can be used to indicate
the end of a command\label[fil:l:l]\ is
\Ver>\everypar{\hskip 0cm plus 1fil }
\indent Later that day, ... <Rev
This will be misunderstood: \TeX\ will see
\Ver>\hskip 0cm plus 1fil L<Rev and \hbox{\n{fil L}} is a~valid,
if bizarre,
way of writing \n{fill} (see Chapter~\ref[gramm]).
One remedy is to write
\Ver>\everypar{\hskip 0cm plus 1fil\relax}<Rev

\spoint[relax:cs] \cs{relax} and \cs{csname}

If a \ver-\csname ... \endcsname- command forms the name
of a previously undefined control sequence,
that control sequence is made equal to \cs{relax},
and the whole statement is also equivalent to \cs{relax}
(see also page~\pgref[cs:name]).

However,  this assignment of \cs{relax} is
\altt
only local:
\Ver>{\xdef\test{\expandafter\noexpand\csname xx\endcsname}}
\test<Rev gives an error message for an
undefined control sequence~\cs{xx}.

Consider as an example the \LaTeX\ environments,
which are delimited by \Ver>\begin{...} ... \end{...}<Rev
The begin and end commands are (in essence)
defined as follows:
\Ver>\def\begin#1{\begingroup\csname#1\endcsname}
\def\end#1{\csname end#1\endcsname \endgroup}<Rev
Thus, for the list environment the commands
\cs{list} and \cs{endlist} are defined, but any
command can be used as an environment name,
even if no corresponding \cs{end...} has been defined.
For instance, \Ver>\begin{it} ... \end{it}<Rev
is equivalent to 
\Ver>\begingroup\it ... \relax\endgroup<Rev
See page~\pgref[begin:end:macros] for the rationale
behind using \cs{begingroup} and \cs{endgroup}
instead of \cs{bgroup} and \cs{egroup}.

\spoint Preventing expansion with \cs{relax}

Because \cs{relax} 
cannot be expanded, a control sequence can be prevented
from being expanded (for instance in an \cs{edef} or a \cs{write})
by making it temporarily equal to \cs{relax}:
\Ver>{\let\somemacro=\relax \write\outfile{\somemacro}}<Rev
will write the string `\cs{somemacro}' to an output file.
It would write the expansion 
of the macro \cs{somemacro} (or give an error message
if the macro is undefined) if the \cs{let} statement
had been omitted.

\spoint[bump:relax] \TeX\ inserts a \cs{relax}

\TeX\ itself inserts \cs{relax} on some occasions.
For instance, \cs{relax} is inserted if \TeX\ encounters an
\cs{or}, \cs{else}, or~\cs{fi} while still determining
the extent of the test. 
\example
\Ver>\ifvoid1\else ... \fi<Rev is changed into
\Ver>\ifvoid1\relax \else ...\fi<Rev internally.
\>

Similarly, if one of the tests \cs{if}, \cs{ifcat}
is given only one comparand, as in \Ver>\if1\else ...<Rev
a \cs{relax} token is inserted. Thus this test
is equivalent to \Ver>\if1\relax\else ...<Rev

Another place where \cs{relax} is used is the following.
While a control sequence is being defined in a \gr{shorthand
definition} \ldash that is, a \gr{registerdef} or \cs{chardef}
or \cs{mathchardef} \rdash  its meaning is temporarily made
equal to \cs{relax}. This makes it possible to write
\ver>\chardef\foo=123\foo>.

\spoint The value of non-macros; \cs{the}

Expansion is a precisely defined activity in \TeX.
\csterm the\par
The full list of tokens that can be expanded
was given above. 
Other tokens than those in the above list may have an `expansion'
in an informal sense. For instance one may wish to `expand'
the \cs{parindent} into its value, say~\n{20pt}.

Converting  the value of (among others) an
\gr{integer parameter}, a \gr{glue parameter}, 
\gr{dimen parameter} or a \gr{token parameter}
into a string of character tokens is done by the expansion processor.
The command \cs{the}
is expanded whenever expansion is not inhibited,
and it takes the value of various sorts of parameters.
Its result (in most cases)
is a string of tokens of category~12, except
that spaces have category code~10.

Here is the list of everything that can be prefixed with \cs{the}.
\description\item \gr{parameter} or \gr{register}
If the parameter or register is of type integer, glue, dimen
or muglue,
its value is given as a string of character tokens;
if it is of type token list (for instance
\cs{everypar} or \cs{toks5}), the result is a string of tokens.
Box registers are excluded here.
\item \gr{codename}\gr{8-bit number}
See page~\pgref[codename].
\item \gr{special register}
The integer registers \cs{prevgraf}, \cs{deadcycles}, \cs{insertpenalties}
\cs{inputlineno}, \cs{badness}, \cs{parshape}, \cs{spacefactor}
(only in horizontal mode), or \cs{prevdepth} (only in vertical mode).
The dimension registers \cs{pagetotal}, \cs{pagegoal}, \cs{pagestretch},
\cs{pagefilstretch}, \cs{pagefillstretch}, \cs{pagefilllstretch},
\cs{pageshrink}, or \cs{pagedepth}.
\item Font properties:
\cs{fontdimen}\gr{parameter number}\gr{font}, 
\cs{skew\-char}\gr{font},
\cs{hy\-phen\-char}\gr{font}.
\item Last quantities:
\cs{lastpenalty}, \cs{lastkern}, \cs{lastskip}.
\item \gr{defined character}
Any control sequence defined by \cs{chardef} or \cs{mathchardef};
the result is the decimal value.
\>
In some cases \cs{the} can give a control sequence token 
or list of such tokens.
\description\item \gr{font}
The result is the control sequence that stands for the
font.
\item \gr{token variable}
Token list registers and \gr{token parameter}s can be prefixed
with \cs{the}; the result is their contents.
\>

Let us consider an example of the use of \cs{the}.
If in a file that is to be \cs{input} the
category code of a character, say the at~sign, is changed,
one could write
\Ver>\edef\restorecat{\catcode`@=\the\catcode`@}<Rev
and call \cs{restorecat} at the end of the file.
If the category code was~11, \cs{restorecat}
is defined equivalent to \Ver>\catcode`@=11<Rev
See page~\pgref[store:cat] for more elaborate macros
for saving and restoring catcodes.


\point Examples

\spoint Expanding after

The most obvious use of \cs{expandafter} is to reach over
a control sequence:
\Ver>\def\stepcounter
    #1{\expandafter\advance\csname 
             #1:counter\endcsname 1\relax}
\stepcounter{foo}<Rev
Here the \cs{expandafter} lets the \cs{csname} command form
the control sequence \cs{foo:counter}; after \cs{expandafter}
is finished the statement has reduced to
\Ver>\advance\foo:counter 1\relax<Rev
It is possible to reach over tokens other than control sequences: in
\Ver>\uppercase\expandafter{\romannumeral \year}<Rev
it expands  \cs{romannumeral} on the other side of the opening
brace.

You can expand after two control sequences:
\Ver>\def\globalstepcounter
    #1{\expandafter\global\expandafter\advance
             \csname #1:counter\endcsname 1\relax}<Rev
If you think of \cs{expandafter} as reversing the evaluation
order of {\sl two\/} control sequences, you can reverse
{\sl three\/} by
\Ver>\expandafter\expandafter\expandafter\a\expandafter\b\c <Rev
which reaches across the three control sequences
\Ver>            \expandafter            \a            \b <Rev 
to expand \cs{c} first.

There is even an unexpected use for \cs{expandafter} in
conditionals;
with \Ver>\def\bold#1{{\bf #1}}<Rev
the sequence \Ver>\ifnum1>0 \bold \fi {word}<Rev
will not give a boldface `word', but
\Ver>\ifnum1>0 \expandafter\bold \fi {word}<Rev will.
The \cs{expandafter} lets \TeX\ see the \cs{fi} and remove it
before it tackles the macro \cs{bold}
(see also page~\pgref[after:cond]).

\spoint Defining inside an \cs{edef}

There is one \TeX\ command that is executed instead of
expanded that is worth pointing out explicitly:
the primitive command \cs{def} (and all other \gr{def} commands)
is not expanded.

Thus the call
\Ver>\edef\next{\def\thing{text}}<Rev will give an `undefined
control sequence' for \cs{thing}, even though after
\cs{def} expansion is ordinarily inhibited (see page~\pgref[noexp:list]).
After \Ver>\edef\next{\def\noexpand\thing{text}}<Rev
the `meaning' of \cs{next} will be \Ver>macro: \def \thing {text}<Rev
The definition \Ver>\edef\next{\def\noexpand\thing{text}\thing}<Rev
will again give an `undefined control sequence' for \cs{thing}
(this time on its second occurrence),
as it will only be defined when \cs{next} is called,
not when \cs{next} is defined.


\spoint[expand:write] Expansion and \cs{write}

The argument token list of \cs{write} is treated in much
\csterm write\par
the same way as the replacement text of an \cs{edef};
that is, expandable control sequences and active characters
are completely expanded. Unexpandable control sequences
are treated by \cs{write} as if they are prefixed
by \cs{string}.

Because of the expansion performed by \cs{write},
some care has to be taken when outputting control
sequences with \cs{write}.
Even more complications arise from the fact that 
the expansion of the argument of \cs{write} is only performed
when it is shipped out. Here follows a worked-out
example.

Suppose \cs{somecs} is a macro, and you
want to write the string 
\disp\ver-\def\othercs-\lb {\italic the expansion of \cs{somecs}}\rb
\dispstop
to a file.

The first attempt is
\Ver>\write\myfile{\def\othercs{\somecs}}<Rev
This gives an error `undefined control sequence' for \cs{othercs},
\altt
because the \cs{write} will try to expand that token. 
Note that the \cs{somecs} is also  expanded, 
so that part is right.

The next attempt is
\Ver>\write\myfile{\def\noexpand\othercs{\somecs}}<Rev
This is almost right, but not quite. The
statement written is
\disp\ver>\def\othercs>\lb{\italic expansion of \cs{somecs}}\rb\dispstop
which looks right.

However, writes \ldash and the expansion of their argument \rdash 
are not executed
on the spot, but saved until the part of the page on which
they occur is shipped out (see Chapter~\ref[io]).
So, in the meantime, the value of \cs{somecs} may have
changed. In other words, the value written may not be the
value at the time the \cs{write} command was given.
Somehow, therefore, the current expansion must be
inserted in the write command.

The following is an attempt at repair:
\Ver>\edef\act{\write\myfile{\def\noexpand\othercs{\somecs}}}
\act<Rev 
Now the  write command will be
\disp\ver>\write\myfile{\def\othercs{>\italic value of\/
     \ver>\somecs}}>\dispstop
The \cs{noexpand} prevented the \cs{edef} from expanding
the \cs{othercs}, but after the definition it has disappeared,
so that execution of the write will again give an undefined control
sequence. The final solution is
\Ver>
\edef\act{\write\myfile
          {\def \noexpand\noexpand \noexpand\othercs{\somecs}}}
\act<Rev 
In this case the write command caused by the expansion of \cs{act}
will be
\disp\ver>\write\myfile{\def\noexpand\othercs>\lb
     {\italic current value of \cs{somecs}}\rb\dispstop
and the string actually written is
\disp\ver>\def\othercs>\lb
     {\italic current value of \cs{somecs}}\rb\dispstop
This mechanism is the basis for cross-referencing
macros in several macro packages.


\spoint Controlled expansion inside an \cs{edef}

Sometimes you may need an \cs{edef} to evaluate current
\howto Control expansion inside an \cs\edef\par
conditions, but you want to expand something in the replacement
text only to a certain level. Suppose that
\Ver>\def\a{\b} \def\b{c} \def\d{\e} \def\e{f}<Rev
is given, and you want to define \cs{g} as \cs{a} expanded
one step, followed by \cs{d} fully expanded. The following
works:
\Ver>\edef\g{\expandafter\noexpand\a \d}<Rev
Explanation: the \cs{expandafter} reaches over the \cs{noexpand}
to expand \cs{a} one step, after which the 
sequence \ver-\noexpand\b- is left.

This trick comes in handy when you need to
construct a control sequence with \cs{csname} inside 
an \cs{edef}. The following sequence inside an \cs{edef}
\Ver>\expandafter\noexpand\csname name\endcsname<Rev
will expand exactly to \cs{name}, but not further.
As an example, suppose
\Ver>\def\condition{true}<Rev has been given, then
\Ver>\edef\setmycondition{\expandafter\noexpand
           \csname mytest\condition\endcsname}<Rev
will let \cs{setmycondition} expand to \cs{mytesttrue}.

\spoint Multiple prevention of expansion

As   was pointed out above, prefixing  a command with
\cs{noexpand} prevents its expansion in commands
such as \cs{edef} and~\cs{write}. However, if a sequence of tokens
passes through more than one expanding command
stronger measures are needed.

The following trick can be used\csterm protect\par:
in order to protect a command against expansion
it can be prefixed with \cs{protect}.
During the stages of processing where expansion is
not desired the definition of \cs{protect} is
\Ver>\def\protect{\noexpand\protect\noexpand}<Rev
Later on, when the command is actually needed,
\cs{protect} is defined as 
\Ver>\def\protect{}<Rev

Why does this work? The expansion of
\Ver>\protect\somecs<Rev is at first
\Ver>\noexpand\protect\noexpand\somecs<Rev
Inside an \cs{edef} this sequence is expanded further,
and the subsequent expansion is
\Ver>\protect\somecs<Rev
That is, the expansion is equal to the original sequence.


\spoint More examples with \cs{relax}

Above, a first example was given in which \cs{relax} served
to prevent \TeX\ from scanning too far.
Here are some more examples, using \cs{relax} to bound
numbers.

After
\Ver>\countdef\pageno=0 \pageno=1
\def\Par{\par\penalty200}<Rev
the sequence \Ver>\Par\number\pageno<Rev is misunderstood as
\Ver>\par\penalty2001<Rev
In this case it is sufficient to define
\Ver>\def\Par{\par\penalty200 }<Rev
as an \gr{optional space} is allowed to follow a number.

Sometimes, however, such a simple escape is not possible.
Consider the definition
\Ver>\def\ifequal#1#2{\ifnum#1=#2 1\else 0\fi}<Rev
The question is whether the space after \ver-#2-
is necessary, superfluous, or simply wrong.
Calls such as \ver-\ifequal{27}{28}- that compare two
numbers (denotations) will correctly give \n1 or~\n0,
and the space is necessary to prevent misinterpretation.

However, \ver-\ifequal\somecounter\othercounter- will
give \n{\char 32 1} if the counters are equal; in this
case the space could have been dispensed with.
The solution that works in both cases is
\Ver>\def\ifequal#1#2{\ifnum#1=#2\relax 1\else 0\fi}<Rev
Note that \cs{relax} is not expanded, so
\Ver>\edef\foo{1\ifequal\counta\countb}<Rev
will define \cs{foo} as either \ver-1\relax1- or~\n{10}.

\spoint[store:cat] Example: category code saving and restoring

In many applications it is necessary to change
\howto Save and restore category codes\par
the category code of a certain character during the
execution of some piece of code. If the writer of
that code is also the writer of the surrounding code,
s/he can simply change the category code back and forth.
However, if the surrounding code is by another author,
the value of the category code will have to be stored
and restored.

Thus one would like to write
\Ver>\storecat@
... some code ...
\restorecat@<Rev
or maybe \Ver>\storecat\%<Rev for characters that
are possibly a comment character (or ignored or invalid).
\alt
The basic idea is to define
\Ver>\def\storecat#1{%
    \expandafter\edef\csname restorecat#1\endcsname
        {\catcode`#1=\the\catcode`#1}}<Rev
so that, for instance, \ver>\storecat$> will define
the single control sequence `\ver>\restorecat$>'
(one control sequence) as \Ver>\catcode`$=3<Rev
The macro \cs{restorecat} can then be implemented as
\Ver>\def\restorecat#1{%
    \csname restorecat#1\endcsname}<Rev
Unfortunately, things are not so simple.

The problems occur with active characters, because these
are expanded inside the \ver>\csname ... \endcsname> pairs.
One might be tempted to write \ver>\noexpand#1> everywhere,
but this is wrong. As was explained above, this is essentially
equal to \cs{relax}, which is unexpandable, and will therefore
lead to an error message when it appears between
\cs{csname} and \cs{endcsname}. The proper solution is then
to use \ver>\string#1>. For the case where the argument
was given as a control symbol (for example~\ver>\%>),
the escape character has to be switched off for a while.
 
Here are the complete macros. The \cs{storecat} macro
gives its argument a default category code of~12.
\Ver>\newcount\tempcounta % just a temporary
\def\csarg#1#2{\expandafter#1\csname#2\endcsname}
\def\storecat#1%
   {\tempcounta\escapechar \escapechar=-1
    \csarg\edef{restorecat\string#1}%
          {\catcode`\string#1=
                \the\catcode\expandafter`\string#1}%
    \catcode\expandafter`\string#1=12\relax
    \escapechar\tempcounta}
\def\restorecat#1%
   {\tempcounta\escapechar \escapechar=-1
    \csname restorecat\string#1\endcsname
    \escapechar\tempcounta}<Rev

\spoint Combining \cs{aftergroup} and boxes

%\tracingmacros=2 \tracingcommands=2
At times, one wants to construct a box and immediately
after it has been constructed to
do something with it. The \cs{aftergroup} command
can be used to put both the commands creating the box,
and the ones handling it, in one macro.

As an example, here is a macro 
\cs{textvcenter}\label[text:vcenter]\
which defines a variant of the \cs{vcenter} box
\howto \cs\vcenter\ outside math mode\par
(see page~\pgref[vcenter]\label[tvcenter])
that can be used outside math mode.
\Ver>\def\textvcenter
   {\hbox \bgroup$\everyvbox{\everyvbox{}%
    \aftergroup$\aftergroup\egroup}\vcenter}<Rev
The idea is that the macro inserts \ver>\hbox {$>,
and that the matching \ver>$}> gets inserted
by the \cs{aftergroup} commands. In order to get the 
\cs{aftergroup} commands inside the box, an
\cs{everyvbox} command is  used.

This macro can even be used with a \gr{box specification}
(see page~\pgref[box:spec]), for example
\Ver>\textvcenter spread 8pt{\hbox{a}\vfil\hbox{b}}<Rev
and because it  is really just an \cs{hbox}, it can also
be used in a \cs{setbox} assignment.

\spoint More expansion

There is a particular charm to macros that work
purely by expansion. See the articles by
\cite[E2], \cite[J], and~\cite[Maus2].

\endinput
