\subject[mathchar] Characters in Math Mode

In math mode every character specifies by its
\cs{mathcode} what position of
a font to access, among other things.
For delimiters this story is a bit
more complicated. This chapter explains the concept
of math codes, and shows how \TeX\ implements variable
size delimiters.
 
\invent
\item mathcode 
      Code of a character determining its treatment in math mode.

\item mathchar 
      Explicit denotation of a mathematical character.

\item mathchardef 
      Define a control sequence to be a synonym for
      a~math character code.

\item delcode 
      Code specifying how a character should be used as delimiter.

\item delimiter 
      Explicit denotation of a delimiter.

\item delimiterfactor 
      1000 times the fraction of a delimited formula that should be
      covered by a delimiter.
      Plain \TeX\ default:~\n{901}

\item delimitershortfall 
      Size of the part of a delimited formula that is allowed 
      to go uncovered by a delimiter.
      Plain \TeX\ default:~\n{5pt}

\item nulldelimiterspace 
      Width taken for empty delimiters. 
      Plain \TeX\ default:~\n{1.2pt}

\item left 
      Use the following character as an open delimiter.

\item right 
      Use the following character as a closing delimiter.

\item big 
      One line high delimiter.

\item Big 
      One and a half line high delimiter.

\item bigg 
      Two lines high delimiter.

\item Bigg 
      Two and a half lines high delimiter.

\item bigl {\MainFont etc.}
      Left delimiters.

\item bigm {\MainFont etc.}
      Delimiters used as binary relations.

\item bigr {\MainFont etc.}
      Right delimiters.

\item radical 
      Command for setting things such as root signs.

\item mathaccent 
      Place an accent in math mode.

\item skewchar 
      Font position of an after-placed accent.

\item defaultskewchar 
      Value of \cs{skewchar} when a font is loaded.

\item skew 
      Macro to shift accents on top of characters explicitly.

\item widehat
      Hat accent that can
      accommodate wide expressions.

\item widetilde
      Tilde accent that can
      accommodate wide expressions.

\inventstop

\point Mathematical characters

Each of the 256 permissible character codes has
\term math characters\par\csterm mathcode\par
an associated \cs{mathcode}, which can be assigned by
\disp\cs{mathcode}\gr{8-bit number}\gr{equals}\gr{15-bit number}\dispstop
When processing in math mode, \TeX\ replaces all characters of
categories 11 and~12, and \cs{char} and \cs{chardef} characters,
by their associated mathcode.

The  15-bit math code is most conveniently denoted hexadecimally
as \ver-"xyzz-, where\disp
\n x${}\leq7$ is the class (see page~\pgref[math:class]),\nl
\n y is the font family number \alt
(see Chapter~\ref[mathfont]), and \nl
\n{zz} is the position of the character in the font.\dispstop

Math codes can also be specified directly by 
\csterm mathchar\par\csterm mathchardef\par
a \gr{math character}, which can be\label[math:character]
\itemlist\item\cs{mathchar}\gr{15-bit number}; 
\item \gr{mathchardef token}, a control sequence that was defined by
\disp\cs{mathchardef}\gr{control sequence}\gr{equals}\gr{15-bit number}
\> or
\item a delimiter command\alt
\disp\cs{delimiter}\gr{27-bit number}\>
 where the last 12 bits
are discarded.\itemliststop
The commands \cs{mathchar} and \cs{mathchardef}
are analogous to \cs{char} and \cs{char\-def} in text mode.
Delimiters are treated below.
A~\gr{mathchardef token} 
can be used as a \gr{number}, even outside math mode.

In \IniTeX\ all letters receive \cs{mathcode} \ver-"71zz- and
all digits receive \ver-"70zz-, where \ver-"zz- is the 
hexadecimal position of the character in the font.
Thus, letters are initially from family~1
(math italic in plain \TeX), and digits are from family~0
(roman).
For all other characters, \IniTeX\ assigns
\disp\cs{mathcode}$\,x=x$,\dispstop
thereby placing them also in family~0.

If the mathcode is \ver-"8000-,
\label[mcode:8000]the smallest integer that is
not a \gr{15-bit number}, the character is treated as an active
character with the original character code. Plain \TeX\
assigns a \cs{mathcode} of \ver-"8000- to the space, underscore and prime.


\point Delimiters

After \cs{left} and \cs{right}
\term delimiters\par\csterm left\par\csterm right\par
commands \TeX\ looks for a delimiter. A~delimiter
is either an explicit \cs{delimiter} command (or a
macro abbreviation for it), or a character with a non-zero
delimiter code.

The \cs{left} and \cs{right} commands
implicitly delimit a group, which is considered as a subformula.
Since the enclosed formula can
be arbitrarily large, the quest for the proper delimiter is
a complicated story of looking at variants in two different
fonts, linked chains of variants in a font, and building
extendable delimiters from repeatable pieces.

The fact that a group enclosed in \ver>\left...\right> is
treated as an independent subformula implies that a
sub- or superscript at the start of this formula is
not considered to belong to the delimiter. 
For example, \TeX\ acts as if 
\ver>\left(_2> is equivalent to \ver>\left({}_2>.
(A~subscript after a \cs{right} delimiter is positioned
with respect to that delimiter.)

\spoint[delcodes] Delimiter codes 

To each character code there corresponds a delimiter
\csterm delcode\par\term delimiter codes\par
code, assigned by
\disp\cs{delcode}\gr{8-bit number}\gr{equals}%
          \gr{24-bit number}\dispstop
A delimiter code thus consists of six hexadecimal digits
\ver-"uvvxyy-, where\disp
\n{uvv} is the small variant of the delimiter, and\nl
\n{xyy} is the large variant;\nl
\n u, \n x are the font families of the variants, and\nl
\n{vv}, \n{yy} are the locations in those fonts.\dispstop
Delimiter codes are used after \cs{left} and \cs{right}
commands.
\IniTeX\ sets all delimiter codes to~$-1$,
except\label[ini:del]
\ver-\delcode`.=0-, which makes the period an empty delimiter.
In plain \TeX\ delimiters have typically \n{u}${}=2$ and~\n{x}${}=3$,
that is, first family~2 is tried, and if no big
enough delimiter turns up family~3 is tried.


\spoint Explicit \cs{delimiter} commands

Delimiters can also be denoted 
\csterm delimiter\par
explicitly by a \gr{27-bit number},
\Ver>\delimiter"tuvvxyy<Rev
where \n{uvvxyy} are the small and large variant of the
delimiter as above;
the extra digit \n{t} (which is~$<8$) denotes the class
(see page~\pgref[math:class]).
For instance, the \cs{langle} macro is defined as
\Ver>\def\langle{\delimiter "426830A }<Rev
which means it belongs to class~4, opening. Similarly,
\cs{rangle} is of class~5, closing; and \cs{uparrow} is of class~3,
relation.

After \cs{left} and \cs{right} \ldash that is, when \TeX\
is looking for a delimiter \rdash  the class digit is ignored;
otherwise \ldash when \TeX\ is not looking for a delimiter \rdash 
the rightmost three digits are ignored, and the
four remaining digits are treated as a~\cs{mathchar}; see above.

\spoint[successor] Finding a delimiter; successors

Typesetting a delimiter is a somewhat involved affair.
\term delimiter sizes\par\term successors\par
First \TeX\ determines the size $y$ of the formula to be covered,
which is twice the maximum of the height and depth of the
formula. Thus the formula may not look optimal if
it is not centred itself.

The size of the delimiter should be at least 
\csterm delimiterfactor\par\csterm delimitershortfall\par
\cs{delimiterfactor}${}\times y/1000$ and at least 
$y-{}$\cs{delimitershortfall}.
\TeX\ then tries first the small variant, and if that one
is not satisfactory (or if the \n{uvv} part of the delimiter
is~\n{000}) it tries the large variant. If trying the large variant
does not meet with success, \TeX\ takes the largest delimiter
encountered in this search; if no delimiter at all was found
(which can happen if  the \n{xyy} part is
\altt
also~\n{000}),
\csterm nulldelimiterspace\par
an empty box of width~\cs{nulldelimiterspace} is taken.

Investigating a variant means, in sequence,
\itemlist \item if the current style (see page~\pgref[math:styles])
is scriptscriptstyle
the \cs{scriptscriptfont} of the family is tried;
\item if the current style is scriptstyle or smaller
the \cs{scriptfont} of the family is tried;
\item otherwise the \cs{textfont} of the family is tried.\itemliststop
The plain format puts the \ver-cmex10- font  in all three
\term extension fonts\par
styles of family~3.

Looking for a delimiter at a certain position in a certain font
means\itemlist\item if the character is large enough, accept it;
\item if the character is extendable, accept it;
\item otherwise, if the character has a successor, that is, it is
part of a chain of increasingly bigger delimiters in the same
font, try the successor.\itemliststop
Information about successors and extensibility of a delimiter
is coded in the font metric file of the font.
An extendable character has a top, a bottom, possibly a mid piece,
and a piece which is repeated directly below the top piece, and
directly above the bottom piece if there is a mid piece.


\spoint \cs{big}, \cs{Big}, \cs{bigg}, and \cs{Bigg}
delimiter macros

In order to be able to use a delimiter outside the 
\ver-\left...\right- context, or to specify a delimiter of
a different size than \TeX\ would have chosen,
four macros for `big' delimiters exist: \cs{big},
\cs{Big}, \cs{bigg}, and \cs{Bigg}. These can be used with
anything that can follow \cs{left} or \cs{right}.

Twelve further macros (for instance \cs{bigl}, \cs{bigm},
\csterm big \Style:roman etc.\par
and~\cs{bigr}) force such delimiters in the context of
an opening symbol, a binary relation, and a closing symbol
respectively:\Ver>\def\bigl{\mathopen\big}
\def\bigm{\mathrel\big} \def\bigr{\mathclose\big}<Rev

The `big' macros themselves put the requested delimiter and
a null delimiter around an empty vertical box:
\Ver>
\def\big#1{{\nulldelimiterspace=0pt \mathsurround=0pt
            \hbox{$\left#1\vbox to 8.5pt{}\right.$}}}<Rev
As an approximate measure,
the \n{Big} delimiters are one and a half times as large (11.5pt) as
\n{big} delimiters; \n{bigg} ones are twice (14.5pt), and \n{Bigg}
ones are two and a half times as large (17.5pt).

\point Radicals

A radical is a compound of a left delimiter and an overlined
math expression.
\term radicals\par\csterm radical\par
The overlined expression is set in the
cramped version of the surrounding style
\alt
(see page~\pgref[math:styles]).

In the plain format and the Computer Modern
math fonts there is only one radical: the square root
construct \Ver>\def\sqrt{\radical"270370 }<Rev
The control sequence \cs{radical} is followed by a \gr{24-bit number}
which specifies a small and a large variant of the left delimiter
as was explained above. Joining the delimiter and the rule
is done by letting the delimiter have a large depth, and a height
which is equal to the desired rule thickness. The rule can then
be placed on the current baseline. After the delimiter and the
ruled expression have been joined the whole is shifted 
vertically to achieve the usual vertical centring 
(see Chapter~\ref[math]).

\point Math accents

Accents in math mode are specified by
\csterm mathaccent\par\term accents in math mode\par
\disp\cs{mathaccent}\gr{15-bit number}\gr{math field}\dispstop
Representing the 15-bit number as \ver>"xyzz>,
only the family~\n{y} and the character position~\n{zz}
are used: an accented expression acts as \cs{mathord} expression
(see Chapter~\ref[math]).

In math mode whole expressions can be accented,
\alt
whereas in text mode only characters can be accented.
Thus in math mode accents can be stacked. However, the top
accent may (or, more likely, will) not be properly positioned
horizontally. Therefore the plain format has a macro \cs{skew}
\csterm skew\par
that effectively shifts the top accent. Its definition is
\Ver>\def\skew#1#2#3{{#2{#3\mkern#1mu}\mkern-#1mu}{}}<Rev
and it is used for instance like
\Ver>$\skew4\hat{\hat x}$<Rev 
\message{skew thing.}
which gives~{\font\tmp=cmmi10 $\textfont\VMIfam=\tmp\skew4\hat{\hat x}
$}.

For the correct positioning of accents over single characters
\csterm skewchar\par
the symbol and extension font have a \cs{skewchar}:
this is the largest accent that adds to the width of an
accented character. Positioning of any accent
is based on the width of the character to be accented,
followed by the skew character. 

The skew characters of the Computer Modern
math italic and symbol fonts are character \n{\hex7F},
\alt
`$\mathchar"!017F$',\message{skew characters}
and \n{\hex30}, `$\mathchar"!0230$', respectively. The \cs{defaultskewchar}
\csterm defaultskewchar\par
value is assigned to the \cs{skewchar} when a font is loaded.
In plain \TeX\ this is~\n{-1}, so fonts ordinarily have no
\cs{skewchar}.

Math accents can adapt themselves to the size of the accented
expression: \TeX\ will look for a successor of an accent
in the same way that it looks for a successor of a delimiter.
In the Computer Modern math fonts this mechanism is used in
\csterm widehat\par\csterm widetilde\par
the \cs{widehat} and \cs{widetilde} macros.
For example,
\disp\ver>\widehat x>, \ver>\widehat{xy}>, \ver>\widehat{xyz}>
\dispstop give
\disp$\widehat x$, $\widehat{xy}$, $\widehat{xyz}$
\dispstop respectively.




\subject[mathfont] Fonts in Formulas

For math typesetting a single current font is not sufficient, as it
is for text typesetting. Instead \TeX\ uses several font families,
and each family can contain three fonts. This chapter
explains how font families are organized, and how \TeX\ determines
from what families characters should be taken.


\invent

\item fam  
      The number of the current font family.

\item newfam  
      Allocate a new math font family.

\item textfont   
      Access the textstyle font of a family.\alt

\item scriptfont  
      Access the scriptstyle font of a family.\alt

\item scriptscriptfont  
      Access the scriptscriptstyle font of a family.\alt

\inventstop

\point Determining the font of a character in math mode

The characters in math formulas can be taken from several
\term font families\par
different fonts (or better, font families) without any user
commands. For instance, in plain \TeX\ math formulas use
the roman font, the math italic font, 
the symbol font and the math extension font.

In order to determine from which font a character is to be
taken, \TeX\ considers for each character in a formula its
\cs{mathcode} (this is treated in Chapter~\ref[mathchar]).
A~\cs{mathcode} is a 15-bit number of the form
\ver."xyzz., where the hex digits
have the following meaning:\disp
\n x:~class,\nl
\n y:~family,\nl
\n{zz}:~position in font.\dispstop

In general only the family determines from what font 
a character is to be taken.
The class of a math character is mostly used to
control spacing and other aspects of  typesetting.
Typical classes include `relation', `operator', `delimiter'.

Class~7 is special in this respect: 
it is called `variable family'. 
If a character has a \cs{mathcode} of the form \ver."7yzz.
it is taken from family \n{y},
unless the parameter \cs{fam} has a value in the range 0--15;
then it is taken from family~\cs{fam}.


\point Initial family settings

Both lowercase and uppercase letters
are defined by \IniTeX\ to have math codes \ver>"71zz>,
\label[ini:fam]%
which means that they are of variable family, initially from
family~1.
As \TeX\ sets \ver.fam=-1., that is,
an invalid value, when a formula starts, 
characters are indeed taken from
family~1, which in plain \TeX\ is math italic.

Digits have math code \ver>"70zz> so they are initially from
family~0, in plain \TeX\ the roman font. 
All other character codes have a mathcode
assigned by \IniTeX\ as
\disp\cs{mathcode}$\,x=x$\dispstop which puts them in class~0,
ordinary, and family~0, roman in plain \TeX.

In plain \TeX, commands such as \cs{sl} then set both a font and
a family:
\Ver>\def\sl{\fam\slfam\tensl}<Rev
so putting \cs{sl} in a formula will cause all letters, digits,
and  uppercase Greek characters, to change to
slanted style.

In most cases, any font can be assigned to any family, but
two families in \TeX\ have a special meaning: these are
families 2 and~3.
For instance, their number of \cs{fontdimen} parameters
is different from the usual~7. Family~2 needs 22 parameters,
and family~3 needs~13. These parameters have all a very
specialized meaning for positioning in math typesetting. 
Their meaning is explained below, but for the full story
the reader is referred to appendix~G of \TeXbook.


\point Family definition

\TeX\ can access 16 families of fonts in math mode;
font families have numbers 0--15. 
The number of the
\csterm fam\par
current family is recorded in the parameter~\cs{fam}.

The macro \cs{newfam} gives the number of an unused family.
\csterm newfam\par
This number is assigned using \cs{chardef} to the control sequence.


Each font family can have a font meant for text style, script style,
and scriptscript style. Below it is explained how \TeX\
determines in what style a (sub-) formula is to be typeset.

Fonts are assigned to a family
\csterm textfont\par\csterm scriptfont\par\csterm scriptscriptfont\par
as follows:
\Ver>\newfam\MyFam
\textfont\MyFam=\tfont \scriptfont\MyFam=\sfont
\scriptscriptfont\MyFam=\ssfont<Rev
for the text, script, and scriptscript fonts of a family.
In general it is not necessary to fill all three members
of a family (but it is for family~3). 
If \TeX\ needs a character from a family member
that has not been filled,
it uses the \cs{nullfont} instead,
a~primitive font that has no characters (nor a \n{.tfm} file).


\point Some specific font changes

\spoint Change the font of ordinary characters and uppercase Greek

All letters and the uppercase Greek characters are
by default in plain \TeX\ of class~7,
variable family, so changing \cs{fam} will change the font
from which they are taken.
For example
\Ver>{\fam=9 x}<Rev  gives an \n{x} from family~9.

Uppercase Greek characters are defined by
\cs{mathchardef} statements in the plain format as \ver>"70zz>,
that is, variable family, initially roman.
Therefore, uppercase Greek character also change with the family.

\spoint Change uppercase Greek independent of text font

In the Computer Modern font layout, uppercase Greek letters
are part of the roman font; see page~\pgref[cmr:table].
\alt
Therefore, introducing another
text font (with another layout)
will change the uppercase Greek characters
(or even make them disappear).
One way of remedying this is by introducing a new family in
which the \n{cmr} font, which contains the uppercase Greek,
resides.
The control sequences accessing these characters then have
to be redefined:
\Ver>
\newfam\Kgreek 
\textfont\Kgreek=cmr10 ...
\def\hex#1{\ifcase#10\or 1\or 2\or 3\or 4\or 5\or 6\or
    7\or 8\or 9\or A\or B\or C\or D\or E\or F\fi}
\mathchardef\Gamma="0\hex\Kgreek00 % was: "0100
\mathchardef\Beta ="0\hex\Kgreek01 % was: "0101
\mathchardef\Gamma ...<Rev
Note, by the way,
the absence of a either a space or a \cs{relax} token after
\n{\#1} in the definition of \cs{hex}. This implies that this
macro can only be called with an argument that is a 
control sequence.

\spoint Change the font of lowercase Greek 
       and mathematical symbols

Lowercase Greek characters have math code
\ver>"01zz>, meaning they are always from the math italic family. 
In order to change this one might redefine them,
for instance \ver.\mathchardef\alpha="710B., 
to make them variable family.
This is not done in plain \TeX, because the Computer Modern
roman font does not
have Greek lowercase, although it does have the uppercase characters.

Another way is to redefine them like \ver.\mathchardef\alpha="0n0B.
where \n{n} is the (hexadecimal) number of a family
compatible with math italic, containing for instance a bold
math italic font.


\point Assorted remarks

\spoint New fonts in formulas

There are two ways to access a font inside mathematics.
\howto Change fonts in a math formula\par
After \cs{font}""\cs{newfont=....} it is not possible to get
the `a' of the new font by \ver-$...{\newfont a}...$-
because \TeX\ does not look at the current font in math mode.
What does work is
\Ver>$ ... \hbox{\newfont a} ...$<Rev
but this precludes the use of the new font in script and 
scriptscript styles.

The proper solution  takes a bit more work:
\Ver>\font\newtextfont=... 
\font\newscriptfont=... \font\newsscriptfont=...
\newfam\newfontfam
\textfont\newfontfam=\newtextfont
\scriptfont\newfontfam=\newscriptfont
\scriptscriptfont\newfontfam=\newsscriptfont
\def\newfont{\newtextfont \fam=\newfontfam}<Rev
after which the font can be used as
\Ver>$... {\newfont a_{b_c}} ...$<Rev
in all three styles.

\spoint Evaluating the families

\TeX\ will only look at what is actually in the \cs{textfont}
et cetera of the various families at the end of the whole
formula. Switching fonts in the families is thus not possible
inside a single formula.
The number of 16 families may therefore turn out to be restrictive
for some applications.


\subject[math] Mathematics Typesetting

\TeX\ has two math modes, display and non-display, and
four styles, display, text, script, and scriptscript style, and
\altt
every object in math mode belongs to one of eight classes.
This chapter treats these concepts.



\invent
\item everymath 
      Token list inserted at the start of a non-display formula.

\item everydisplay
      Token list inserted at the start of a display formula.

\item displaystyle 
      Select the display style of mathematics typesetting.

\item textstyle 
      Select the text style of mathematics typesetting.

\item scriptstyle 
      Select the script style of mathematics typesetting.

\item scriptscriptstyle 
      Select the scriptscript style of mathematics typesetting.

\item mathchoice 
      Give four variants of a formula for the four styles
      of mathematics typesetting.

\item mathord 
      Let the following character or subformula function 
      as an ordinary object.

\item mathop 
      Let the following character or subformula function 
      as a large operator.

\item mathbin 
      Let the following character or subformula function 
      as a binary operation.

\item mathrel 
      Let the following character or subformula function as a relation.

\item mathopen 
      Let the following character or subformula function 
      as a opening symbol.

\item mathclose 
      Let the following character or subformula function
      as a closing symbol.

\item mathpunct 
      Let the following character or subformula function 
      as a punctuation symbol.

\item mathinner 
      Let the following character or subformula function 
      as an inner formula.

\item mathaccent 
      Place an accent in math mode.

\item vcenter 
      Construct a vertical box, vertically centred
      on the math axis.

\item limits 
      Place limits over and under a large operator.

\item nolimits 
      Place limits of a large operator as subscript and 
      superscript expressions.

\item displaylimits 
      Restore default placement for limits.

\item scriptspace 
      Extra space after subscripts and superscripts.
      Plain \TeX\ default:~\n{0.5pt}

\item nonscript 
      Cancel the next glue item if it occurs in 
      scriptstyle or scriptscriptstyle.

\item mkern 
      Insert a kern measured in mu units.

\item mskip 
      Insert glue measured in mu units.

\item muskip 
      Prefix for skips measured in mu units. 

\item muskipdef 
      Define a control sequence to be a synonym for
      a~\cs{muskip} register.

\item newmuskip 
      Allocate a new muskip register.

\item thinmuskip 
      Small amount of mu glue.

\item medmuskip 
      Medium amount of mu glue.

\item thickmuskip 
      Large amount of mu glue. 

\item mathsurround 
      Kern amount placed before and after in-line formulas.

\item over
      Fraction.

\item atop
      Place objects over one another.

\item above
      Fraction with specified bar width. 

\item overwithdelims
      Fraction with delimiters.

\item atopwithdelims
      Place objects over one another with delimiters.

\item abovewithdelims
      Generalized fraction with delimiters.

\item underline 
      Underline the following \gr{math symbol} or group.

\item overline 
      Overline the following \gr{math symbol} or group.


\item relpenalty 
      Penalty for breaking after a binary relation
      not enclosed in a subformula.
      Plain \TeX\ default:~\n{500}

\item binoppenalty 
      Penalty for breaking after a binary operator not enclosed in
      a subformula.
      Plain \TeX\ default:~\n{700}

\item allowbreak 
      Macro for creating a breakpoint.

\inventstop

\point[math:modes] Math modes

\TeX\ changes to math mode when it encounters a math shift
\term math modes\par\term math shift character\par
character, category~3, in the input. After such an opening
math shift it investigates (without expansion) the next
token to see whether this is another math shift.
In the latter case \TeX\ starts processing in display math mode
until a closing double math shift is encountered:
\disp\ver> .. $$ >{\italic displayed formula}\ver> $$ ..>\dispstop
Otherwise it starts processing an in-line formula
in non-display math mode:
\disp\ver> .. $ >{\italic in-line formula}\ver> $ ..>\dispstop
The single math shift character is a \gr{horizontal command}.

Exception: displays are not possible in restricted horizontal
mode, so inside an \cs{hbox} the sequence
\ver>$$> is an empty math formula and
not the start of a displayed formula.

Associated with the two math modes are two \gr{token parameter}
registers (see also Chapter~\ref[token]):
at the start of an in-line formula the \cs{everymath} tokens
\csterm everymath\par
are inserted; at the start of a displayed formula the
\cs{everydisplay} tokens are inserted.
Display math is treated further in the next chapter.

Math modes can be tested for: \cs{ifmmode} is true
in display and non-display math mode, and \cs{ifinner}
is true in non-display mode, but not in display mode.

\point[math:styles] Styles in math mode

Math formulas are set in any of eight styles:
\term math styles\par
\description \item D
display style, \item T
text style, \item S
script style, \item SS
scriptscript style,\descriptionstop
and the four `cramped' variants $D'$, $T'$, $S'$, $SS'$ of
\term cramped styles\par
these. The cramped styles differ mainly in the
fact that superscripts are not raised as far as in
the original styles.

\spoint Superscripts and subscripts

\TeX\ can typeset a symbol or group
\term superscript\par\term subscript\par
as a superscript (or subscript) to the preceding
symbol or group, if that preceding item
does not already have a superscript
(subscript). Superscripts (subscripts) are specified by
the syntax
\disp\gr{superscript}\gr{math field}\dispstop
or 
\disp\gr{subscript}\gr{math field}\dispstop
where a \gr{superscript} (\gr{subscript}) is either a character
of category~7 (8), or a control sequence \cs{let} to such
a character.
The plain format has the control
\csterm\char94\par\csterm\char95\par
sequences
\Ver>\let\sp=^ \let\sb=_<Rev as implicit superscript
and subscript characters.

Specifying a superscript (subscript) expression as the first
item in an empty math list is equivalent to specifying
it as the superscript (subscript) of an empty expression.
For instance, \disp
\ver>$^{...}>\quad is equivalent to\quad \ver>${}^{...}>\dispstop

For \TeX's internal calculations, superscript and subscript
expressions are made wider by \cs{scriptspace};
\csterm scriptspace\par
the value of this in plain \TeX\ is~\n{0.5pt}.

\spoint Choice of styles

Ordering the four styles $D$, $T$, $S$, and~$SS$, and
considering the other four as mere variants, the
style rules for math mode are as follows:
\itemlist\item In any style superscripts and subscripts
are taken from the next smaller style. Exception:
in display style they are taken in script style.
\item Subscripts are always in the cramped variant of
the style; superscripts are only cramped if the original
style was cramped.
\item In an \ver-{..\over..}- formula in any style
the numerator and denominator are taken from the next
smaller style.
\item The denominator is always in cramped style;
the numerator is only in cramped style if the original
style was cramped.
\item Formulas under a \cs{sqrt} or \cs{overline}
are in cramped style.\itemliststop

Styles can be forced by the explicit commands
\alt
\csterm displaystyle\par\csterm textstyle\par
\csterm scriptstyle\par\csterm scriptscriptstyle\par
\cs{displaystyle}, \cs{textstyle}, \cs{scriptstyle},
and~\cs{scriptscriptstyle}.


In display style and text style the \cs{textfont} of the
current family is used, 
in scriptstyle the \cs{scriptfont} is used, and in
\alt
scriptscriptstyle the \cs{scriptscriptfont} is used.

The primitive command
\csterm mathchoice\par
\disp\cs{mathchoice}\lb {\it D\/\rb\lb T\/\rb\lb S\/\rb\lb SS\/}\rb
\dispstop
lets the user specify four variants of a formula for the
four styles. 
\TeX\ constructs all four and inserts the appropriate one.

\point[math:class] Classes of mathematical objects

Objects in math mode belong to one of eight classes. Depending
\term math classes\par
on the class the object may be surrounded by
some amount of white space,
or treated specially in some way. Commands exist to force
symbols, or sequences of symbols, to act as
belonging to a certain class.
In the hexadecimal representation \ver>"xyzz>
the class is the \gr{3-bit number}~\n x.

This is the list of classes and commands that force those
classes. The examples are from the plain format 
(see the tables starting at page~\pgref[math:sym:tables]).
\enumerate \SetCounter:item=-1
\item {\em ordinary\/}: lowercase Greek characters and those symbols
      that are `just symbols'; 
      the command \cs{mathord} forces this class.\csterm mathord\par
\item {\em large operator\/}: integral and sum signs,
      and `big' objects such as \cs{bigcap} or \cs{bigotimes};
      the command \cs{mathop} forces this class.\csterm mathop\par
      Characters that are
      large operators are centred vertically, and they
      may behave differently in display style from in the
      other styles; see below.
\item {\em binary operation\/}: plus and minus,
      and things such as \cs{cap} or \cs{otimes};
      the command \cs{mathbin} forces this class.\csterm mathbin\par
\item {\em relation\/} (also called {\em binary relation\/}): 
      equals, less than, and greater than signs, subset and
      superset, perpendicular, parallel;
      the command \cs{mathrel} forces this class.\csterm mathrel\par
\item {\em opening symbol\/}: opening brace, bracket, parenthesis, angle,
 \altt
      floor, ceiling;
      the command \cs{mathopen} forces this class.\csterm mathopen\par
\item {\em closing symbol\/}: closing brace, bracket, parenthesis, angle,
 \altt
      floor, ceiling;
      the command \cs{mathclose} forces this class.\csterm mathclose\par
\item {\em punctuation\/}: most punctuation marks, but
      \n:~is a relation, the \cs{colon} is a punctuation colon;
      the command \cs{mathpunct} forces this class.\csterm mathpunct\par
\item {\em variable family\/}: symbols in this class change font
      with the \cs{fam} parameter; in plain \TeX\ uppercase
      Greek letters and ordinary letters and digits are
      in this class.
\enumeratestop

There is one further class: the {\em inner\/} subformulas.
No characters can be assigned to this class, but characters and
subformulas can be forced into it by \cs{mathinner}.
\csterm mathinner\par
The \gr{generalized fraction}s and \ver-\left...\right- groups
are inner formulas. Inner formulas are surrounded
by some white space; see the table below.

Other subformulas than those that are inner are treated as
ordinary symbols. In particular, subformulas enclosed in
braces are ordinary: \ver-$a+b$- looks like `$a\mathop+b$', but
\message{Check a+b look}%
\ver-$a{+}b$- looks like~`$a{+}b$'. Note, however, that
in \ver-${a+b}$- the whole subformula is treated as an
ordinary symbol, not its components; 
therefore the result is~`${a+b}$'.

\point Large operators and their limits

The large operators in the Computer Modern fonts come in
two sizes: one for text style and one for display style.
Control sequences such as \cs{sum} are simply defined by
\cs{mathchardef} to correspond to a position in a font:
\Ver>\mathchardef\sum="1350<Rev but if the
current style is display style, \TeX\ looks to see whether
that character has a successor in the font.

Large operators in text style behave as if they are followed
\csterm limits\par\csterm nolimits\par
by \cs{nolimits}, which places the limits as sub/superscript
expressions after the
operator:\disp$\sum_{k=1}^\infty$\dispstop
In display style they behave as if they are followed by
\cs{limits}, which places the limits over and under
the operator:\disp$\displaystyle\sum_{k=1}^\infty$\dispstop
The successor mechanism (see page~\pgref[successor])
\alt
lets \TeX\ take a larger variant
of the delimiter here.

The integral sign has been defined in plain \TeX\ as
\Ver>\mathchardef\intop="1352 \def\int{\intop\nolimits}<Rev
which places the limits after the operator, even in display style:
\disp$\displaystyle\int_0^\infty e^{-x^2}\,dx=\sqrt{\pi}/2$
\dispstop

With \ver-\limits\nolimits- or \ver-\nolimits\limits- the
\csterm displaylimits\par
last specification has precedence; the default placement
can be restored by \cs{displaylimits}. For instance,
\Ver>$ ... \sum\limits\displaylimits ... $<Rev
is equivalent to \Ver>$ ... \sum ... $<Rev and 
\Ver>$$ ... \sum\nolimits\displaylimits ... $$<Rev
is equivalent to
\Ver>$$ ... \sum ... $$<Rev

\point Vertical centring: \cs{\vcenter}

Each formula has an {\em axis\/}, which is for an in-line
\term axis of math formulas\par\term centring of math formulas\par
formula about half the x-height of the surrounding
text; the exact value is the \cs{fontdimen22} of the
font in family~2, the symbol font, in the current style.

The bar line in fractions is placed on the axis; large
operators, delimiters and \cs{vcenter} boxes are centred on it.

A \cs{vcenter}\label[vcenter]
box is a vertical box that is arranged
\csterm vcenter\par
so that it is centred on the math axis.
It is possible to give a \n{spread} or \n{to}
specification with a \cs{vcenter} box.

The \cs{vcenter} box is allowed only in math mode, and
it does not behave like other boxes; for instance, it can
not be stored in a box register. It does not qualify as
a~\gr{box}. See page~\pgref[tvcenter] for a macro that
repairs this.

\point[muglue] Mathematical spacing: \n{mu} glue

Spacing around mathematical objects is measured in \n{mu}
\term math spacing\par\term math unit\par\term mu glue\par
units. A~\n{mu} is $1/18$th part of \cs{fontdimen6}
of the font in family~2 in the current style,
the `quad' value of the symbol font.

\spoint Classification of \n{mu} glue

The user can specify \n{mu} spacing by \cs{mkern} or~\cs{mskip},
\csterm mkern\par\csterm mskip\par
but most \n{mu} glue is inserted automatically by \TeX,
based on the classes to which objects belong (see above).
First, here are some rules of thumb describing the global 
behaviour.

\itemlist \item A \cs{thickmuskip} (default value in plain
\TeX: \n{5mu plus 5mu})
\csterm thickmuskip\par
is inserted around (binary) relations, except where these are
preceded or followed by other relations or punctuation, and
except if they follow an open, or precede a close symbol.
\item A \cs{medmuskip} (default value in plain
\csterm medmuskip\par
\TeX: \n{4mu plus 2mu minus 4mu}) 
is put around binary operators.
\item A \cs{thinmuskip} 
\csterm thinmuskip\par
(default value in plain \TeX: \n{3mu}) follows after
punctuation, and is put around inner objects, except where these
are followed by a close or preceded by an open symbol, and
except if the other object is a large operator or a
binary relation.
\item No \n{mu} glue is inserted after an open or before a close
symbol except where the latter is preceded by punctuation;
no \n{mu} glue is inserted also before punctuation, except where
the preceding object is punctuation or an inner object.
\itemliststop 

The following table gives the complete definition of mu glue
between math objects.
\disp\leavevmode
\vbox{\offinterlineskip
    \halign{#\enspace\hfil&#\enspace\hfil\vrule
           &&\hfil\enspace#\hfil\strut\cr
    \omit\hfil&\omit\hfil& 0:& 1:& 2:& 3:& 4:& 5:& 6:\cr
    \omit\hfil&\omit\hfil&\hfill Ord&\hfill Op&\hfill Bin&\hfill Rel&
                  \hfill Open&\hfill Close&\hfill Punct&\hfill Inner\cr
    \omit\hfil&\omit\hfil&\multispan8\hrulefill\cr
    0:&Ord&    0&  1&(2)&(3)&  0&  0&  0&(1)\cr
    1:&Op&     1&  1&  *&(3)&  0&  0&  0&(1)\cr
    2:&Bin&  (2)&(2)&  *&  *&(2)&  *&  *&(2)\cr
    3:&Rel&  (3)&(3)&  *&  0&(2)&  *&  *&(2)\cr
    4:&Open&   0&  0&  *&  0&  0&  0&  0&  0\cr
    5:&Close&  0&  1&(2)&(3)&  0&  0&  0&(1)\cr
    6:&Punct&(1)&(1)&  *&(1)&(1)&(1)&(1)&(1)\cr
      &Inner&(1)&  1&(2)&(3)&(1)&  0&(1)&(1)\cr
%    \omit\hfil&\omit\hfil&\multispan8\hrulefil\cr
}}
\>
where the symbols have the following meanings:
\itemlist\item 0, no space; 1, thin space; 2, medium space;
     3, thick space;
\item $(\cdot)$, insert only in text and display
     mode, not in script or scriptscript mode;
\item    cases * cannot occur, because a Bin object is converted
    to Ord if it is the first in the list, preceded by
    Bin, Op, Open, Punct, Rel, or followed by Close,
    Punct, and Rel; also, a Rel is converted to Ord when
    \alt
    it is followed by Close or Punct.
\>

Stretchable \n{mu} glue is set according to the same rules that
govern ordinary glue. However, only \n{mu} glue on the outer
level can be stretched or shrunk; any \n{mu} glue enclosed
in  a~group is set at natural width.

\spoint Muskip registers

Like ordinary glue, \n{mu} glue can be stored in registers,
\csterm muskip\par\csterm muskipdef\par\csterm newmuskip\par
the \cs{muskip} registers,
of which there are 256 in \TeX. 
The registers are denoted by
\disp\cs{muskip}\gr{8-bit number}\dispstop
and they can be assigned to a control sequence by
\disp\cs{muskipdef}\gr{control sequence}\gr{equals}\gr{8-bit number}
\dispstop
and there is a macro that allocates unused registers:
\disp\cs{newmuskip}\gr{control sequence}\dispstop
Arithmetic for mu glue exists as for glue; see
Chapter~\ref[glue].

\spoint Other spaces in math mode

In math mode space tokens are ignored; however,
the math code of the space character is \ver-"8000-
in plain \TeX,
so if its category is made `letter' or `other character', it
will behave like an active character in math mode.
See also page~\pgref[mcode:8000].

Admissible glue in math mode is of type~\gr{mathematical skip},
which is either a \gr{horizontal skip} (see Chapter~\ref[hvmode]) 
or~\cs{mskip}\gr{muglue}. Leaders in math mode can be specified
with a \gr{mathematical skip}.

A glue item preceded by \cs{nonscript}
\csterm nonscript\par
is cancelled if it occurs in scriptstyle or scriptscriptstyle.

Control space functions in math mode
\alt
as it does in horizontal mode.

In-line formulas are surrounded by kerns of size
\csterm mathsurround\par
\cs{mathsurround}, the so-called `math-on' and
`math-off' items. Line breaking can occur at the front of
the math-off kern if it is followed by glue.

\point Generalized fractions

Fraction-like objects can be set with six primitive commands
of type \gr{generalized fraction}.
\term generalized fractions\par
Each of these takes the preceding and the following subformulas
and puts them over one another, if necessary with a fraction
bar and with delimiters.
\description \item \cs{over}
   is the ordinary fraction; the bar thickness is \cs{fontdimen8}
   \csterm over\par
   of the extension font: 
   \disp\ver>$\pi\over2$>\quad gives\quad `$\pi\over2$'\message{pi over 2}\>
\item \cs{atop}
   \csterm atop\par
   is equivalent to a fraction with zero bar thickness:
   \disp\ver>$\pi\atop2$>\quad gives\quad `$\pi\atop2$'\dispstop
\item \cs{above}\gr{dimen}
   specifies the thickness
   \csterm above\par
   of the bar line explicitly:
   \disp\ver>$\pi\above 1pt 2$>\quad gives\quad `$\pi\above 1pt 2$'\dispstop
\descriptionstop 

To each of these three there corresponds a \cs{...withdelims} variant
\csterm overwithdelims\par\csterm atopwithdelims\par
\csterm abovewithdelims\par
that lets the user specify delimiters for the expression.
For example, the most general command, in terms of which
all five others could have been defined, is
\disp\cs{abovewithdelims}\gr{delim$_1$}\gr{delim$_2$}\gr{dimen}.
\dispstop
Delimiters in these generalized fractions do not grow with the
enclosed expression: in display mode a delimiter is taken
which is at least \cs{fontdimen20} high, otherwise
\alt
it has to be
at least \cs{fontdimen21} high.
These dimensions are taken
from the font in family~2, the symbol font, in the current style.

The control sequences \cs{over}, \cs{atop}, and \cs{above}
are primitives, although they could have been defined
as \cs{...withdelims..}, that is, with two null delimiters.
Because of these implied surrounding null delimiters,
there is a kern of size \cs{nulldelimiterspace} before and after
these simple generalized fractions. 

\point Underlining, overlining

The primitive commands \cs{underline} and \cs{overline} take a 
\csterm underline\par\csterm overline\par
\gr{math field} argument, that is, a \gr{math symbol} or
a group, and draw a line under or over it.
The result is an `Under' or `Over' atom, which
is appended to the current math list.
The line thickness is font dimension~8 of the extension font,
which also determines the clearance between the line and
the \gr{math field}.

Various other \cs{over...} and \cs{under...} commands exist
in plain \TeX;
these are all macros
that use the \TeX\ \cs{halign} command.

\point Line breaking in math formulas

In-line formulas can be broken after relations and binary operators.
\csterm relpenalty\par\csterm binoppenaly\par
\term penalties in math mode\par
The respective penalties are the \cs{relpenalty} 
and the~\cs{binoppenalty}. However, \TeX\ will only break
after such symbols if they are not enclosed in braces.
Other breakpoints can be created with~\cs{allowbreak},
\csterm allowbreak\par\term breakpoints in math lists\par
which is an abbreviation for~\cs{penalty0}.

Unlike in horizontal or vertical mode where putting two penalties
in a row is equivalent to just placing the smallest one,
in math mode a penalty placed at a break point \ldash that is,
after a relation or binary operator \rdash  will effectively
replace the old penalty by the new one.

\point[fam23:fontdims] Font dimensions of families 2 and 3

If a font is used in text mode, \TeX\ will look at its
first 7 \cs{fontdimen} parameters
(see page~\pgref[font:dims]), for instance to
control spacing.
In math, however, more font dimensions are needed.
\TeX\ will look at the first 22 parameters of the
fonts in family~2, and the first 13 of the fonts in
family~3, to control various
aspects of math typesetting. The next two subsections
have been quoted loosely from~\cite[BB:ISO].

\spoint Symbol font attributes

Attributes of the font in family 2 mainly specify the
\term symbol font\par
initial vertical positioning
of parts of fractions, subscripts, superscripts, et cetera.
The position determined by applying these
attributes may be further modified because of other
conditions, for example the presence of a fraction bar.

One text font dimension, number~6,
the quad, determines the size of mu glue;
see above.

Fraction numerator attributes: minimum shift up, from
the main baseline, of the baseline of the numerator
of a generalized fraction,
\enumerate \SetCounter:item=7
\item num1:
 for display style,
\item num2:
 for text style or smaller if a fraction bar is present,
\item num3:
 for text style or smaller if no fraction bar is present.
\>

Fraction denominator attributes: minimum shift down, from
the main baseline, of the baseline of the denominator
of a generalized fraction,
\enumerate \SetCounter:item=10
\item denom1:
for display style,
\item denom2:
for text style or smaller.
\>

Superscript attributes: minimum shift up, from the main baseline,
of the baseline of a superscript,
\enumerate \SetCounter:item=12
\item sup1:
for display style,
\item sup2:
for text style or smaller, non-cramped,
\item sup3:
for text style or smaller, cramped.
\>

Subscript attributes: minimum shift down, from the main baseline,
of the baseline of a subscript,
\enumerate \SetCounter:item=15
\item sub1:
when no superscript is present,
\item sub2:
when a superscript is present.
\>

Script adjustment attributes: for use only with non-glyph,
that is, composite, objects.
\enumerate \SetCounter:item=17
\item sup\_drop:
maximum distance of superscript baseline below top of nucleus
\item sub\_drop:
minimum distance of subscript baseline below bottom of nucleus.
\>

Delimiter span attributes: height plus depth of delimiter enclosing
a generalized fraction,
\enumerate \SetCounter:item=19
\item delim1:
in display style,
\item delim2:
in text style or smaller.
\>

A parameter with many uses, the height of the math axis,
\enumerate \SetCounter:item=21
\item axis\_height:
the height above the baseline
of the fraction bar, and the centre of large delimiters
and most operators and relations. This position is
used in vertical centring operations.
\>

\spoint Extension font attributes

Attributes of the font in family 3 mostly specify
the way the limits of large operators are set.

The first parameter, number 8, default\_rule\_thickness,
serves many purposes. It
is the thickness of the rule used for overlines,
underlines, radical extenders (square root), 
and fraction bars. Various clearances are  also specified
in terms of this dimension: between the fraction bar and
the numerator and denominator, between an object and
the rule drawn by an underline, overline, or radical,
and between the bottom of superscripts and top of subscripts.

Minimum clearances around large operators are as follows:
\enumerate \SetCounter:item=8
\item big\_op\_spacing1:
minimum clearance between baseline of upper limit and top
of large operator; see below.
\item big\_op\_spacing2:
minimum clearance between bottom of large operator and top of 
lower limit.
\item big\_op\_spacing3:
minimum clearance between baseline of
upper limit and top of large operator,
taking into account depth of upper limit; see below.
\item big\_op\_spacing4:
minimum clearance between bottom of large operator and top of lower
limit, taking into account height of lower limit; see below.
\item big\_op\_spacing5:
clearance above upper limit or below lower limit of a large operator.
\>
The resulting clearance above an operator is the maximum
of parameter~7, and parameter~11 minus the depth of the
upper limit.
The resulting clearance below an operator is the maximum
of parameter~10, and parameter~12 minus the height of the
lower limit.

\spoint Example: subscript lowering

The location of a subscript depends on whether there
\alt
\howto Adjust subscript lowering\par
is a superscript; for instance
\disp $X_1+Y^2_1=1$\>
If you would rather have that look like
\disp $\global\tempdima=\fontdimen16\textfont2\relax
       \global\tempdimb=\fontdimen17\textfont2\relax
       \fontdimen16\textfont2=3pt \fontdimen17\textfont2=3pt
       X_1+Y^2_1=1$,$\fontdimen16\textfont2=\tempdima\relax
                     \fontdimen17\textfont2=\tempdimb\relax$
\>\message{check lowering}
it suffices to specify
\Ver>\fontdimen16\textfont2=3pt \fontdimen17\textfont2=3pt<Rev
which makes the subscript drop equal in both cases.

\subject[displaymath] Display Math

Displayed formulas are set on a line of their own, usually
somewhere in a paragraph. This chapter explains
how surrounding white space (both above/below and to the
left/right) is calculated.


\invent
\item abovedisplayskip \cs{belowdisplayskip}
      Glue above/""below a display.
      Plain \TeX\ default:~\n{12pt plus 3pt minus 9pt}

\item abovedisplayshortskip \cs{belowdisplayshortskip}
      Glue above/""below a display if the line preceding the display 
      was short.
      Plain \TeX\ defaults:~\n{0pt plus 3pt} and
      \n{7pt plus 3pt minus 4pt} respectively.

\item predisplaypenalty \cs{postdisplaypenalty}
      Penalty placed in the vertical list above/""below a display.
      Plain \TeX\ defaults:~\n{10$\,$000} and~\n{0}
      respectively.

\item displayindent 
      Distance by which the box, in which the display 
      is centred, is indented owing to hanging indentation.

\item displaywidth 
      Width of the box in which the display is centred.

\item predisplaysize 
      Effective width of the line preceding the display.

\item everydisplay 
      Token list inserted at the start of a display.

\item eqno 
      Place a right equation number in a display formula.

\item leqno 
      Place a left equation number in a display formula.

\inventstop

\point Displays

\TeX\ starts building a display when it encounters two
\term displays\par
math shift characters (characters of category~3,
\ver>$>~in plain \TeX) in a row.
Another such pair (possibly followed
\alt by one optional space) indicates the end of the display.

Math shift is a \gr{horizontal command}, but displays are only
allowed in unrestricted horizontal mode
(\ver>$$>~is an empty math formula in restricted horizontal mode).
Displays themselves, however, are started in the
surrounding (possibly internal) vertical mode in order to calculate
quantities such as~\cs{prevgraf}; the result of the display is
appended to the vertical list.

The part of the paragraph above the display is broken into
lines as an independent paragraph (but \cs{prevgraf} is
carried over; see below), and the remainder of the
paragraph is set, starting with an empty list and \cs{spacefactor}
equal to~1000. 
The \cs{everypar} tokens are not inserted for the part of the
paragraph after the display, nor is \cs{parskip} glue inserted.
 
Right at the beginning of the display the \cs{everydisplay}
\csterm everydisplay\par
token list is inserted (but after the calculation of
\cs{displayindent}, \cs{displaywidth}, and \cs{predisplaysize}).
See page~\pgref[left:display] for an example of the use
of \cs{everydisplay}.

The page builder is exercised
before the display 
(but after the \cs{everydisplay} tokens have been inserted),
and after the display finishes.

The `display style' of math typesetting was treated in 
Chapter~\ref[mathfont].

\point Displays in paragraphs

Positioning of a display in a paragraph may be influenced
by hanging indentation or a \cs{parshape} specification.
For this, \TeX\ uses the \cs{prevgraf} parameter
(see Chapter~\ref[par:shape]), and
acts as if the display is three lines deep.

If $n$ is the value of \cs{prevgraf} when the display starts
\ldash so there are $n$ lines of text above the display \rdash 
\cs{prevgraf} is set to to $n+3$ when the paragraph resumes.
The display occupies, as it were, lines $n+1$, $n+2$, and~$n+3$.
The shift and line width for the display are those
that would hold for line~$n+2$.

The shift for the display is recorded in \cs{displayindent};
\csterm displayindent\par\csterm displaywidth\par
the line width is recorded in \cs{displaywidth}. These parameters
(and the \cs{predisplaysize} explained below)
are set immediately after the \ver>$$> has been scanned.
Usually they are equal to zero and \cs{hsize} respectively.
The user can change the values of these parameters; 
\TeX\ will use the
values that hold after the math list 
of the display has been processed.

Note that a display is vertical material, and therefore
not influenced by settings of \cs{leftskip} and \cs{rightskip}.

\point Vertical material around displays

A display is preceded in the vertical list by
\itemlist\item a penalty of size \cs{predisplaypenalty}
   \csterm predisplaypenalty\par\csterm abovedisplayskip\par
   \csterm abovedisplayshortskip\par
(plain \TeX\ default~$10\,000$), and
\item glue of size \cs{abovedisplayskip} 
or \cs{abovedisplayshortskip}; this glue is omitted in
cases where a~\cs{leqno} equation number is set on
a line of its own (see below).\itemliststop
A display is followed by 
\itemlist\item a penalty of size \cs{postdisplaypenalty}
   \csterm postdisplaypenalty\par\csterm belowdisplayskip\par
   \csterm belowdisplayshortskip\par
(default~0), and possibly
\item glue of size \cs{belowdisplayskip} or 
\cs{belowdisplayshortskip}; this glue is omitted in cases
where an~\cs{eqno} equation number is set on a line of
its own (see below).\itemliststop

The `short' variants of the glue are taken if
there is no \cs{leqno} left equation number, and if
the last line of the paragraph above the display is
short enough for the display to be raised a bit without
coming too close to that line.
In order to decide this, the effective width of the
preceding line is saved in \cs{predisplaysize}.
    \csterm predisplaysize\par
This value is calculated immediately after the opening \ver>$$>
of the display has
been scanned, together with the \cs{displaywidth}
and \cs{displayindent} explained above.

Remembering that the part of the paragraph above the display
has already been broken into lines, the following method
for finding the effective width of the last line ensues.
\TeX\ takes the last box of the list, which is a horizontal
box containing the last line, and locates the right edge
of the last box in it. The \cs{predisplaysize} is then
the place of that rightmost edge, plus any amount by which
the last line was shifted, plus two ems in the current font.

There are two exceptions to this. The \cs{predisplaysize}
is taken to be $-$\cs{maxdimen} if there was no previous line,
that is,
the display started the paragraph, or it followed another display;
\cs{predisplaysize} is taken to be \cs{maxdimen}
\term machine dependence\par
if the glue in the last line was not set at its natural width,
which may happen if the \cs{parfillskip} contained only finite
stretch. The reason for the last clause is that glue
setting is slightly machine"-dependent, and such dependences
should be kept out of \TeX's global decision processes.

\point Glue setting of the display math list

The display has to fit in \cs{displaywidth}, 
but in addition to the formula there
may be an equation number. The minimum separation
between the formula and the equation number should
be one em in the symbol font, that is,
\cs{font\-dimen\-6}""\cs{textfont2}.

If the formula plus any equation number
and separation fit into \cs{displaywidth},
the glue in the formula is set at its natural width. 
If it does not fit,
but the formula contains enough shrink, it is shrunk.
Otherwise \TeX\ puts any equation number
on a line of its own, and the glue in the formula is
set to fit it in \cs{displaywidth}.
With the equation
number on a separate line the formula may now very well fit in the
display width; however,
if it was a very long formula the box in which it is
set may still be overfull. \TeX\ nevers breaks a displayed
formula.

\point Centring the display formula: displacement

Based on the width of the box containing the formula \ldash which
may not really `contain' it; it may be overfull \rdash 
\TeX\ tries to centre the formula in the \cs{displaywidth},
that is, without taking the equation number into account.
Initially, a displacement is calculated that is 
half the difference between \cs{displaywidth} and the
width of the formula box.

However, if there is an equation number that will not
be put on a separate line and the displacement is less than
twice the width of the equation number, a new displacement
is calculated. This new displacement is zero if the formula
started with glue; otherwise it is such that the
formula box is centred in the space left by the equation
number.

If there was no equation number, or if the equation number
will be put on a separate line, the formula box
is now placed, shifted right by \cs{displayindent} plus
the displacement calculated above.

\point Equation numbers

The user can specify a equation number for a display
by ending it with 
\csterm eqno\par\csterm leqno\par\term equation numbering\par
\Disp\cs{eqno}\gr{math mode material}\ver>$$>\Dispstop
for an equation number placed on the right, or
\Disp\cs{leqno}\gr{math mode material}\ver>$$>\Dispstop
for an equation number placed on the left.

\spoint Ordinary equation numbers

Above it was described how \TeX\ calculates a displacement
from the display formula and the equation number, if this
is to be put on the same line as the formula.

If the equation number was a  \cs{leqno} number,
\TeX\ places a box containing
\itemlist\item the equation number,
\item a kern with the size of the displacement calculated, and
\item the formula.\itemliststop
This box is shifted right by \cs{displayindent}.

If the equation number was an \cs{eqno} number,
\TeX\ places a box containing
\itemlist\item the formula,
\item a kern with the size of the displacement calculated, and
\item the equation number.\itemliststop
This box is shifted right by \cs{displayindent} plus
the displacement calculated.

\spoint The equation number on a separate line

Since displayed formulas may become rather big, \TeX\ can decide
(as was described above)
that any equation number should be placed on a line of its own.
A~left-placed equation number is then to be placed above the
display, in a box that is shifted right by \cs{displayindent};
a right-placed equation number will be placed below the display,
in a box that is shifted to the right 
by \cs{displayindent} plus \cs{displaywidth} minus the width of
the equation number box.

In both cases a penalty of $10\,000$ is placed between the equation
number box and the formula.

\TeX\ does not put extra glue above a left-placed
equation number or below
a right-placed equation number; \TeX\ here relies on
the baselineskip mechanism.


\point[left:display] Non-centred displays

As a default, \TeX\ will centre displays.
\term displays, non-centred\par
In order to get non-centred displays some
macro trickery is needed. 

One approach would
be to write a macro \cs{DisplayEquation}
that would basically look like
\Ver>\def\DisplayEquation#1{%
    \par \vskip\abovedisplayskip
    \hbox{\kern\parindent$\displaystyle#1$}
    \vskip\belowdisplayskip \noindent}<Rev
but it would be nicer if one could just write
\Ver>$$ ... \eqno ... $$<Rev
and having this come out as a left"-aligning display.

Using the \cs{everydisplay} token list, the above
idea can be realized. The basic idea is to write
\Ver>\everydisplay{\IndentedDisplay}
\def\IndentedDisplay#1$${ ...<Rev
so that the macro \cs{IndentedDisplay}
will receive the formula, including any equation number.
The first step is now to extract an equation number
if it is present. This makes creative use of delimited
macro parameters.\Ver>
\def\ExtractEqNo#1\eqno#2\eqno#3\relax
   {\def\Equation{#1}\def\EqNo{#2}}
\def\IndentedDisplay#1$${%
    \ExtractEqNo#1\eqno\eqno\relax<Rev
Next the equation should be set in the available
space \cs{displaywidth}:
\Ver>    \hbox to \displaywidth
        {\kern\parindent
         $\displaystyle\Equation$\hfil$\EqNo$}$$
    }<Rev
Note that the macro ends in the closing \ver>$$>
to balance the opening dollars that caused
insertion of the \cs{everydisplay} tokens.
This also means that the box containing the
displayed material will automatically be
surrounded by \cs{abovedisplayskip} and
\cs{belowdisplayskip} glue.
There is no need to use \cs{displayindent} anywhere
in this macro, because \TeX\ itself will shift the
display appropriately.

\endinput
baselineskip around displays?



