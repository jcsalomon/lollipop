% appendix.tex copyright1992 Victor Eijkhout
%                       copyright 2014 Vafa Khalighi
%
%
%    This program is free software: you can redistribute it and/or modify
%    it under the terms of the GNU General Public License as published by
%    the Free Software Foundation, either version 3 of the License, or
%    (at your option) any later version.
%
%    This program is distributed in the hope that it will be useful,
%    but WITHOUT ANY WARRANTY; without even the implied warranty of
%    MERCHANTABILITY or FITNESS FOR A PARTICULAR PURPOSE.  See the
%    GNU General Public License for more details.
%
%    You should have received a copy of the GNU General Public License
%    along with this program.  If not, see <http://www.gnu.org/licenses/>.
%
%
\Chapter Example styles

To show you the strength of \Lollipop, this chapter collects a few
example style definitions. The first one is that of this manual.

\Section The style definition for this book

In case you were wondering how this book was typeset, here is the
full style definition. By the standards of what Lollipop can do it is
pretty pedestrian.

One thing that may have provide intellectual titilation is the
definition of \cs{Example} and \cs{OutExample}.
It allowed me to keep the examples in sync with their output.

Of course that
doesn't really rely on \Lollipop. It does illustrate the fact that
\Lollipop\ is interfaceable to arbitrary macros. (But don't try
loading \Lollipop\ on top of \LaTeX! On second thought, do. It
disables most of \LaTeX. Just kidding.)

\begingroup \PointSize:8 \tt 
\verbatimfile{mandefs.tex}\endgroup

\Section[sec:address] Address book

The following macros generate an address book. Several noteworthy
features:
\Itemize\item Most titles are short, that is, delimited by the line
end.
 \item Since a page will now have several dozens of headings, the
number of marks placed will become a problem, therefore the option
\opt{nomarks} is included everywhere. Without this you would easily
have memory overflows.
 \item The \cs{At} heading writes its information to an external
file. This is then parsed by the macro \cs{CompNam}. A~slight amount
of knowledge of Lollipop internals is used here for parameter parsing,
but not more than can be gleaned from simply looking at the external
file.\par
 Then a token list is created for each company, and these lists are
printed somewhere down the file. This is a bit of \TeX\ programming
that is not quite elementary, but still \Lollipop saves you a lot of
work.
 \>
If you want to see the output, run \TeX\ with Lollipop twice on the
\file{address.tex} file.

 \begingroup \PointSize:8 \tt 
\verbatimfile{address.tex}\endgroup

\endinput