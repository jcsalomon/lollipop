% lollipop-fonts.tex copyright 1992/3 Victor Eijkhout
%                 copyright 2014    Vafa Khalighi
%
%
%    This program is free software: you can redistribute it and/or modify
%    it under the terms of the GNU General Public License as published by
%    the Free Software Foundation, either version 3 of the License, or
%    (at your option) any later version.
%
%    This program is distributed in the hope that it will be useful,
%    but WITHOUT ANY WARRANTY; without even the implied warranty of
%    MERCHANTABILITY or FITNESS FOR A PARTICULAR PURPOSE.  See the
%    GNU General Public License for more details.
%
%    You should have received a copy of the GNU General Public License
%    along with this program.  If not, see <http://www.gnu.org/licenses/>.
%
%
%%%%%%%%%%%%%%%%%%%%%%%%%%%%%%%%%%%%%%%%%%%
%%%%%%%%% Font selection macros %%%%%%%%%%%

%%%%
%%%% Text Fonts
%%%%

\NewTrace:font

% User commands to change the three parameters

% \Typeface:FACE
%     \F@name <-- FACE
%
\def\Typeface:#1
   {\if\UndefinedCS{#1@code}\Wmessage{Undefined typeface: `#1'}\fi
    \def\F@name{#1}}
\let\TypeFace\Typeface

% \Pointsize:NN
%     \Fsize <-- NN
%     compute script/sscript sizes
%     if necessary install math fonts
%
\def\PointSize:#1 {\tempcounta=\@ww{#1}%
    \edef\Fsize{\the\tempcounta}%this line is used in \set@mathfont
    \xp\@setsizes\xp{\Fsize}%
    \xp\pointsize@baselineskip\xp{\Fsize}}
\let\Pointsize=\PointSize

% \Style:STYLE
%     \F@style <-- STYLE
%     load actual font
%     put font in \textfont0
%     install value of \@space
%
\def\Style:#1 {\def\F@style{#1}%
    \ifmmode\SetFamily\else\SetFont\fi}
\def\ifStyle:#1 {\xp\if\xp\EqualStringX\xp{\F@style}{#1}}

% and together this makes:
%
\def\current@font{\F@name\F@style\romannumeral\Fsize}


% abbreviations for common styles
%
\def\tty{\Style:typewriter }
\def\rm{\Style:roman }\def\bf{\Style:bold }\def\it{\Style:italic }
\def\sl{\Style:italic }

% Command for actual loading is also user command.
% If this is in a style definition, fill the math families.
%
\def\SetFont{\maybe@load@font
    \if@started\else\store@main@font\fi
    \ifx\cs@a\relax %error condition if load not succeeded
    \else \csname\cs@a\endcsname
          \@@space=.5em\relax
    \fi}
\def\maybe@load@font{\edef\cs@a{\current@font}%
    \if\UndefinedCS{\cs@a}\load@font\fi}
\def\load@font{\def@f@ext@name
    \Tmessage[font]{Loading font (text): \cs@a=`\f@ext@name'}%
    \edef\cs@e{\global\font\CSname{\cs@a}=\f@ext@name\relax}\cs@e
    }
\def\def@f@ext@name
   {\edef\f@ext@name{\ext@combo}%
    \edef\f@ext@name{\ext@name
         \ifx\f@ext@name\@empty \ext@style\ext@size
         \else \f@ext@name \fi}}
    
\def\face@tag{\F@name @code}
\def\cs@face@tag{\csname\face@tag\endcsname}

% Find relative sizes of the current size
% and install math fonts
%
\def\@ps@rls#1{ps@rls\romannumeral#1}
\def\PointSizeScriptSizes:#1=#2,#3,#4 
   {\csarg\def{\@ps@rls#1}{{#2}{#3}{#4}}}
\def\@setsizes#1{\if\UndefinedCS{\@ps@rls{#1}}\tempcounta#1\relax
        \tempcountb\tempcounta \multiply\tempcountb7 \divide\tempcountb10
        \tempcountc\tempcounta \divide\tempcountc2
        \edef\@@setsizes
           {{\the\tempcounta}{\the\tempcountb}{\the\tempcountc}}%
    \else
        \edef\@@setsizes{\csname\@ps@rls{#1}\endcsname}%
    \fi
    \expandafter\@setsize\@@setsizes}
\PointSizeScriptSizes:0=0,0,0
\PointSizeScriptSizes:1=1,1,1
\PointSizeScriptSizes:2=2,2,2
\PointSizeScriptSizes:3=2,3,3
\PointSizeScriptSizes:4=4,4,4
\PointSizeScriptSizes:5=5,5,5
\PointSizeScriptSizes:6=6,5,5
\PointSizeScriptSizes:7=7,6,5
\PointSizeScriptSizes:8=8,6,5
\PointSizeScriptSizes:9=9,7,5
\PointSizeScriptSizes:10=10,7,5
\PointSizeScriptSizes:11=11,8,6

% Fill in a math font family corresponding to the
% surrounding size
\def\SetFamily{\ifface@size@changed\begingroup
%\tracingmacros=2 \tracingcommands\tracingmacros
    \edef\tmp@sizes{\csname\@ps@rls{\Fsize}\endcsname}
    \edef\Fsize{\xp\first@of@three\tmp@sizes}\maybe@load@font
         \edef\cs@e{\global\textfont
              \CSname{\F@style @fam}\CSname{\cs@a}}\cs@e
    \edef\Fsize{\xp\second@of@three\tmp@sizes}\maybe@load@font
         \edef\cs@e{\global\scriptfont
              \CSname{\F@style @fam}\CSname{\cs@a}}\cs@e
    \edef\Fsize{\xp\third@of@three\tmp@sizes}\maybe@load@font
         \edef\cs@e{\global\scriptscriptfont
              \CSname{\F@style @fam}\CSname{\cs@a}}\cs@e
    \endgroup\fi
    \csarg\fam{\F@style @fam}}
\newfam\roman@fam  \def\roman@face@size{{foo}{bar}}
\newfam\italic@fam \def\italic@face@size{{foo}{bar}}
\newfam\bold@fam   \def\bold@face@size{{foo}{bar}}
\newfam\tty@fam    \def\tty@face@size{{foo}{bar}}
\def\ifface@size@changed{\iftrue}
\def\temporarily@out@of@order
   {\edef\cs@a{\csn\F@style @face@size\ecs}
    \if\EqualString{\F@name}{\xp\take@former\cs@a}
       \if\EqualString{\Fsize}{\xp\take@latter\cs@a}
             \csn iffalse\xp\xp\xp\ecs
       \else \Tmessage[font]{Loading family <\F@style>}
             \csarg\edef{\F@style @face@size}%
                {{\F@name}{\Fsize}}
             \csn iftrue\xp\xp\xp\ecs\fi
    \else \Tmessage[font]{Loading family <\F@style>}
          \csarg\edef{\F@style @face@size}%
                {{\F@name}{\Fsize}}
          \csn iftrue\xp\ecs \fi}

% Define relative sizes,
% and install math fonts,
% loading if necessary
%
\def\@setsize#1#2#3{%
    \def\normal{\def\Fsize{#1}\SetFont}\relax
    \def\script{\def\Fsize{#2}\SetFont}\relax
    \def\scriptscript{\def\Fsize{#3}\SetFont}\relax
    \def\style@sizes{#1,#2,#3 }}
\newif\ifloading@math
\def\math@setsize#1,#2,#3 {%
    \SaveFont \loading@mathyes
    \set@mathfont{#1}\textfont
    \set@mathfont{#2}\scriptfont
    \set@mathfont{#3}\scriptscriptfont
    \RestoreFont \loading@mathno
    }

% these lines are in 'text.tex'
%\everymath{\the\every@everymath}
%\everydisplay{\the\every@everymath}
%\every@everymath{\high@fam\count18\relax \xp\math@setsize\style@sizes}

% Strut: .8\baselineskip+.2\baselineskip
% note: struts should be locally defined!
%
\def\make@strut{\tempdima=.8\baselineskip \tempdimb=.2\baselineskip
    \edef\strut{\vrule height \the\tempdima width \z@
                       depth \the\tempdimb \relax}}

% Main Font
% is the font that's current at the end of the style definition
% this is switched to in the output routine (head/footline!)
% and the user can use it too.
% this does not take properly care of baselineskip and strut.
% nor math fonts, come to think of it.
%
\def\store@main@font
   {\edef\MainFace{\def\nxp\F@name{\F@name}}
    \SaveFont \let\MainFont\RestoreFont}
\let\MainFont\relax %cs is used in output@setup, which may be called
                    %from Lollipop.tex

% Save and restore fonts
%
\def\SaveFont
   {\edef\RestoreFont{\def\nxp\F@name{\F@name}%
         \def\nxp\Fsize{\Fsize}%
         \def\nxp\F@style{\F@style}%
         \nxp\SetFont
         \baselineskip=\the\baselineskip\relax
                  }
    }
\def\RestoreFont{\Wmessage{No `Savefont' has been given}}

% Defined Fonts
% rather limited, because not upscalable.
%
\def\DefineFont:#1=#2,#3,#4 {\begingroup
    \Typeface:#2 \PointSize:#3 \Style:#4 
    \edef\cs@e{\global\let\CSname{#1}=\CSname{\current@font}}\cs@e
    \endgroup}


% Increase / decrease point size
%
\def\PointSizeLarger{\if\NextChar[%]
    \ps@plus{\ps@plus[1]}}
\def\PointSizeSmaller{\if\NextChar[%]
    \ps@minus{\ps@minus[1]}}
\def\ps@plus[#1]{\let\save@Fsize\Fsize
    \tempcounta\Fsize\relax \advance\tempcounta#1\relax
    \edef\Fsize{\the\tempcounta}\SetFont
    \let\Fsize\save@Fsize}
\def\ps@minus[{\ps@plus[-%]
    }%]

% Baselineskip
%
\def\BaselineSkip:#1 {\baselineskip\@ww{#1}\make@strut}
\def\@ps@bls#1{ps@bls\romannumeral#1}
\def\@@ps@bls#1{\tempcounta#1\relax\baselineskip\tempcounta pt
    \baselineskip\bls@ps@ratio\baselineskip\relax}
\def\BaselineSkipPointSizeRatio:#1 {\def\bls@ps@ratio{#1}}
\BaselineSkipPointSizeRatio:1.2
\def\PointSizeBaselineSkip:#1
   {\edef\cs@e{\nxp\pointsize@baselineskip{\w@w{#1}}}\cs@e}%nxp?
\def\pointsize@baselineskip#1{%also used in \PointSize
    \if\UndefinedCS{\@ps@bls{#1}}\@@ps@bls{#1}%
    \else \baselineskip\csname \@ps@bls{#1}\endcsname \fi
    \make@strut}
\def\SetPointSizeBaselineSkip:#1=#2 
   {\edef\cs@e{\w@w{#1}}%
    \csarg\edef{\@ps@bls\cs@e}{\w@w{#2}}}

\def\exblssizes#1{\ifcase#1\z@ \or \z@ \or \z@ \or \z@ \or \z@
    \or 6.1pt  %5
    \or 7.3pt  %6
    \or 8.5pt  %7
    \or 9.7pt  %8
    \or 11pt   %9
    \or 12pt   %10
    \or 13pt   %11
    \or 14.5pt %12
    \or 15.6pt %13 
    \or 16.8pt %14
    \or 18pt   %15
    \or 19.2pt %16
    \or 20.4pt %17
    \or 21.6pt %18 
    \or 22.8pt %19 
    \or 24pt   %20
    \else \ifnum#1=24 28pt \else
          \ifnum#1=30 36pt \else
          \ifnum#1=32 38pt \else
          \ifnum#1=48 56pt \else
          \z@  \Emessage{Unknown pointsize: #1}\fi\fi\fi\fi
    \fi \normalbaselineskip\baselineskip} 

%%%% Math Fonts
% Math fonts are loaded through the \everymath and \everydisplay.
% This means that they are undefined outside math.
% Therefore, any load during the style definition is global.
%
% Symbol and extension fonts don't change with type
% Control sequences \m@.. are \let to 0 or 1 to indicate that
%  math at a certain size has been loaded

% The ten point extension font is special because it is used
% for all smaller sizes
\font\exx=cmex10

% Size to external name conversion for math fonts
\def\mi@size#1{\ifcase#1 \or \or \or \or \or 5\or 6\or 7\or 8\or
    9\or 10\or 10 \sc@led \magstephalf \or 10 \sc@led \magstep1\or
    \or 10 \sc@led \magstep2\or \or 10 \sc@led \magstep3\or
    \or 10 \sc@led \magstep3\or \or 10 \sc@led \magstep4\fi}
\let\sy@size=\mi@size \let\cm@size=\mi@size
\def\ex@size#1{\ifnum#1<10 10 \else\mi@size{#1}\fi}

% For a size, test if it has been loaded yet, if not do;
% then install as given style
%
\def\set@mathfont#1#2%% #1=size #2=\text/script/scriptscript style
   {\if\UndefinedCS{m@\romannumeral#1}\begingroup
     \Tmessage[font]{Loading math style \string#2 at #1pt}%\globaldefs\@ne
     \load@mathfammember{cm}{MathIT}{mi}{#1}\mi@size{'177}\relax
     \load@mathfammember{cm}{MathSY}{sy}{#1}\sy@size{'060}\relax
     \load@mathfammember{cm}{MathRM} {r}{#1}\cm@size{-1}\relax
     \ifnum#1>10 \load@mathfammember{cm}{MathEX}{ex}{#1}\ex@size{-1}\relax
     \else \global\csarg\let{MathEX\romannumeral#1}\exx
     \fi
     \global\csarg\let{m@\romannumeral#1}1\relax
    \endgroup\fi
    \if\UndefinedCS{\F@name VMI\romannumeral#1}\begingroup
     \def\Fsize{#1}\def\F@style{mitalic}%
     \load@mathfammember{\cs@face@tag}
            {\F@name VMI}{\ext@style}{#1}
            {\ext@size\take@one}{-1}\relax
    \endgroup\fi
    \def\Fsize{#1}\SetFont #20\the\font
    #21 \csname MathIT\romannumeral#1\endcsname
    #22 \csname MathSY\romannumeral#1\endcsname
    #23 \csname MathEX\romannumeral#1\endcsname
    #2\Kgriek \csname MathRM\romannumeral#1\endcsname
    #2\VMIfam \csname \F@name VMI\romannumeral#1\endcsname}
\def\load@mathfammember
    #1#2#3#4#5#6%
        %% #1=root of external name #2=internal tag
        %% #3=external tag #4=size #5=size convertor #6=skew character
   {\Tmessage[font]{Loading font (math)
                \csname#2\romannumeral#4\endcsname=#1#3#5{#4}}%
    \global\xp\font\csname#2\romannumeral#4\endcsname=
        #1#3#5{#4}\relax
    \global\xp\skewchar\csname#2\romannumeral#4\endcsname=#6\relax
   }

% Allocate a family for cmr, which has Greek capitals
%
\storecat"    
\ifini{\newfam\Kgriek}
\mathchardef\Gamma="0\hexdigit\Kgriek 00
\mathchardef\Delta="0\hexdigit\Kgriek 01
\mathchardef\Theta="0\hexdigit\Kgriek 02
\mathchardef\Lambda="0\hexdigit\Kgriek 03
\mathchardef\Xi="0\hexdigit\Kgriek 04
\mathchardef\Pi="0\hexdigit\Kgriek 05
\mathchardef\Sigma="0\hexdigit\Kgriek 06
\mathchardef\Upsilon="0\hexdigit\Kgriek 07
\mathchardef\Phi="0\hexdigit\Kgriek 08
\mathchardef\Psi="0\hexdigit\Kgriek 09
\mathchardef\Omega="0\hexdigit\Kgriek 0A

% Math italic variables have to be from the current typeface
% maybe this should be switchable
%
\def\loop@redef@until#1{
\loop \mathcode\tempcountb\tempcountc
      \advance\tempcountb\@ne 
      \advance\tempcountc\@ne 
      \ifnum\tempcountb>#1\relax\else \repeat}
\ifini{
\newfam\VMIfam 
\tempcounta=-"100 \advance\tempcounta"\hexdigit\VMIfam00
\tempcountb=`a \tempcountc=\mathcode`a
\advance\tempcountc\tempcounta
\loop@redef@until{`z}
\tempcountb=`A \tempcountc=\mathcode`A
\advance\tempcountc\tempcounta
\loop@redef@until{`Z}
}

\restorecat"
\def\cal{\fam2}

% Typewriter is a bit of a problem
%
 \font\ttix=cmtt9
 \font\ttx=cmtt10 \font\ttxi=cmtt10 \sc@led \magstephalf
 \font\ttxii=cmtt10 \sc@led \magstep1
 \font\ttxiv=cmtt10 \sc@led \magstep2
 \font\ttxvii=cmtt10 \sc@led \magstep3
 \font\ttxx=cmtt10 \sc@led \magstep4

\def\tt{\Typeface:ComputerModern \Style:typewriter }
\def\terugstreep{{\tt \char`\\}}

% Typeface Definition
% \DefineTypeface{Naam}{rump}{style:exts}{size:exts}{combo:exts}
% after this \Typeface:Naam can be used
% the rump and the extension construct the external file names
%
\def\DefineTypeface#1#2#3#4{\csarg\def{#1@code}{#2}
    \csarg\def{#1@styles}{#3 }
    \csarg\def{#1@sizes}{#4 }
    \if\NextChar[%]
        {\DefineTypeface@combos{#1}}{\DefineTypeface@combos{#1}[]}}
\def\DefineTypeface@combos#1[#2]{\csarg\def{#1@combos}{#2}}

\def\style@selector{\mswitch \xp\if\xp\EqualStringX\xp{\F@style} in: }
\def\a@style@selector#1{\mswitch \xp\if\xp\EqualStringX\xp{#1} in: }
\def\size@selector{\fswitch {\ifnum\Fsize} in: }
\def\a@size@selector#1{\fswitch {\ifnum#1} in: }

\def\ext@combo{\csname\F@name @combos\endcsname}
\def\ext@style{\xp\xp\xp\style@selector
               \csname\F@name @styles\endcsname\endswitch}
\def\ext@size {\xp\xp\xp\size@selector
               \csname\F@name @sizes\endcsname\endswitch}
\def\ext@name{\csname\F@name @code\endcsname}

%\noeol
\endinput

% 92/11/05 math family setting
% 93/07/03 removed \bold, \italic etc because of change in tools
% 93/07/24 F@size -> Fsize, fontdefs file split off