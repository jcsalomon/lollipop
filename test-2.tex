\StyleDefinition

%\input comment

%\input tools
%\input document
%\input define
%\input fonts

%%%% Typeface
%\Trace:font

\Typeface:Computer
\PointSize:12
%\tracingmacros=2 \tracingcommands\tracingmacros
\Style:roman

%%%% Distance parameters
\Distance:parindent=1em
\Distance:basicindent=30pt
\Distance:halfindent=13pt
\Distance:whitebefore={11pt plus 2pt}
\Distance:whiteafter=7pt
\Distance:parskip={6pt plus 2pt minus 1pt}

%%%% Define table of contents
\DefineExternalFile:Contents=toc
\WriteContents:yes

%%%% Define headings and how they appear in the toc
\DefineHeading:Section counter:1 whitebefore:18pt whiteafter:15pt
    breakbefore:yes
    Pointsize:14 Style:bold
    line:start block:start SectionCounter literal:. Spaces:1.5 stickout:left 
               title 
        line:stop
    label:start command:SectionCounter literal:, Spaces:1 
        Style:italic title
        label:stop
    external:Contents title external:stop
    Stop
\macro:TitleNPage=title Spaces:2 Style:italic Page \par
\DefineExternalItem:Section file:Contents Style:bold
    item:left Style:roman SectionLabel item:stop
    macro:TitleNPage
    Stop

\DefineHeading:SubSection counter:i
    block:start SectionCounter literal:, SubSectionCounter literal:.
          fillupto:basicindent title
    external:Contents title external:stop
    Stop
\GoverningCounter:SubSection=Section
\DefineExternalItem:SubSection  file:Contents Style:italic
    item:right SectionLabel literal:. 
               SubSectionLabel Spaces:1 item:stop
    macro:TitleNPage
    Stop

\DefineHeading:SubSubSection counter:a
    block:start SectionCounter literal:, SubSectionCounter literal:,
                SubSubSectionCounter literal:.
          fillupto:basicindent title
    external:Contents title external:stop
    Stop
\GoverningCounter:SubSubSection=SubSection
\DefineExternalItem:SubSubSection file:Contents PushListLevel Style:roman
    item:left SubSubSectionLabel item:stop
    macro:TitleNPage
    Stop

%%%% Text blocks
\DefineTextBlock:IndentBlock whitebefore:0pt whiteafter:0pt
    whiteleft:basicindent
    Stop

\DefineTextBlock:DisplayEq whitebefore:abovedisplayskip
    whiteafter:belowdisplayskip whiteleft:basicindent
    test:GvtWarningExists
        Style:italic Substitute:GvtWarning command:nl
        test:stop
    literal:$ command:displaystyle text literal:$
    Stop
\DefineSubstitution:GvtWarning

\DefineTextBlock:EqNum embedded:yes whitebefore:{0pt plus 1fill}
    whiteafter:10pt counter:1 Pointsize:8 Style:bold
    literal:( EqNumCounter literal:) command:>
    label:start Style:roman command:SectionCounter
                literal:--
                Style:bold  command:EqNumCounter
    label:stop
    Stop

%%%% Lists
\macro:listwhite=whitebefore:0pt whiteafter:0pt

\DefineList:Enumerate counter:1 macro:listwhite
    item:left Style:italic itemCounter literal:. item:stop
    Stop
\DefineList:Itemize macro:listwhite 
    item:left itemsign item:stop
    Stop
\DefineList:Describe counter:i macro:listwhite
    item:left Style:roman itemCounter literal:. Spaces:.75
              Style:bold description Spaces:1 item:stop
    Stop

\WriteExtern:yes

\StyleDefinitionStop

%%%%%%%%%%%%%%%%
\Start

% remove the next two lines and see what the difference in output is
\GvtWarning The following message is only meant
for the `oplettende lezertjes'.

\Section First section

Some text comes before section~\ref[s:sec] on page~\pgref[s:sec].
\IndentBlock blah blah\par rubarb rubarb\par blah blah\nl blah blah\>
Still more nonsense.
%\tracingmacros=2 \tracingcommands\tracingmacros
\Ver>#$&}}}<Rev
And nonsense.

\begingroup
\BaselineSkip:13pt
This is a paragraph with artificially increased baselineskip (13pt)
for a given pointsize. 
This is a paragraph with artificially increased baselineskip
for a given pointsize. 
This is a paragraph with artificially increased baselineskip
for a given pointsize. \bigskip
\endgroup

\begingroup
\Distance:linedistance=14pt \BaselineSkip:linedistance
This is a paragraph with artificially increased baselineskip (14pt)
for a given pointsize. 
This is a paragraph with artificially increased baselineskip
for a given pointsize. 
This is a paragraph with artificially increased baselineskip
for a given pointsize. \bigskip
\endgroup

\begingroup
\PointSizeBaselineSkip:15
This is a paragraph with artificially increased baselineskip (15pt)
for a given pointsize. 
This is a paragraph with artificially increased baselineskip
for a given pointsize. 
This is a paragraph with artificially increased baselineskip
for a given pointsize. \bigskip
\endgroup

\begingroup
\Value:othersize=16 \PointSizeBaselineSkip:othersize
This is a paragraph with artificially increased baselineskip (16pt)
for a given pointsize. 
This is a paragraph with artificially increased baselineskip
for a given pointsize. 
This is a paragraph with artificially increased baselineskip
for a given pointsize. \bigskip
\endgroup

\begingroup
\PointSize:17 \SetFont
   \SetPointSizeBaselineSkip:10=baselineskip
\PointSize:10 \SetFont
This is a paragraph with artificially increased baselineskip (17pt)
for a given pointsize. 
This is a paragraph with artificially increased baselineskip
for a given pointsize. 
This is a paragraph with artificially increased baselineskip
for a given pointsize. \bigskip
\endgroup

\Section[s:sec] Second section

\SubSection A sub section

{\PointSize:15 \Style:italic 
More \PointSizeLarger Font \PointSizeLarger[2] Fiddlings
\PointSizeSmaller Flying} Blah
\DisplayEq M: \sum_{n=1}^\infty 1/n^2={\pi^2\over6} \EqNum\>
Halb.
\DisplayEq e^{\pi i}+1=0 \EqNum[euler:eq]\>
and
\DisplayEq 1+1=2 \EqNum\>
are formul\ae. In particular \ref[euler:eq] is cute one.

\SubSubSection Three deep\par Yes!
\SubSubSection And 3 again\par Yaaazzz!!!

\Section A nother section\par
\SubSection A nother sub section

Some more text.

\Enumerate
\item[it:one] and a list, last item has number~\ref[it:three]
\Itemize \PopListLevel
\item hum
\item tee dum\>
\item[it:two] with an item
\Itemize 
\item hum
\item tee dum\>
\item[it:three] or two with previous item~\ref[it:two].
\Enumeratestop
And again text.
\Itemize
\item once
\item and for all
\>
This works too!\par This works too!\nl
And again text.
\Itemize \PushListLevel
\item once
\item and for all
\>
Phew!
\Itemize \item hop
\Enumerate \item hop
\Itemize \item hop
\Enumerate \item hop
\Itemize \item hop
\Enumerate \item hop
\Itemize \item hop
\Itemize \item hop
\>\>\>\>\>\>\>\>

\Section Third section

\SubSection A sub section

Blah
\Describe
\item[MP] Do, a deer, a female deer
and what more could we say of that?
\item Re
and this one I really can't say anything about. Except that
it's a legal term, and that mr. Fowler disapproves
of any other use of it.
\item {}
No need to repeat the doe of item~\ref[MP].
\>
Blah blah.

\SubSection A sub section

Some more text.

\LoadExternalFile:Contents

\Stop

to be done:
references I-2.3 / 2.3 depending on whether same chapter or other

